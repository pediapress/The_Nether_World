\hypertarget{headerContainer}{}
\hypertarget{navigationHeader}{}
\protect\hypertarget{headerprevious}{}{←\href{/wiki/The_Nether_World/Volume_2/Chapter_11}{Volume
2, Chapter 11}}

\textbf{\protect\hypertarget{header_title_text}{}{\href{/wiki/The_Nether_World}{The
Nether World}}} \emph{by
\href{/wiki/Author:George_Gissing}{\protect\hypertarget{header_author_text}{}{{George
Gissing}}}}\\
\protect\hypertarget{header_section_text}{}{Volume 2, Chapter 12}

\protect\hypertarget{headernext}{}{\href{/wiki/The_Nether_World/Volume_2/Chapter_13}{Volume
2, Chpater 13}→}

\hypertarget{navigationNotes}{}

\hypertarget{ws-data}{}
\protect\hypertarget{ws-article-id}{}{2032005}\protect\hypertarget{ws-title}{}{\href{/wiki/The_Nether_World}{The
Nether World} --- \emph{Volume 2, Chapter
12}}\protect\hypertarget{ws-author}{}{George Gissing}

{\protect\hypertarget{245}{}{}}

{CHAPTER XII.}

A DOUBLE CONSECRATION.

\textsc{Bessie Byass} and her husband had, as you may suppose, devoted
many an hour to intimate gossip on the affairs of their top-floor
lodgers. Having no relations with Clerkenwell Close, they did not even
hear the rumours which spread from Mrs. Peckover's house at the time of
Jane's departure thence; their curiosity, which only grew keener as time
went on, found no appeasement save in conjecture. That Sidney Kirkwood
was in the secret from the first they had no doubt; Bessie made a sly
attempt now and then to get a hint from him, but without the least
result. The appearance on the scene of Jane's father revived their
speculation, and just after the old man's illness in the month of August
occurred something which gave them still fresh matter for
{\protect\hypertarget{246}{}{}}argument. The rooms on the first floor
having become vacant, Michael proposed certain new anrrangements. His
own chamber was too much that of an invalid to serve any longer as
sitting-room for Jane; he desired to take the front room below for that
purpose, to make the other on the same floor Jane's bedroom, and then to
share with the Byasses the expense of keeping a servant, whose lodging
would be in the chamber thus set free. Hitherto Bessie and Jane and an
occasional charwoman had done all the work of the house; it was a day of
jubilation for Mrs. Byass when she found herself ruling over a capped
and aproned maid. All these things set it beyond doubt that Michael
Snowdon had means greater than one would have supposed from his way of
living hitherto. Jane's removal from work could, of course, be explained
by her grandfather's growing infirmities, but Bessie saw more than this
in the new order of things; she began to look upon the girl with a
certain awe, as one whose future might reveal marvels.

For Jane, as we know, the marvels had
{\protect\hypertarget{247}{}{}}already begun. She came back from Danbury
not altogether like herself; unsettled a little, as it appeared; and
Michael's illness, befalling so soon, brought her into a nervous state
such as she had not known for a long time. The immediate effect of the
disclosure made to her by Michael whilst he was recovering was to
overwhelm her with a sense of responsibilities, to throw her mind into
painful tumult. Slow of thought, habituated to the simplest views of her
own existence, very ignorant of the world beyond the little circle in
which her life had been passed, she could not at once bring into the
control of her reflection this wondrous future to which her eyes had
been opened. The way in which she had been made acquainted with the
facts was unfortunate. Michael Snowdon, in spite of his deep affection
for her, and of the trust he had come to repose in her character, did
not understand Jane well enough to bring about this revelation with the
needful prudence. Between him, a man burdened with the sorrowful
memories of a long life, originally of stern
{\protect\hypertarget{248}{}{}}temperament, and now, in the feebleness
of his age, possessed by an enthusiasm which in several respects
disturbed his judgment, which made him desperately eager to secure his
end now that he felt life slipping away from him, --- between him and
such a girl as Jane there was a wider gulf than either of them could be
aware of. Little as he desired it, he could not help using a tone which
seemed severe rather than tenderly trustful. Absorbed in his great idea,
conscious that it had regulated every detail in his treatment of Jane
since she came to live with him, he forgot that the girl herself was by
no means adequately prepared to receive the solemn injunctions which he
now delivered to her. His language was as general as were the ideas of
beneficent activity which he desired to embody in Jane's future; but
instead of inspiring her with his own zeal, he afflicted her with
grievous spiritual trouble. For a time she could only feel that
something great and hard and high was suddenly required of her; the old
man's look seemed to keep repeating, ``Ai'e you worthy?''
{\protect\hypertarget{249}{}{}}The tremor of bygone days came back upon
her as she listened, the anguish of timidity, the heart-sinking, with
which she had been wont to strain her attention when Mrs. Peckover or
Clem imposed a harsh task.

One thing alone had she grasped as soon as it was uttered; one word of
reassurance she could recall when she sat down in solitude to collect
her thoughts. Her grandfather had mentioned that Sidney Kirkwood abeady
knew this secret. To Sidney her whole being turned in this hour of
distress; he was the friend who would help her with counsel and teach
her to be strong. But hereupon there revived in her a trouble which for
the moment she had forgotten, and it became so acute that she was driven
to speak to Michael in a way which had till now seemed impossible. When
she entered his room,---it was the morning after their grave
conversation,---Michael welcomed her with a face of joy, which, however,
she still felt to be somewhat stern and searching in its look. When they
had talked for a few moments, Jane said:

{\protect\hypertarget{250}{}{}}``I may speak about this to Mr. Kirkwood,
grandfather?''

``I hope you will, Jane. Strangers needn't know of it yet, but we can
speak freely to him.'' After many endeavours to find words that would
veil her thought, she constrained herself to ask:

``Does he think I can be all you wish?''

Michael looked at her, with a smile.

``Sidney has no less faith in you than I have, be sure of that.''

``I've been thinking---that perhaps he distrusted me a little.''

``Why, my child?''

``I don't quite know. But there's been a little difference in him, I
think, since we came back.''

Michael's countenance fell.

``Difference? How?''

But Jane could not go further. She wished she had not spoken. Her face
began to grow hot, and she moved away.

``It's only your fancy.'' continued Michael.

{\protect\hypertarget{251}{}{}}``But may be {that{{------}}.} You think
he isn't quite so easy in his talking to you as he was?'

``I've fancied it. But it was {only''{{------}}}

``Well, you may be partly right,'' said her grandfather, softening his
voice. "See, Jane, I'll tell you something. I think there's no harm;
perhaps I ought to. You must know that I hadn't meant to speak to Sidney
of these things just when I did. It came about, because \emph{he} had
something to tell \emph{me}, and something I was well pleased to hear.
It was about you, Jane, and in that way I got talking,---something about
you, my child. Afterwards, I asked him whether he wouldn't speak to you
yourself, but he said no,---not till you'd heard all that was before
you. I think I understood him, and I daresay you will, if you think it
over.''

Matter enough for thinking over, in these words. Did she understand them
aright? Before leaving the room she had not dared to look her
grandfather in the face, but she knew well that he was regarding her
still with the same smile. Did she understand him aright?

{\protect\hypertarget{252}{}{}}Try to read her mind. The world had all
at once grown very large, a distress to her imagination; worse still,
she had herself become a person of magnified importance, irrecognisable
in her own sight, moving, thinking so unnaturally. Jane, I assure you,
had thought very little of herself hitherto,---in both senses of the
phrase. Joyous because she could not help it, full of gratitude,
admiration, generosity, she occupied her thoughts very much with other
people, but knew not self-seeking, knew not self-esteem. The one thing
affecting herself over which she mused frequently was her suffering as a
little thrall in Clerkenwell Close, and the result was to make her very
humble. She had been an ill-used, ragged, work-worn child, and something
of that degradation seemed, in her feeling, still to cling to her. Could
she have known Bob Hewett's view of her position, she would have felt
its injustice, but at the same time would have bowed her head. And in
this spirit had she looked up to Sidney Kirkwood, regarding him as when
she was a child, save for that subtle
{\protect\hypertarget{253}{}{}}modification which began on the day when
she brought news of Clara Hewett's disappearance. Perfect in kindness,
Sidney had never addressed a word to her which implied more than
friendship,---never until that evening at the farm; then for the first
time had he struck a new note. His words seemed spoken with the express
purpose of altering his and her relations to each other. So much Jane
had felt, and his change since then was all the more painful to her, all
the more confusing. Now that of a sudden she had to regard herself in an
entirely new way, the dearest interest of her life necessarily entered
upon another phase. Struggling to understand how her grandfather could
think her worthy of such high trust, she inevitably searched her mind
for testimony as to the account in which Sidney held her. A fearful hope
had already flushed her cheeks before Michael spoke the words which
surely could have but one meaning.

On one point Sidney had left her no doubts; that his love for Clara
Hewett was a thing of the past he had told her distinctly. And why
{\protect\hypertarget{254}{}{}}did he wish her to be assured of that?
Oh, had her grandfather been mistaken in those words he reported? Durst
she put faith in them, coming thus to her by another's voice?

Doubts and dreads and self-reproofs might still visit her from hour to
hour, but the instinct of joy would not allow her to refuse admission to
this supreme hope. As if in spite of herself, the former gladness---nay,
a gladness multiplied beyond conception---reigned once more in her
heart. Her grandfather would not speak lightly in such a matter as this;
the meaning of his words was confessed, to all eternity immutable. Had
it, then, come to this? The friend to whom she looked up with such
reverence, with voiceless gratitude, when he condescended to speak
kindly to \emph{her}, the Peckovers' miserable little servant,---he,
after all these changes and chances of life, sought her now that she was
a woman, and had it on his lips to say that he loved her. Hitherto the
impossible, the silly thought to be laughed out of her head, the desire
for which she would have chid herself durst she
{\protect\hypertarget{255}{}{}}have faced it seriously,---was it become
a very truth? ``Keep a good heart, Jane; things'll be better some day.''
How many years since the rainy and windy night when he threw his coat
over her and spoke those words? Yet she could hear them now, and the
tears that rushed to her eyes as she blessed him for his manly goodness
were as much those of the desolate child as of the full-hearted woman.

And the change that she had observed in him since that evening at
Danbury? A real change, but only of manner. He would not say to her what
he had meant to say until she knew the truth about her own
circumstances. In simple words, she being rich and he having only what
he earned by his daily work, Sidney did not think it right to speak
whilst she was still in ignorance. The delicacy of her instincts, and
the sympathies awakened by her affection, made this perfectly clear to
her, strange and difficult to grasp as the situation was at first. When
she understood, how her soul laughed with exulting merriment!
Consecration to a great {\protect\hypertarget{256}{}{}}idea, endowment
with the means of wide beneficence,---this not only left her cold, but
weighed upon her, afflicted her beyond her strength. What was it, in
truth, that restored her to herself and made her heart beat joyously?
Knit your brows against her; shake your head and raze her name from that
catalogue of saints whereon you have inscribed it in anticipation. Jane
rejoiced simply because she loved a poor man, and had riches that she
could lay at his feet.

Great sums of money, vague and disturbing to her imagination when she
was bidden hold them in trust for unknown people, gleamed and made music
now that she could think of them as a gift of love. By this way of
thought she could escape from the confusion in which Michael's solemn
appeal had left her. Exalted by her great hope, calmed by the assurance
of aid that would never fail her, she began to feel the beauty of the
task to which she was summoned; the appalling responsibility became a
high privilege now that it was to be shared with one in whose
{\protect\hypertarget{257}{}{}}wisdom and strength she had measureless
confidence. She knew now what wealth meant; it was a great and glorious
power, a source of blessings incalculable. This power it would be hers
to bestow, and no man more worthy than he who should receive it at her
hands.

It was not without result that Jane had been so long a listener to the
conversations between Michael and Kirkwood. Defective as was her
instruction in the ordinary sense, those evenings spent in the company
of the two men had done much to refine her modes of thought. In spite of
the humble powers of her mind and her narrow experience, she had learned
to think on matters which are wholly strange to girls of her station, to
regard the life of the world and the individual in a light of idealism
and with a freedom from ignoble association rare enough in any class.
Her forecast of the future to be spent with Sidney was pathetic in its
simplicity, but had the stamp of nobleness. Thinking of the past years,
she made clear to herself {\protect\hypertarget{258}{}{}}all the
significance of her training. In her general view of things, wealth was
naturally allied with education, but she understood why Michael had had
her taught so little. A wealthy woman is called a lady; yes, but that
was exactly what she was not to become. On that account she had gone to
work, when in reality there was no need for her to do so. Never must she
remove herself from the poor and the laborious, her kin, her care; never
must she forget those bitter sufferings of her childhood, precious as
enabling her to comprehend the misery of others for whom had come no
rescue. She saw, moreover, what was meant by Michael's religious
teaching, why he chose for her study such parts of the Bible as taught
the beauty of compassion, of service rendered to those whom the world
casts forth and leaves to perish. All this grew upon her, when once the
gladness of her heart was revived. It was of the essence of her being to
exercise all human and self-forgetful virtues, and the consecration to a
life of beneficence moved her profoundly now
{\protect\hypertarget{259}{}{}}that it followed upon consecration to the
warmer love\ldots{}.

When Sidney paid his next visit Jane was alone in the new sitting-room;
her grandfather said he did not feel well enough to come down this
evening. It was the first time that Kirkwood had seen the new room.
After making his inquiries about Michael he surveyed the arrangements,
which were as simple as they could be, and spoke a few words regarding
the comfort Jane would find in them. He had his hand on a chair, but did
not sit down, nor lay aside his hat. Jane suffered from a constraint
which she had never before felt in his presence.

``You know what grandfather has been telling me?'' she said at length,
regarding him with grave eyes.

``Yes. He told me of his intention.''

``I asked him if I might speak to you about it. It was hard to
understand at first.''

``It would be, I've no doubt.''

Jane moved a little, took up some sewing,
{\protect\hypertarget{260}{}{}}and seated herself. Sidney let his hat
drop on to the chair, but remained standing, his arms resting on the
back.

``It's a very short time since I myself knew of it,'' he continued.
``Till then, I as little imagined as you did {that''{{------}}.} He
paused, then resumed more quickly, ``But it explains many things which I
had always understood in a simpler way.''

``I feel, too, that I know grandfather much better than I did,'' Jane
said. " He's always been thinking about the time when I should be old
enough to hear what plans he'd made for me. I do so hope he really
trusts me, Mr. Kirkwood! I don't know whether I speak about it as he
wishes. It isn't easy to say all I think, but I mean to do my best to be
what {he"{{------}}}

``He knows that very well. Don't be anxious; he feels that all his hopes
have been realised in you.''

There was silence. Jane made a pretence of using her needle, and Sidney
watched her hands.

{\protect\hypertarget{261}{}{}}``He spoke to you of a lady called Miss
Lant?'' were his next words.

``Yes. He just mentioned her.''

``Are you going to see her soon?''

``I don't know. Have \emph{you} seen her?''

``No. But I believe she's a woman you could soon be friendly with. I
hope your grandfather will ask her to come here before long.''

``I'm rather afraid of strangers.''

``No doubt,'' said the other, smiling. ``But you'll get over that. I
shall do my best to persuade Mr. Snowdon to make you acquainted with
her.''

Jane drew in her breath uneasily.

``She won't want me to know other people, I hope?''

``Oh, if she does, they'll be kind and nice and easy to talk to.''

Jane raised her eyes and said half-laughingly:

``I feel as if I was very childish, and that makes me feel it still
more. Of course, if it's necessary, I'll do my best to talk to
strangers. {\protect\hypertarget{262}{}{}}But they won't expect too much
of me, at first? I mean, if they find me a little slow, they won't be
impatient?''

``You mustn't think that hard things are going to be asked of you.
You'll never be required to say or do anything that you haven't akeady
said and done many a time, quite naturally. Why, it's some time since
you began the kind of work of which your grandfather has been
speaking.''

``I have begun it? How?''

``Who has been such a good friend to Pennyloaf, and helped her as nobody
else could have done?''

``Oh, but that's nothing!''

Sidney was on the point of replying, but suddenly altered his intention.
He raised himself from the leaning attitude, and took his hat.

``Well, we'll talk about it another time,'' he said carelessly. ``I
can't stop long to-night, so 111 go up and see your grandfather.''

Jane rose silently.

``I'll just look in and say good-night before I go,'' Sidney added, as
he left the room.

{\protect\hypertarget{263}{}{}}He did so, twenty minutes after. When he
opened the door Jane was sewing busily, but it was only on hearing his
footsteps that she had so applied herself. He gave a friendly nod, and
departed.

Still the same change in his manner. A little while ago he would have
chatted freely and forgotten the time.

Another week, and Jane made the acquaintance of the lady whose name we
have once or twice heard, Miss Lant, the friend of old Mr. Percival. Of
middle age and with very plain features, Miss Lant had devoted herself
to philanthropic work; she had an income of a few hundred pounds, and
lived almost as simply as the Snowdons in order to save money for
charitable expenditure. Unfortunately the earlier years of her life had
been joyless, and in the energy which she brought to this self-denying
enterprise there was just a touch of excess, common enough in those who
have been defrauded of their natural satisfactions and find a resource
in altruism. She was no pietist, but there is now-a-days
{\protect\hypertarget{264}{}{}}coming into existence a class of persons
who substitute for the old religious acerbity a narrow and oppressive
zeal for good works of purely human sanction, and to this order Miss
Lant might be said to belong. However, nothing but what was agreeable
manifested itself in her intercourse with Michael and Jane; the former
found her ardent spirit very congenial, and the latter was soon at ease
in her company.

It was a keen distress to Jane when she heard from Pennyloaf that Bob
would allow no future meetings between them. In vain she sought an
explanation; Pennyloaf professed to know nothing of her husband's
motives, but implored her friend to keep away for a time, as any
disregard of Bob's injunction would only result in worse troubles than
she yet had to endure. Jane sought the aid of Kirkwood, begging him to
interfere with young Hewett; the attempt was made, but proved fruitless.
\emph{``Sic volo, sic jubeo''} was Bob's standpoint, and he as good as
bade Sidney mind his own affairs.

{\protect\hypertarget{265}{}{}}Jane suffered, and more than she herself
would have anticipated. She had conceived a liking, almost an affection,
for poor, shiftless Pennyloaf, strengthened, of course, by the devotion
with which the latter repaid her. But something more than this injury to
her feelings was involved in her distress on being excluded from those
sorry lodgings. Pennyloaf was comparatively an old friend; she
represented the past, its contented work, its familiar associations, its
abundant happiness. And now, though Jane did not acknowledge to herself
that she regretted the old state of things, still less that she feared
the future, it was undeniable that the past seemed very bright in her
memory, and that something weighed upon her heart, forbidding such
gladsomeness as she had known.
