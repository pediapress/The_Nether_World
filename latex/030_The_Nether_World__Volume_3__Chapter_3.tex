{}

{CHAPTER III.}

ON A BARREN SHORE.

\textsc{About} this time Mr. Scawthorne received one morning a letter
which, though not unexpected, caused him some annoyance, and even
anxiety. It was signed ``C. V.,'' and made brief request for an
interview on the evening: of the next day at Waterloo Station.

The room in which our friend sat at breakfast was of such very modest
appearance that it seemed to argue but poor remuneration for the
services rendered by him in the office of Messrs. Percival \& Peel. It
was a parlour on the second floor of a lodging-house in Chelsea:
Scawthorne's graceful person and professional bearing were out of place
amid the trivial appointments. He lived here for the simple reason that
in order to enjoy a few of the luxuries of civilisation he had to spend
as little as possible on bare necessaries. His habits {}away from home
were those of a man to whom a few pounds are no serious consideration;
his pleasant dinner at the restaurant, his occasional stall at a
theatre, his easy acquaintance with easy livers of various kinds, had
become indispensable to him, and as a matter of course his expenditure
increased although his income kept at the same figure. That figure was
not contemptible, regard had to the path by which he had come thus far;
Mr Percival esteemed his abilities highly, and behaved to him with
generosity. Ten years ago Scawthorne would have lost his senses with joy
at the prospect of such a salary; to-day he found it miserably
insufficient to the demands he made upon life. Paltry debts harassed
him; inabilities fretted his temperament and his pride; it irked him to
have no better abode than this musty corner to which he could never
invite an acquaintance. And then, notwithstanding his mental endowments,
his keen social sense, his native tact, in all London not one refined
home was open to him, not one domestic circle of educated people could
he approach and find a welcome.

Scawthorne was passing out of the stage when a man seeks only the
gratification of his {}propensities; he began to focus his outlook upon
the world, and to feel the significance of maturity. The double
existence he was compelled to lead,---that of a laborious and
clear-brained man of business in office hours, that of a hungry rascal
in the time which was his own,---not only impressed him with a sense of
danger, but made him profoundly dissatisfied with the unreality of what
he called his enjoyments. What, he asked himself, had condemned him to
this kind of career? Simply the weight under which he started, his poor
origin, his miserable youth. However carefully regulated his private
life had been, his position to-day could not have been other than it
was; no degree of purity would have opened to him the door of a
civilised house. Suppose he had wished to marry; where, pray, was he to
find his wife? A barmaid? Why, yes, other men of his standing wedded
barmaids and girls from the houses of business, and so on; but they had
neither his tastes nor his brains. Never had it been his lot to exchange
a word with an educated woman,---save in the office on rare occasions.
There is such a thing as self-martyrdom in the cause of personal
integrity; {}another man might have said to himself, ``Providence
forbids me the gratification of my higher instincts, and I must be
content to live a life of barrenness, that I may at least be above
reproach.'' True, but Scawthorne happened not to be so made. He was of
the rebels of the earth. Formerly he revolted because he could not
indulge his senses to their full; at present his ideal was changed, and
the past burdened him.

Yesterday he had had an interview with old Mr. Percival which, for the
first time in his life, opened to him a prospect of the only kind of
advancement conformable with his higher needs. The firm of Percival \&
Peel was, in truth, Percival \& Son, Mr. Peel having been dead for many
years; and the son in question lacked a good deal of being the capable
lawyer whose exertions could supplement the failing energy of the senior
partner. Mr. Percival, having pondered the matter for some time, now
proposed that Scawthorne should qualify himself for admission as a
solicitor (the circumstances required his being under articles for three
years only), and then, if everything were still favourable, accept a
junior partnership in {}the firm. Such an offer was a testimony of the
high regard in which Scawthorne was held by his employer; it stirred him
with hope he had never dared to entertain since his eyes were opened to
the realities of the world, and in a single day did more for the
ripening of his prudence than years would have effected had his position
remained unaltered. Scawthorne realised more distinctly what a hazardous
game he had been playing.

And here was this brief note, signed ``C. Y.'' An ugly affair to look
back upon, all that connected itself with those initials. The worst of
it was, that it could not be regarded as done with. Had he anything to
fear from ``C. V.'' directly? The meeting must decide that. He felt now
what a fortunate thing it was that his elaborate plot to put an end to
the engagement between Kirkwood and Jane Snowdon had been accidentally
frustrated,---a plot which \emph{might} have availed himself nothing,
even had it succeeded. But was he, in his abandonment of rascality in
general, to think no more of the fortune which had so long kept his
imagination uneasy? Had he not, rather, a vastly better chance of
getting some of that money into his own {}pocket? It really seemed as if
Kirkwood---though he might be only artful---had relinquished his claim
on the girl, at all events for the present; possibly he was an honest
man, which would explain his behaviour. Michael Snowdon could not live
much longer; Jane would be the ward of the Percivals, and certainly
would be aided to a position more correspondent with her wealth. Why
should it then be impossible for \emph{him} to become Jane's husband?
Joseph, beyond a doubt, could be brought to favour that arrangement, by
means of a private understanding more advantageous to him than anything
he could reasonably hope from the girl's merely remaining unmarried.
This change in his relations to the Percivals would so far improve his
social claims that many of the difficulties hitherto besieging such a
scheme as this might easily be set aside. Come, come; the atmosphere was
clearing. Joseph himself, now established in a decent business, would
become less a fellow-intriguer than an ordinary friend bound to him, in
the way of the world, by mutual interests. Things must be put in order;
by some device the need of secrecy in his {}course with Joseph, must
come to an end. In fact, there remained but two hazardous points. Could
the connection between Jane and Kirkwood be brought definitely to an
end. And was anything to be feared from poor ``C. V.''?

Waterloo Station is a convenient rendezvous; its irregular form provides
many corners of retirement, out-of-the-way recesses where talk can be
carried on in something like privacy. To one of these secluded spots
Scawthorne drew aside with the veiled woman who met him at the entrance
from Waterloo Road. So closely was her face shrouded, that he had at
first a difficulty in catching the words she addressed to him. The noise
of an engine getting up steam, the rattle of cabs and porters' barrows,
the tread and voices of a multitude of people made fitting accompaniment
to a dialogue which in every word presupposed the corruptions and
miseries of a centre of modern life.

``Why did you send that letter to my father?'' was Clara's first
question.

``Letter? What letter?''

``Wasn't it you who let him know about me?''

{}``Certainly not. How should I have known his address? When I saw the
newspapers, I went down to Bolton and made inquiries. When I heard your
father had been, I concluded you had yourself sent for him. Otherwise, I
should, of course, have tried to be useful to you in some way. As it
was, I supposed you would scarcely thank me for coming forward.''

It might or might not be the truth, as far as Clara was able to decide.
Possibly the information had come from some one else. She knew him well
enough to be assured by his tone that nothing more could be elicited
from him on that point.

``You are quite recovered, I hope?'' Scawthorne added, surveying her as
she stood in the obscurity. ``In your general health?''

He was courteous, somewhat distant.

``I suppose I'm as well as I shall ever be,'' she answered coldly. ``I
asked you to meet me because I wanted to know what it was you spoke of
in your last letters. You got my answer, I suppose.''

``Yes, I received your answer. But---in fact, it's too late. The time
has gone by; and {}perhaps I was a little hasty in the hopes I held out.
I had partly deceived myself.''

``Never mind. I wish to know what it was,'' she said impatiently.

``It can't matter now Well, there's no harm in mentioning it. Naturally
you went out of your way to suppose it was something dishonourable.
Nothing of the kind; I had an idea that you might come to terms with an
Australian who was looking out for actresses for a theatre in
Melbourne,---that was all. But he wasn't quite the man I took him for. I
doubt whether it could have been made as profitable as I thought at
first.''

``You expect me to believe that story?''

``Not unless you like. It's some time since you put any faith in my
good-will. The only reason I didn't speak plainly was because I felt
sure that the mention of a foreign country would excite your suspicions.
You have always attributed evil motives to me rather than good. However,
this is not the time to speak of such things. I sympathise with
you---deeply. Will you tell me if I can---can help you at all?''

``No, you can't. I wanted to make quite {}sure that you were what I
thought you, that's all.''

``I don't think, on the whole, you have any reason to complain of
ill-faith on my part. I secured you the opportunities that are so hard
to find.''

``Yes, you did. We don't owe each other anything,---that's one comfort.
I'll just say that you needn't have any fear I shall trouble you in
future; I know that's what you're chiefly thinking about.''

``You misjudge me; but that can't be helped. I wish very much it were in
my power to be of use to you.''

``Thank you.''

On that last note of irony they parted. True enough, in one sense, that
there remained debt on neither side. But Clara, for all the fierce
ambition which had brought her life to this point, could not divest
herself of a woman's instincts. That simple fact explained various
inconsistencies in her behaviour to Scawthorne since she had made
herself independent of him; it explained also why this final interview
became the bitterest charge her memory preserved against him.

{}Her existence for some three weeks kept so gloomy a monotony that it
was impossible she should endure it much longer. The little room which
she shared at night with Annie and Amy was her cell throughout the day.
Of necessity she had made the acquaintance of Mrs. Eagles, but they
scarcely saw more of each other than if they had lived in different
tenements on the same staircase; she had offered to undertake a share of
the housework, but her father knew that everything of the kind was
distasteful to her, and Mrs. Eagles continued to assist Amy as hitherto.
To save trouble, she came into the middle room for her meals, at these
times always keeping as much of her face as possible hidden. The
children could not overcome a repulsion, a fear, excited by her veil and
the muteness she preserved in their presence; several nights passed
before little Annie got to sleep with any comfort. Only with her father
did Clara hold converse; in the evening he always sat alone with her for
an hour. She went out perhaps every third day, after dark, stealing
silently down the long staircase, and walking rapidly until she had
escaped the {}neighbourhood,---like John Hewett when formerly he
wandered forth in search of her. Her strength was slight; after
half-an-hour's absence she came back so wearied that the ascent of
stairs cost her much suffering.

The economy prevailing in to-day's architecture takes good care that no
depressing circumstance shall be absent from the dwellings in which the
poor find shelter. What terrible barracks, those Farringdon Road
Buildings! Vast, sheer walls, unbroken by even an attempt at ornament;
row above row of windows in the mud-coloured surface, upwards, upwards,
lifeless eyes, mirky openings that tell of bareness, disorder,
comfortlessness within. One is tempted to say that Shooter's Gardens are
a preferable abode. An inner courtyard, asphalted, swept
clean,---looking up to the sky as from a prison. Acres of these
edifices, the tinge of grime declaring the relative dates of their
erection; millions of tons of brute brick and mortar, crushing the
spirit as you gaze. Barracks, in truth; housing for the army of
industrialism, an army fighting with itself, rank against rank, man
against man, that the survivors may have {}whereon to feed. Pass by in
the night, and strain imagination to picture the weltering mass of human
weariness, of bestiality, of unmerited dolour, of hopeless hope, of
crushed surrender, tumbled together within those forbidding walls.

Clara hated the place from her first hour in it. It seemed to her that
the air was poisoned with the odour of an unclean crowd. The yells of
children at play in the courtyard tortured her nerves; the regular
sounds on the staircase, day after day repeated at the same hours,
incidents of the life of poverty, irritated her sick brain and filled
her with despair to think that as long as she lived she could never hope
to rise again above this world to which she was born. Gone for ever, for
ever, the promise that always gleamed before her whilst she had youth
and beauty and talent. With the one, she felt as though she had been
robbed of all three blessings; her twenty years were now a meaningless
figure; the energies of her mind could avail no more than an idiot's
mummery. For the author of her calamity she nourished no memory of
hatred: her resentment was {}against the fate which had cursed her
existence from its beginning.

For this she had dared everything, had made the supreme sacrifice.
Conscience had nothing to say to her, but she felt herself an outcast
even among these wretched toilers whose swarming aroused her disgust.
Given the success which had been all but in her grasp, and triumph ant
pride would have scored out every misgiving as to the cost at which the
victory had been won. Her pride was unbroken; under the stress of
anguish it became a scorn for goodness and humility; but in the
desolation of her future she read a punishment equal to the daring
wherewith she had aspired. Excepting her poor old father, not a living
soul that held account of her. She might live for years and years. Her
father would die, and then no smallest tribute of love or admiration
would be hers for ever. More than that; perforce she must gain her own
living, and in doing so she must expose herself to all manner of
insulting wonder and pity. Was it a life that could be lived?

Hour after hour she sat with her face buried in her hands. She did not
weep; tears were {}trivial before a destiny such as this. But groans and
smothered cries often broke the silence of her solitude,---cries of
frenzied revolt, wordless curses. Once she rose up suddenly, passed
through the middle room, and out on to the staircase; there a gap in the
wall, guarded by iron railings breast-high, looked down upon the
courtyard. She leaned forward over the bar and measured the distance
that separated her from the ground; a ghastly height! Surely one would
not feel much after such a fall? In any case, the crashing agony of but
an instant. Had not this place tempted other people before now?

Some one coming upstairs made her shrink back into her room. She had
felt the horrible fascination of that sheer depth, and thought of it for
days, thought of it until she dreaded to quit the tenement, lest a power
distinct from will should seize and hurl her to destruction. She knew
that that must not happen here; for all her self-absorption, she could
not visit with such cruelty the one heart that loved her. And thinking
of him, she understood that her father's tenderness was not wholly the
idle thing that it had been to her at first; her love could never
{}equal his, had never done so in her childhood, but she grew conscious
of a soothing power in the gentle and timid devotion with which he
tended her. His appearance of an evening was something more than a
relief after the waste of hours which made her day. The rough,
passionate man made himself as quiet and sympathetic as a girl when he
took his place by her. Compared with her, his other children were as
nothing to him. Impossible that Clara should not be touched by the sense
that he who had everything to forgive, whom she had despised and
abandoned, behaved now as one whose part it is to beseech forgiveness.
She became less impatient when he tried to draw her into conversation;
when he held her thin soft hand in those rude ones of his, she knew a
solace in which there was something of gratitude.

Yet it was John who revived her misery in its worst form. Pitying her
unoccupied loneliness, he brought home one day a book that he had
purchased from a stall in Farringdon Street; it was a novel (with a
picture on the cover which seemed designed to repel any person not
wholly without taste), and might perhaps serve the end of averting her
thoughts {}from their one subject. Clara viewed it contemptuously, but
made a show of being thankful, and on the next day she did glance at its
pages. The story was better than its illustration; it took a hold upon
her; she read all day long. But when she returned to herself, it was to
find that she had been exasperating her heart's malady. The book dealt
with people of wealth and refinement, with the world to which she had
all her life been aspiring, and to which she might have attained. The
meanness of her surroundings became in comparison more mean, the
bitterness of her fate more bitter. You must not lose sight of the fact
that since abandoning her work-girl existence Clara had been constantly
educating herself, not only by direct study of books, but, through her
association with people, her growth in experience. Where in the old days
of rebellion she had only an instinct, a divination to guide her, there
was now just enough of knowledge to give occupation to her developed
intellect and taste. Far keener was her sense of the loss she had
suffered than her former longing for what she knew only in dream. The
activity of her mind received a new {}impulse when she broke free from
Scawthorne and began her upward struggle in independence. Whatever books
were obtainable she read greedily; she purchased numbers of plays in the
acting-editions, and studied with the utmost earnestness such parts as
she knew by repute; no actress entertained a more superb ambition, none
was more vividly conscious of power. But it was not only at
stage-triumph that Clara aimed; glorious in itself, this was also to
serve her as a means of becoming nationalised among that race of beings
whom birth and breeding exalt above the multitude. A notable illusion;
pathetic to dwell upon. As a work-girl, she nourished envious hatred of
those the world taught her to call superiors; they were then as remote
and unknown to her as gods on Olympus. From her place behind the
footlights she surveyed the occupants of boxes and stalls in a changed
spirit; the distance was no longer insuperable; she heard of fortunate
players who mingled on equal terms with men and women of refinement.
There, she imagined, was her ultimate goal. "It is to \emph{them} that I
belong! Be my origin what it may, I have the intelligence and the
desires {}of one born to freedom. Nothing in me, nothing, is akin to
that gross world from which I have escaped!'' So she thought---with
every drop of her heart's blood crying its source from that red fountain
of revolt whereon never yet did the upper daylight gleam! Brain and
pulses such as hers belong not to the mild breed of mortals fostered in
sunshine. But for the stroke of fate, she might have won that reception
which was in her dream, and with what self-mockery when experience had
matured itself! Never yet did true rebel, who has burst the barriers of
social limitation, find aught but ennui in the trim gardens beyond.

When John asked if the book had given her amusement, she said that
reading made her eyes ache. He noticed that her hand felt feverish, and
that the dark mood had fallen upon her as badly as ever to-night.

``It's just what I said,'' she exclaimed with abruptness, after long
refusal to speak. ``I knew your friend would never come as long as I was
here.''

John regarded her anxiously. The phrase ``your friend'' had a peculiar
sound that disturbed him. It made him aware that she had {}been thinking
often of Sidney Kirk wood since his name had been dismissed from their
conversation. He, too, had often turned his mind uneasily in the same
direction, wondering whether he ought to have spoken of Sidney so
freely. At the time it seemed best, indeed almost inevitable; but habit
and the force of affection were changing his view of Clara in several
respects. He recognised the impossibility of her continuing to live as
now, yet it was as difficult as ever to conceive a means of aiding her.
Unavoidably he kept glancing towards Kirkwood. He knew that Sidney was
no longer a free man; he knew that, even had it been otherwise, Clara
could be nothing to him. In spite of facts, the father kept brooding; on
what might have been. His own love was perdurable: how could it other
than intensify when its object was so unhappy? His hot, illogical mood
all but brought about a revival of the old resentment against Sidney.

``I haven't seen him for a week or two,'' he replied, in an embarrassed
way.

``Did he tell you he shouldn't come?''

``No. After we'd talked about it, you know,---when you told me you
didn't mind,---I just {}said a word or two; and he nodded, that was
all.''

She became silent. John, racked by doubts as to whether he should say
more of Sidney or still hold his peace, sat rubbing the back of one hand
with the other and looking about the room.

``Father,'' Clara resumed presently, ``what became of that child at Mrs.
Peckover's, that her grandfather came and took away? Snowdon; yes, that
was her name; Jane Snowdon.''

``You remember they went to live with somebody you used to know,'' John
replied, with hesitation. ``They're still in the same house.''

``So she's grown up. Did you ever hear about that old man having a lot
of money?''

``Why, my dear, I never heard nothing but what them Peckovers talked at
the time. But there was a son of his turned up as seemed to have some
money. He married Mrs. Peckover's daughter.''

Clara expressed surprise.

``A son of his? Not the girl's father?''

``Yes; her father. I don't know nothing about his history. It's for him,
or partly for {}him, as I'm workin' now, Clara. The firm's Lake,
Snowdon, \& Co.''

``Why didn't you mention it before?''

``I don't hardly know, my dear.''

She looked at him, aware that something was being kept back.

``Tell me about the girl, what does she do?''

``She goes to work, I believe; but I haven't heard much about her since
a good time. Sidney Kirkwood's a friend of her grandfather. He often
goes there, I believe.''

``What is she like?'' Clara asked, after a pause. ``She used to be such
a weak, ailing thing, I never thought she'd grow up. What's she like to
look at?''

``I can't tell you, my dear. I don't know as ever I see her since those
times.''

Again a silence.

``Then it's Mr. Kirkwood that has told you what you know of her?''

``Why, no. It was chiefly Mrs. Peckover told me. She did say,
Clara,---but then I can't tell whether it's true or not,---she did say
something about Sidney and her.''

He spoke with difliculty, feeling constrained {}to make the disclosure,
but anxious as to its result. Clara made no movement, seemed to have
heard with indifference.

``It's maybe partly `cause of that,'' added John, in a low voice, ``that
he doesn't like to come here.''

``Yes; I understand.''

They spoke no more on the subject.
