\chapter{Woman and Actress}

\textsc{In} a tenement on the same staircase, two floors below, lived a
family with whom John Hewett was on friendly terms. Necessity calling
these people out of London for a few days, they had left with John the
key of their front door; a letter of some moment might arrive in their
absence, and John undertook to re-post it to them. The key was hung on a
nail in Clara's room.

``I'll just go down and see if the postman's left anything at Mrs.
Holland's this morning,'' said Amy Hewett, coming in between breakfast
and the time of starting for school.

She reached up to the key, but Clara, who sat by the fire with a cup of
tea on her lap, the only breakfast she ever took, surprised her by
saying, ``You needn't trouble, Amy. I shall be going out soon, and I'll
look in as I pass.''

{\protect\hypertarget{71}{}{}}The girl was disappointed, for she liked
this private incursion into the abode of other people, but the
expression of a purpose by her sister was so unusual that, after a
moment's hesitating, she said, ``Very well,'' and left the room again.

When silence informed Clara that the children were gone, Mrs. Eagles
being the only person besides herself who remained in the tenement, she
put on her hat, drew down the veil which was always attached to it, and
with the key in her hand descended to the Hollands' rooms. Had a letter
been delivered that morning, it would have been---in default of
box---just inside the door; there was none, but Clara seemed to have
another purpose in view. She closed the door and walked forward into the
nearest room; the blind was down, but the dusk thus produced was
familiar to her in consequence of her own habit, and, her veil thrown
back, she examined the chamber thoughtfully. It was a sitting-room,
ugly, orderly; the air felt damp, and even in semi-darkness she was
conscious of the layers of London dust which had softly deposited
themselves since the family went away forty-eight hours ago. A
{\protect\hypertarget{72}{}{}}fire was laid ready for lighting, and the
smell of moist soot spread from the grate. Having stood on one spot for
nearly ten minutes, Clara made a quick movement and withdrew; she
latched the front door with as little noise as possible, ran upstairs
and shut herself again in her own room.

Presently she was standing at her window, the blind partly raised. On a
clear day the view from this room was of wide extent, embracing a great
part of the City; seen under a low, blurred, dripping sky, through the
ragged patches of smoke from chimneys innumerable, it had a gloomy
impressiveness well in keeping with the mind of her who brooded over it.
Directly in front, rising mist-detached from the lower masses of
building, stood in black majesty the dome of St. Paul's; its vastness
suffered no diminution from this high outlook, rather was exaggerated by
the flying scraps of mirky vapour which softened its outline and at
times gave it the appearance of floating on a vague troubled sea.
Somewhat nearer, amid many spires and steeples, lay the surly bulk of
Newgate, the lines of its construction shown plan-wise; its
{\protect\hypertarget{73}{}{}}little windows multiplied for points of
torment to the vision. Nearer again, the markets of Smithfield,
Bartholomew's Hospital, the tract of modern deformity, cleft by a gulf
of railway, which spreads between Clerkenwell Koad and Charterhouse
Street. Down in Farringdon Street the carts, waggons, vans, cabs,
omnibuses, crossed and intermingled in a steaming splash-bath of mud;
human beings, reduced to their due paltriness, seemed to toil in
exasperation along the strips of pavement, bound on errands which were a
mockery, driven automaton-like by forces they neither understood nor
could resist.

``Can I go out into a world like that---alone?'' was the thought which
made Clara's spirit fail as she stood gazing. ``Can I face life as it is
for women who grow old in earning bare daily bread among those terrible
streets? Year after year to go in and out from some wretched garret that
I call home, with my face hidden, my heart stabbed with misery till it
is cold and bloodless!''

Then her eye fell upon the spire of St. James's Church, on Clerkenwell
Green, whose bells used to be so familiar to her. The
{\protect\hypertarget{74}{}{}}memory was only of discontent and futile
aspiration, but---Oh, if it were possible to be again as she was then,
and yet keep the experience with which life had since endowed her! With
no moral condemnation did she view the records of her rebellion; but how
easy to see now that ignorance had been one of the worst obstacles in
her path, and that, like all unadvised purchasers, she had paid a price
that might well have been spared. A little more craft, a little more
patience,---it is with these that the world is conquered. The world was
her enemy, and had proved too strong; woman though she was,---only a
girl striving to attain the place for which birth adapted
her,---pursuing only her irrepressible instincts,---fate flung her to
the ground pitilessly, and bade her live out the rest of her time in
wretchedness.

No! There remained one more endeavour that was possible to her, one bare
hope of saving herself from the extremity which only now she estimated
at its full horror. If that failed, why, then, there was a way to cure
all ills.

From her box, that in which were hidden away many heart-breaking
mementoes of her life as an actress, she took out a sheet of
{\protect\hypertarget{75}{}{}}notepaper and an envelope. ``Without much
thought, she wrote nearly three pages, folded the letter, addressed it
with a name only: ``Mr. Kirkwood.'' Sidney's address she did not know;
her father had mentioned Red Lion Street, that was all. She did not even
know whether he still worked at the old place, but in that way she must
try to find him. She cloaked herself, took her umbrella, and went out.

At a corner of St. John's Square she soon found an urchin who would run
an errand for her. He was to take this note to a house that she
indicated, and to ask if Mr. Kirkwood was working there. She scarcely
durst hope to see the messenger returning with empty hands, but he did
so. A terrible throbbing at her heart, she went home again.

In the evening, when her father returned, she surprised him by saying
that she expected a visitor.

``Do you want me to go out of the way?'' he asked, eager to submit to
her in everything.

``No. I've asked my friend to come to Mrs. Holland's. I thought there
would be no great harm. I shall go down just before nine o'clock.''

``Oh no, there's no harm,'' conceded her
{\protect\hypertarget{76}{}{}}father. ``It's only if the neighbours
opposite got talkin' to them when they come back.''

``I can't help it. They won't mind. I can't help it.''

John noticed her agitated repetition, the impatience with which she
flung aside difficulties.

``Clara,---it ain't anything about work, my dear?''

``No, father. I wouldn't do anything without telling you; I've
promised.''

``Then I don't care; it's all right.''

She had begun to speak immediately on his entering the room, and so it
happened that he had not kissed her as he always did at homecoming. When
she had sat down, he came with awkwardness and timidity and bent his
face to hers.

``What a hot cheek it is to-night, my little girl!'' he murmured. ``I
don't like it; you've got a bit of fever hangin' about you.''

She wished to be alone; the children must not come into the room until
she had gone downstairs. When her father had left her, she seated
herself before the looking-glass, abhorrent as it was to her to look
thus in her own face, and began dressing her hair with
{\protect\hypertarget{77}{}{}}quite unusual attention. This beauty at
least remained to her; arranged as she had learned to do it for the
stage, the dark abundance of her tresses crowned nobly the head which
once held itself with such defiant grace. She did not change her dress,
which, though it had suffered from wear, was well-fitting and of better
material than Farringdon Road Buildings were wont to see; a sober
draping which became her tall elegance as she moved. At a quarter to
nine she arranged the veil upon her head so that she could throw her hat
aside without disturbing it; then, taking the lamp in her hand, and the
key of the Hollands' door, she went forth.

No one met her on the stairs. She was safe in the cold deserted parlour
where she had stood this morning. Cold, doubtless, but she could not be
conscious of it; in her veins there seemed rather to be fire than blood.
Her brain was clear, but in an unnatural way; the throbbing at her
temples ought to have been painful, but only excited her with a strange
intensity of thought. And she felt, amid it all, a dread of what was
before her; only the fever, to which she abandoned herself with a
{\protect\hypertarget{78}{}{}}sort of reckless confidence, a faith that
it would continue till this interview was over, overcame an impulse to
rush back into her hiding-place, to bury herself in shame, or
desperately whelm her wretchedness in the final {oblivion{.~.~.~.}}

He was very punctual. The heavy bell of St. Paul's had not reached its
ninth stroke when she heard his knock at the door.

He came in without speaking, and stood as if afraid to look at her. The
lamp, placed on a side-table, barely disclosed all the objects within
the four walls; it illumined Sidney's face, but Clara moved so that she
was in shadow. She began to speak.

``You understood my note? The people who live here are away, and I have
ventured to borrow their room. They are friends of my father's.''

At the first word, he was surprised by the change in her voice and
accentuation. Her speech was that of an educated woman; the melody which
always had such a charm for him had gained wonderfully in richness. Yet
it was with difficulty that she commanded utterance, and her agitation
touched him in a way quite other than he was prepared for. In
{\protect\hypertarget{79}{}{}}truth, he knew not what experience he had
anticipated, but the reality, now that it came, this unimaginable
blending of memory with the unfamiliar, this refinement of something
that he had loved, this note of pity struck within him by such subtle
means, affected his inmost self. Immediately he laid stern control upon
his feelings, but all the words which he had designed to speak were
driven from memory. He could say nothing, could only glance at her
veiled face and await what she had to ask of him.

``Will you sit down? I shall feel grateful if you can spare me a few
minutes. I have asked you to see me because---indeed, because I am sadly
in want of the kind of help a friend might give me. I don't venture to
call you that, but I thought of you; I hoped you wouldn't refuse to let
me speak to you. I am in such difficulties---such a hard
position''{{------}}

``You may be very sure I will do anything I can to be of use to you,''
Sidney replied, his thick voice contrasting so strongly with that which
had just failed into silence that he coughed and lowered his tone after
the first few syllables. He meant to express himself
{\protect\hypertarget{80}{}{}}without a hint of emotion, but it was
beyond his power. The words in which she spoke of her calamity seemed so
pathetically simple that they went to his heart. Clara had recovered all
her faculties. The fever and the anguish and the dread were no whit
diminished, but they helped instead of checking her. An actress
improvising her part, she regulated every tone with perfect skill, with
inspiration; the very attitude in which she seated herself was a triumph
of the artist's felicity.

``I just said a word or two in my note,'' she resumed, ``that you might
have replied if you thought nothing could be gained by my speaking to
you. I couldn't explain fully what I had in mind. I don't know that I've
anything very clear to say even now, but---you know what has happened to
me; you know that I have nothing of to look forward to, that I can only
hope to keep from being a burden to my father. I am getting stronger;
it's time I tried to find somethins: to do. But I''{{------}}

Her voice failed again. Sidney gazed at her, and saw the dull lamplight
just glisten on her hair. She was bending forward a little, her hands
joined and resting on her knee.

{\protect\hypertarget{81}{}{}}``Have jou thought what kind of---of work
would be best for you?'' Sidney asked. The ``work'' stuck in his throat,
and he seemed to himself brutal in his way of uttering it. But he was
glad when he had put the question thus directly; one at least of his
resolves was carried out.

``I know I've no right to choose, when there's necessity,'' she
answered, in a very low tone. ``Most women would naturally think of
needlework; but I know so little of it; I scarcely ever did any. If I
could---I might perhaps do that at home, and I feel---if I could only
avoid---if I could only be spared going among strangers''{{------}}

Her faltering voice sank lower and lower; she seemed as if she would
have hidden her face even under its veil.

``I feel sure you will have no difficulty,'' Sidney hastened to reply,
his own voice unsteady. ``Certainly you can get work at home. Why do you
trouble yourself with the thought of going among strangers? There'll
never be the least need for that; I'm sure there won't. Haven't you
spoken about it to your father?''

{\protect\hypertarget{82}{}{}}``Yes. But he is so kind to me that he
won't hear of work at all. It was partly on that account that I took the
step of appealing to you. He doesn't know who I am meeting here
to-night. Would you---I don't know whether I ought to ask---but perhaps
if you spoke to him in a day or two, and made him understand how strong
my wish is. He dreads lest we should be parted, but I hope I shall never
have to leave him. And then, of course, father is not very well able to
advise me---about work, I mean. You have more experience. I am so
helpless. Oh, if you knew how helpless I feel!''

``If you really wish it, I will talk with your father''{{------}}

``Indeed, I do wish it. My coming to live here has made everything so
uncomfortable for him and the children. Even his friends can't visit him
as they would; I feel that, though he won't admit that it's made any
difference.''

Sidney looked to the ground. He heard her voice falter as it continued.

"If I'm to live here still, it mustn't be at the cost of all his
comfort. I keep almost {\protect\hypertarget{83}{}{}} always in the one
room. I shouldn't be in the way if any one came. I've been afraid, Mr.
Kirkwood, that perhaps you feared to come lest, whilst I was not very
well, it might have been an inconvenience to us. Please don't think
that. I shall never---see either friends or strangers unless it is
absolutely needful."

There was silence.

``You do feel much better, I hope?'' fell from Sidney's lips.

"Much stronger. It's only my mind; everything; is so dark to me. You
know how little patience I always had. It was enough if any one said,
'You \emph{must} do this,' or `You must put up with that,'---at once I
resisted. It was my nature; I couldn't bear the feeling of control.
That's what I've had to struggle with since I recovered from my delirium
at the hospital, and hadn't even the hope of dying. Can you put yourself
in my place, and imagine what I have suffered?"

Sidney was silent. His own life had not been without its passionate
miseries, but the modulations of this voice which had no light of
countenance to aid it raised him above the
{\protect\hypertarget{84}{}{}} plane of common experience and made
actual to him the feelings he knew only in romantic story. He could not
stir, lest the slightest sound should jar on her speaking. His breath
rose visibly upon the chill air, but the discomfort of the room was as
indifferent to him as to his companion. Clara rose, as if impelled by
mental anguish; she stretched out her hand to the mantelpiece, and so
stood, between him and the light, her admirable figure designed on a
glimmering background.

``I know why you say nothing,'' she continued, abruptly but without
resentment. ``You cannot use words of sympathy which would be anything
but formal, and you prefer to let me understand that. It is like you.
Oh, you mustn't think I mean the phrase as a reproach. Anything but
that. I mean that you were always honest, and time hasn't changed
you---in that.'' A slight, very slight, tremor on the close. ``I'd
rather you behaved to me like \^{}''our old self A sham sympathy would
drive me mad."

``I said nothing,'' he replied, " only because words seemed
meaningless."

"Not only that. You feel for me, I know. {\protect\hypertarget{85}{}{}}
because you are not heartless; but at the same time you obey your
reason, which tells you that all I suffer comes of my own self-will."

``I should like you to think better of me than that. I'm not one of
those people, I hope, who use every accident to point a moral, and begin
by inventing the moral to suit their own convictions. I know all the
details of your misfortune.''

``Oh, wasn't it cruel that she should take such revenge upon me!'' Her
voice rose in unrestrained emotion. "Just because she envied me that
poor bit of advantage over her~! How could I be expected to refuse the
chance that was offered? It would have been no use; she couldn't have
kept the part. And I was so near success. I had never had a chance of
showing what I could do. It wasn't much of a part, really, but it was
the lead, at all events, and it would have made people pay attention to
me. You don't know how strongly I was always drawn to the stage; there I
found the work for which I was meant. And I strove so hard to make my
way. I had no friends, no money. I earned only just enough to supply my
needs. I know what {\protect\hypertarget{86}{}{}} people think about
actresses. Mr. Kirkwood, do you imagine I have been living at my ease,
congratulating myself that I had escaped from all hardships?"

He could not raise his eyes. As she still awaited his answer, he said in
rather a hard voice:

``As I have told you, Ire ad all the details that were published.''

``Then you know that I was working hard and honestly, --- working far,
far harder than when I lived in Clerkenwell Close. But I don't know why
I am talking to you about it. It's all over. I went my own way, and I
all but won what I fought for. You may very well say, what's the use of
mourning over one's fate?''

Sidney had risen.

``You were strong in your resolve to succeed,'' he said gravely, ``and
you will find strength to meet even this trial.''

``A weaker woman would suffer far less. One with a little more strength
of character would kill herself.''

"No. In that you mistake. You have not yourself only to think of. It
would be an {\protect\hypertarget{87}{}{}} easy thing to put an end to
your life. You have a duty to your father."

She bent her head.

``I think of him. He is goodness itself to me. There are fathers who
would have shut the door in my face. I know better now than I could when
I was only a child how hard his life has been~; he and I are like each
other in so many ways~; he has always been fighting against cruel
circumstances. It's right that you, who have been his true and helpful
friend, should remind me of my duty to him.''

A pause; then Sidney asked:

``Do you wish me to speak to him very soon about your finding
occupation?''

``If you will. If you could think of anything.''

He moved, but still delayed his offer to take leave.

``You said just now,'' Clara continued, falteringly, ``that you did not
try to express sympathy, because words seemed of no use. How am I to
find words of thanks to you for coming here and listening to what I had
to say?''

``But surely so simple an act of friendship''---
{\protect\hypertarget{88}{}{}} ``Have I so many friends? And what right
have I to look to you for an act of kindness? Did I merit it by my words
when I last''---

There came a marvellous chancre---a chancre such as it needed either
exquisite feeling or the genius of simulation to express by means so
simple. Unable to show him by a smile, by a light in her eyes, what mood
had come upon her, what subtle shifting in the direction of her thought
had checked her words,---by her mere movement as she stepped lightly
towards him, by the carriage of her head, by her hands half held out and
half drawn back again, she prepared him for what she was about to say.
No piece of acting was ever more delicately finished. He knew that she
smiled, though nothing of her face was visible; he knew that her look
was one of diffident, half-blushing pleasure. And then came the
sweetness of her accents, timorous, joyful, scarcely to be recognised as
the voice which an instant ago had trembled sadly in self-reproach.

``But that seems to you so long ago, doesn't it? You can forgive me now
Father has told me what happiness you have found, and I---I am so
glad!''

{\protect\hypertarget{89}{}{}} Sidney drew back a step, involuntarily;
the movement came of the shock with which he heard her make such
confident reference to the supposed relations between himself and Jane
Snowdon. He reddened---stood mute. For a few seconds his mind was in the
most painful whirl and conflict; a hundred impressions, arguments,
apprehensions, crowded upon him, each with its puncturing torment. And
Clara stood there waiting for his reply, in the attitude of consummate
grace.

``Of course I know what you speak of,'' he said at length, with the
bluntness of confusion. ``But your father was mistaken. I don't know who
can have led him to believe that It's a mistake, altogether.''

Sidney would not have believed that any one could so completely rob him
of self-possession, least of all Clara Hewett. His face grew still more
heated. He was angry with he knew not whom, he knew not why,---perhaps
with himself in the first instance.

``A mistake?'' Clara murmured, under her breath. "Oh, you mean people
have been too hasty in speaking about it. Do pardon me.
{\protect\hypertarget{90}{}{}} I ought never to have taken such a
liberty,---but I felt " She hesitated.

``It was no liberty at all. I dare say the mistake is natural enough to
those who know nothing of of Miss Snowdon's circumstances. I myself,
however, have no right to talk about her. But what you have been told is
absolute error.''

Clara walked a few paces aside.

``Again I ask you to forgive me.'' Her tones had not the same clearness
as hitherto. ``In any case, I had no right to approach such a subject in
speaking with you.''

``Let us put it aside,'' said Sidney, mastering himself " We were just
agreeing that I should see your father, and make known your wish to
him."

``Thank you. I shall tell him, when I go upstairs, that you w\^{}ere the
friend whom I had asked to come here. I felt it to be so uncertain
whether you would come.''

``I hope you couldn't seriously doubt it.''

``You teach me to tell the truth. No. I knew too well your kindness. I
knew that even to me''--- {\protect\hypertarget{91}{}{}} Sidney could
converse no longer. He felt the need of being alone, to put his thoughts
in order, to resume his experiences during this strange hour. An extreme
weariness was possessing him, as though he had been straining his
intellect in attention to some difficult subject. And all at once the
dank, cold atmosphere of the room struck into his blood; he had a fit of
trembling.

``Let us say good-bye for the present.''

Clara gave her hand silently. He touched it for the first time, and
could not but notice its delicacy; it was very warm, too, and moist.
Without speaking she wxnt with him to the outer door. His footsteps
sounded along the stone staircase; Clara listened until the last echo
was silent.

She too had begun to feel the chilly air. Hastily putting on her hat,
she took up the lamp, glanced round the room to see that nothing was
left in disorder, and hastened up to the fifth storey.

In the middle room, through which she had to pass, her father and Mr.
Eagles were talking together. The latter gave her a ``good-evening,''
respectful, almost as to a social {\protect\hypertarget{92}{}{}}
superior. Within, Amy and Annie were just going to bed. She sat with
them in her usual silence for a quarter of an hour, then, having
ascertained that Eagles was gone into his own chamber, went out to speak
to her father.

``My friend came,'' she said. *'Do you suspect who it was?"

``Why, no, I can't guess, Clara.''

``Haven't you thought of Mr. Kirk wood?''

``You don't mean that?''

``Father, you are quite mistaken about Jane Snowdon---quite.''

John started up from his seat.

``Has he told you so, himself?''

``Yes. But listen; you are not to say a word on that subject to him. You
will be very careful, father?''

John gazed at her wonderingly. She kissed his forehead, and withdrew to
the other room.
