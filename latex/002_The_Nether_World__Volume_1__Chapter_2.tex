\chapter{A Friend in Request}

It was the hour of the unyoking of men. In the highways and byways of
Clerkenwell there was a thronging of released toilers, of young and old,
of male and female. Forth they streamed from factories and work-rooms,
anxious to make the most of the few hours during which they might live
for themselves. Great numbers were still bent over their labour, and
would be for hours to come, but the majority had leave to wend
stablewards. Along the main thoroughfares the wheel-track was
clangorous; every omnibus that clattered by was heavily laden with
passengers; tarpaulins gleamed over the knees of those who sat outside.
This way and that the lights were blurred into a misty radiance;
overhead was mere {\protect\hypertarget{24}{}{}}darkness, whence
descended the lashing rain. There was a ceaseless scattering of mud;
there were blocks in the traffic, attended with rough jest or angry
curse; there was jostling on the crowded pavement. Public-houses began
to brighten up, to bestir themselves for the evening's business. Streets
that had been hives of activity since early morning were being abandoned
to silence and darkness and the sweeping wind.

At noon to-day there was sunlight on the Surrey hills; the fields and
lanes were fragrant with the first breath of spring, and from the
shelter of budding copses many a primrose looked tremblingly up to the
vision of blue sky. But of these things Clerkenwell takes no count; here
it had been a day like any other, consisting of so many hours, each
representing a fraction of the weekly wage. Go where you may in
Clerkenwell, on every hand are multiform evidences of toil, intolerable
as a nightmare. It is not as in those parts of London where the main
thoroughfares consist of shops and warehouses and
{\protect\hypertarget{25}{}{}}workrooms, whilst the streets that are
hidden away on either hand are devoted in the main to dwellings. Here
every alley is thronged with small industries; all but every door and
window exhibits the advertisement of a craft that is carried on within.
Here you may see how men have multiplied toil for toil's sake, have
wrought to devise work superfluous, have worn their lives away in
imagining new forms of weariness. The energy, the ingenuity daily put
forth in these grimy burrows task the brain's power of wondering. But
that those who sit here through the live-long day, through every season,
through all the years of the life that is granted them, who strain their
eyesight, who overtax their muscles, who nurse disease in their frames,
who put resolutely from them the thought of what existence might
be---that these do it all without prospect or hope of reward save the
permission to eat and sleep and bring into the world other creatures to
strive with them for bread, surely that thought is yet more marvellous.

{\protect\hypertarget{26}{}{}}Workers in metal, workers in glass and in
enamel, workers in wood, workers in every substance on earth, or from
the waters under the earth, that can be made commercially valuable. In
Clerkenwell the demand is not so much for rude strength as for the
cunning fingers and the contriving brain. The inscriptions on the
house-fronts would make you believe that you were in a region of gold
and silver and precious stones. In the recesses of dim byways, where
sunshine and free air are forgotten things, where families herd together
in dear-rented garrets and cellars, craftsmen are for ever handling
jewellery, shaping bright ornaments for the necks and arms of such as
are born to the joy of life. Wealth inestimable is ever flowing through
these workshops, and the hands that have been stained with gold-dust
may, as likely as not, some day extend themselves in petition for a
crust. In this house, as the announcement tells you, business is carried
on by a trader in diamonds, and next door is a den full of children who
wait for their day's one meal {\protect\hypertarget{27}{}{}}until their
mother has come home with her chance earnings. A strange enough region
wherein to wander and muse. Inextinguishable laughter were perchance the
fittest result of such musing; yet somehow the heart grows heavy,
somehow the blood is troubled in its course, and the pulses begin to
throb hotly.

Amid the crowds of workpeople, Jane Snowdon made what speed she might.
It was her custom, whenever dispatched on an errand, to run till she
could run no longer, then to hasten along panting until breath and
strength were recovered. When it was either of the Peckovers who sent
her, she knew that reprimand was inevitable on her return, be she ever
so speedy; but her nature was incapable alike of rebellion and of that
sullen callousness which would have come to the aid of most girls in her
position. She did not serve her tyrants with willingness, for their
brutality filled her with a sense of injustice; yet the fact that she
was utterly dependent upon them for her livelihood,
{\protect\hypertarget{28}{}{}}that but for their grace---as they were
perpetually reminding her---she would have been a workhouse child, had a
mitigating effect upon the bitterness she could not wholly subdue.

There was, however, another reason why she sped eagerly on her present
mission. The man to whom she was conveying Mrs. Hewett's message was one
of the very few persons who had ever treated her with human kindness.
She had known him by name and by sight for some years, and since her
mother's death (she was eleven when that happened) he had by degrees
grown to represent all that she understood by the word ``friend.'' It
was seldom that words were exchanged between them; the opportunity came
scarcely oftener than once a month; but whenever it did come, it made a
bright moment in her existence. Once before she had fetched him of an
evening to see Mrs. Hewett, and as they walked together he had spoken
with what seemed to her wonderful gentleness, with consideration
{\protect\hypertarget{29}{}{}}inconceivable from a tall, bearded man,
well-dressed and well to do in the world. Perhaps he would speak in the
same way to-night; the thought of it made her regardless of the cold
rain that was drenching her miserable garment, of the wind that now and
then, as she turned a corner, took away her breath and made her cease
from running.

She reached St. John's Square, and paused at length by a door on which
was the inscription: ``H. Lewis, Working Jeweller.'' It was just
possible that the men had already left; she waited for several minutes
with anxious mind. No; the door opened and two workmen came forth.
Jane's eagerness impelled her to address one of them.

``Please, sir, Mr. Kirkwood hasn't gone yet, has he?''

``No, he ain't,'' the man answered pleasantly; and turning back, he
called to some one within the doorway: ``Hollo, Sidney! here's your
sweetheart waiting for you.''

Jane shrank aside; but in a moment she {\protect\hypertarget{30}{}{}}saw
a familiar figure; she advanced again and eagerly delivered her message.

``All right, Jane! I'll walk on with you,'' was the reply. And whilst
the other two men were laughing good-naturedly, Kirkwood strode away by
the girl's side. He seemed to be absent-minded, and for some hundred
yards' distance was silent; then he stopped of a sudden and looked down
at his companion.

``Why, Jane,'' he said, ``you'll get your death, running about in
weather like this.'' He touched her dress. ``I thought so; you're wet
through.''

There followed an inarticulate growl, and immediately he stripped off
his short overcoat.

``Here, put this on, right over your head. Do as I tell you, child!''

He seemed impatient to-night. Wasn't he going to talk with her as
before? Jane felt her heart sinking. With her hunger for kind and gentle
words, she thought nothing of the character of the night, and that
Sidney Kirkwood might reasonably be anxious to get over the ground as
quickly as possible.

{\protect\hypertarget{31}{}{}}``How is Mrs. Hewett?'' Sidney asked when
they were walking on again. ``Still poorly, eh? And the baby?''

Then he was again mute. Jane had something she wished to say to
him---wished very much indeed, yet she felt it would have been difficult
even if he had encouraged her. As he kept silence and walked so quickly,
speech on her part was utterly forbidden. Kirkwood, however, suddenly
remembered that his strides were disproportionate to the child's steps.
She was an odd figure thus disguised in his over-jacket; he caught a
glimpse of her face by a street-lamp, and smiled, but with a mixture of
pain.

``Feel a bit warmer so?'' he asked.

``Oh yes, sir.''

``Haven't you got a jacket, Jane?''

``It's all to pieces, sir. They're goin' to have it mended, I think.''

``They'' was the word by which alone Jane ventured to indicate her aunt.

``Going to, eh? I think they'd better be quick about it.''

{\protect\hypertarget{32}{}{}}Ha! that was the old tone of kindness! How
it entered into her blood and warmed it! She allowed herself one quick
glance at him.

``Do I walk too quick for you?''

"Oh no, sir. Mr. Kirkwood, please, there's something I{{------}}"

The sentence had, as it were, begun itself, but timidity cut it short.
Sidney stopped and looked at her.

``What? Something you wanted to tell me, Jane?''

He encouraged her, and at length she made her disclosure. It was of what
had happened in the public-house. The young man listened with much
attention, walking very slowly. He got her to repeat her second-hand
description of the old man who had been inquiring for people named
Snowdon.

``To think that you should have been just too late!'' he exclaimed with
annoyance.

``Have you any idea who he was?''

``I can't think, sir,'' Jane replied sadly.

{\protect\hypertarget{33}{}{}}Sidney took a hopeful tone---thought it
very likely that the inquirer would pursue his search with success,
being so near the house where Jane's parents had lived.

``I'll keep my eyes open,'' he said. ``Perhaps I might see him. He'd be
easy to recognise, I should think.''

``And would you tell him, sir?'' Jane asked eagerly.

``Why, of course I would. You'd like me to, wouldn't you?''

Jane's reply left small doubt on that score. Her companion looked down
at her again, and said with compassionate gentleness:

``Keep a good heart, Jane. Things'll be better some day, no doubt.''

``Do you think so, sir?''

The significance of the simple words was beyond all that eloquence could
have conveyed. Sidney muttered to himself, as he had done before, like
one who is angry. He laid his hand on the child's shoulder for a moment.

A few minutes more and they were passing
{\protect\hypertarget{34}{}{}}along by the prison wall, under the
ghastly head, now happily concealed by darkness. Jane stopped a little
short of the house and removed the coat that had so effectually
sheltered her.

``Thank you, sir,'' she said, returning it to Sidney.

He took it without speaking, and threw it over his arm. At the door, now
closed, Jane gave a single knock; they were admitted by Clem, who, in
regarding Kirkwood, wore her haughtiest demeanour. This young man had
never paid homage of any kind to Miss Peckover, and such neglect was by
no means what she was used to. Other men who came to the house took
every opportunity of paying her broad compliments, and some went so far
as to offer practical testimony of their admiration. Sidney merely had a
``How do you do, miss?'' at her service. Coquetry had failed to soften
him; Clem accordingly behaved as if he had given her mortal offence on
some recent occasion. She took care, moreover, to fling a few fierce
words at Jane {\protect\hypertarget{35}{}{}}before the latter
disappeared into the house. Thereupon Sidney looked at her sternly; he
said nothing, knowing that interference would only result in harsher
treatment for the poor little slave.

``You know your way upstairs, I b'lieve,'' said Clem, as if he were all
but a stranger.

``Thank you, I do,'' was Sidney's reply.

Indeed he had climbed these stairs innumerable times during the last
three years; the musty smells were associated with ever so many bygone
thoughts and states of feeling; the stains on the wall (had it been
daylight), the irregularities of the bare wooden steps, were
remembrancers of projects and hopes and disappointments. For many months
now every visit had been with heavier heart; his tap at the Hewetts'
door had a melancholy sound to him.

A woman's voice bade him enter. He stepped into a room which was not
disorderly or unclean, but presented the chill discomfort of poverty.
The principal, almost the only, articles of furniture were a large bed,
a {\protect\hypertarget{36}{}{}}wash-hand stand, a kitchen table, and
two or three chairs, of which the cane seats were bulged and torn. A few
meaningless pictures hung here and there, and on the mantelpiece, which
sloped forward somewhat, stood some paltry-ornaments, secured in their
places by a piece of string stretched in front of them. The living
occupants were four children and their mother. Two little girls, six and
seven years old respectively, were on the floor near the fire; a boy of
four was playing with pieces of firewood at the table. The remaining
child was an infant, born but a fortnight ago, lying at its mother's
breast. Mrs. Hewett sat on the bed and bent forward in an attitude of
physical weakness. Her age was twenty-seven, but she looked several
years older. At nineteen she had married; her husband, John Hewett,
having two children by a previous union. Her face could never have been
very attractive, but it was good-natured, and wore its pleasantest
aspect as she smiled on Sidney's entrance. You would have classed her at
once with those {\protect\hypertarget{37}{}{}}feeble-willed,
weak-minded, yet kindly-disposed women, who are only too ready to meet
affliction half-way, and who, if circumstances be calamitous, are more
harmful than an enemy to those they hold dear. She was rather wrapped up
than dressed, and her hair, thin and pale-coloured, was tied in a ragged
knot. She wore slippers, the upper parts of which still adhered to the
soles only by miracle. It looked very much as if the same relation
subsisted between her frame and the life that informed it, for there was
no blood in her cheeks, no lustre in her eye. The baby at her bosom
moaned in the act of sucking; one knew not how the poor woman could
supply sustenance to another being.

The children were not dirty nor uncared for, but their clothing hung
very loosely upon them; their flesh was unhealthy, their voices had an
unnatural sound.

Sidney stepped up to the bed and gave his hand.

``I'm so glad you've come before Clara,'' said Mrs. Hewett. "I hoped you
would. {\protect\hypertarget{38}{}{}}But she can't be long, an I want to
speak to you first. It's a bad night, isn't it? Yes, I feel it in my
throat, and it goes right through my chest---'ust `ere, look! And I
haven't slep' not a hour a night this last week; it makes me feel that
low. I want to get to the Orspital, if I can, in a day or two."

``But doesn't the doctor come still?'' asked Sidney, drawing a chair
near to her.

``Well, I didn't think it was right to go on payin' him, an' that's the
truth. I'll go to the Orspital, an' they'll give me somethin'. I look
bad, don't I, Sidney?''

``You look as if you'd no business to be out of bed,'' returned the
young man in a grumbling voice.

"Oh, I \emph{can't} lie still, so it's no use talkin'! But see, I want
to speak about Clara. That woman Mrs. Tubbs has been here to see me,
talkin' an' talkin'. She says she'll give Clara five shillin' a week, as
well as board an lodge her. I don't know what to do about it, that I
don't. Clara, she's that set on goin', an' her father's that set against
it. It seems as {\protect\hypertarget{39}{}{}}if it 'ud be a good thing,
don't it, Sidney? I know \emph{you} don't want her to go, but what's to
be done? What \emph{is} to be done?"

Her wailing voice caused the baby to wail likewise. Kirkwood looked
about the room with face set in anxious discontent.

``Is it no use, Mrs. Hewett?'' he exclaimed suddenly, turning to her.
``Does she mean it? Won't she ever listen to me?''

The woman shook her head miserably; her eyes filled with tears.

``I've done all I could,'' she replied, half sobbing. ``I have; you know
I have, Sidney! She's that 'eadstrong, it seems as if she wouldn't
listen to nobody---at least nobody as we knows anything about.''

``What do you mean by that?'' he inquired abruptly. ``Do you think
there's any one else?''

"How can I tell? I've got no reason for thinkin' it, but how can I tell?
No, I believe it's nothin' but her self-will an' the fancies she's got
into her `ead. Both her an' Bob, there's no doin' nothin' with them.
{\protect\hypertarget{40}{}{}}Bob, he's that wasteful with his money;
an' now he talks about goin' an' gettin' a room in another 'ouse, when
he might just as well make all the savin' he can. But no, that ain't his
idea, nor yet his sister's. I suppose it's their mother as they take
after, though their father he won't own to it, an' I don't blame him for
not speakin' ill of her as is gone. I should be that wretched if I
thought my own was goin' to turn out the same. But there's John, he
ain't a wasteful man; no one can't say it of him. He's got his fancies,
but they've never made him selfish to others, as well you know, Sidney.
He's been the best `usband to me as ever a poor woman had, an' I'll say
it with my last breath."

She cried pitifully for a few moments. Sidney, mastering his own
wretchedness, which he could not altogether conceal, made attempts to
strengthen her.

``When things are at the worst they begin to mend,'' he said. "It can't
be much longer before he gets work. And look here, Mrs. Hewett, I won't
hear a word against it; you {\protect\hypertarget{41}{}{}}must and shall
let me lend you something to go on with!"

``I dursn't, I dursn't, Sidney! John won't have it. He's always
a-saying: `Once begin that, an' it's all up; you never earn no more of
your own.` It's one of his fancies, an' you know it is. You'll only make
trouble, Sidney.''

``Well, all I can say is, he's an unreasonable and selfish man!''

``No, no; John ain't selfish! Never say that! It's only his fancies,
Sidney.''

``Well, there's one trouble you'd better get rid of, at all events. Let
Clara go to Mrs. Tubbs. You'll never have any peace till she does, I can
see that. Why shouldn't she go, after all? She's seventeen; if she can't
respect herself now, she never will, and there's no help for it. Tell
John to let her go.''

There was bitterness in the tone with which he gave this advice; he
threw out his hands impatiently, and then flung himself back, so that
the cranky chair creaked and tottered.

{\protect\hypertarget{42}{}{}}"An' if `arm comes to her, what then?"
returned Mrs. Hewett plaintively. "We know well enough why Mrs. Tubbs
wants her; it's only because she's good-lookin', an' she'll bring more
people to the bar. John knows that, an' it makes him wild. Mind what I'm
tellin' you, Sidney; if any `arm comes to that girl, her father'll go
out of his 'ead. I know he will! I know he will! He worships the ground
as she walks on, an' if it hadn't been for that, she'd never have given
him the trouble as she is doin'. It `ud a been better for her if she'd
had a father like mine, as was a hard, careless man. I don't wish to say
no 'arm of him as is dead an' buried, an my own father too, but he was a
hard father to us, an' as long as he lived we dursn't say not a word as
he didn't like. He'd a killed me if I'd gone on like Clara. It was a
good thing as he was gone, before{{------}}"

``Don't, don't speak of that,'' interposed Kirkwood, with kindly
firmness. ``That's long since over and done with and forgotten.''

{\protect\hypertarget{43}{}{}}``No, no; not forgotten. Clara knows, an
that's partly why she makes so little of me; I know it is.''

"I don't believe it! She's a good-hearted girl{{------}}"

A heavy footstep on the stairs checked him. The door was thrown open,
and there entered a youth of nineteen, clad as an artisan. He was a
shapely fellow, though not quite so stout as perfect health would have
made him, and had a face of singular attractiveness, clear-complexioned,
delicate featured, a-gleam with intelligence. The intelligence was
perhaps even too pronounced; seen in profile, the countenance had an
excessive eagerness; there was selfish force about the lips, moreover,
which would have been better away. His noisy entrance indicated an
impulsive character, and the nod with which he greeted Kirkwood was
self-sufficient.

``Where's that medal I cast last night, mother?'' he asked, searching in
various corners of the room and throwing things about.

{\protect\hypertarget{44}{}{}}``Now, do mind what you're up to, Bob!''
remonstrated Mrs. Hewett. ``You'll find it on the mantel in the other
room. Don't make such a noise.''

The young man rushed forth, and in a moment returned. In his hand, which
was very black, and shone as if from the manipulation of metals, he held
a small bright medal. He showed it to Sidney, saying, ``What d'you think
o' that?''

The work was delicate and of clever design; it represented a racehorse
at fullspeed, a jockey rising in the stirrups and beating it with
orthodox brutality.

``That's 'Tally-ho' at the Epsom Spring Meetin','' he said. ``I've got
money on him!''

And, with another indifferent nod, he flung out of the room.

Before Mrs. Hewett and Kirkwood could renew their conversation, there
was another step at the door, and the father of the family presented
himself.
