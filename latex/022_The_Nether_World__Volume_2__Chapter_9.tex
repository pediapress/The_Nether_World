{}

{CHAPTER IX.}

WATCHING FROM AMBUSH.

\textsc{Mr. Joseph Snowdon}, though presenting a calm countenance to the
world and seeming to enjoy comparative prosperity, was in truth much
harassed by the difficulties of his position. Domestic troubles he had
anticipated, but the unforeseen sequel of his marriage resulted in a
martyrdom at the hands of Clem and her mother such as he had never
dreamed of. His faults and weaknesses distinctly those of the civilised
man, he found himself in disastrous alliance with two savages, whose
characters so supplemented each other as to constitute in unison a
formidable engine of tyranny. Clem,---suspicious, revengeful, fierce,
watching with cruel eyes every opportunity of taking payment on account
for the ridicule to which she had exposed herself; {}Mrs.
Peckover,---ceaselessly occupied with the basest scheming, keen as an
Indian on any trail she happened to strike, excited by the scent of
money as a jackal by that of caiTion; for this pair Joseph was no match.
Not only did they compel him to earn his daily bread by dint of
methodical effort such as was torture to his indolent disposition, but,
moreover, in pursuance of Mrs. Peckover's crafty projects, he was
constrained to an assiduous hypocrisy in his relations with Michael and
Jane which wearied him beyond measure. Joseph did not belong to the most
desperate class of hungry mortals; he had neither the large ambitions
and the passionate sensual desires which make life an unending fever,
nor was he possessed with that foul itch of covetousness which is the
explanation of the greater part of the world's activity. He understood
quite sufficiently the advantages of wealth, and was prepared to go
considerable lengths for the sake of enjoying them, but his character
lacked persistence. This defect explained the rogueries and calamities
of his {}life. He had brains in abundance, and a somewhat better
education would have made of him either a successful honest man or a
rascal of superior scope,---it is always a toss-up between these two
results where a character such as his is in question. Ever since he
abandoned the craft to which his father had had him trained, he had
lived on his wits; there would be matter for a volume in the history of
his experiences at home and abroad, a volume infinitely more valuable
considered as a treatise on modern civilisation than any professed work
on that subject in existence. With one episode only in his past can we
here concern ourselves; the retrospect is needful to make clear his
relations with Mr. Scawthorne.

On his return from America, Joseph possessed a matter of a hundred
pounds; the money was not quite legally earned (pray let us reserve the
word honesty for a truer use than the common one), and on the whole he
preferred to recommence life in the old country under a
pseudonym,---that little affair {}of the desertion of his child would
perhaps, in any case, have made this advisable. A hundred pounds will
not go very far, but Joseph took care to be well dressed, and allowed it
to be surmised by those with whom he came in contact that the resources
at his command were considerable. In early days, as we know, he had
worked at electroplating, and the natural bent of his intellect was
towards mechanical and physical science; by dint of experimenting at his
old pursuit, he persuaded himself, or at all events attained
plausibility for the persuading of others, that he had discovered a new
and valuable method of plating with nickel. He gave it out that he was
in search of a partner to join him in putting this method into practice.
Gentlemen thus situated naturally avail themselves of the advertisement
columns of the newspaper, and Joseph by this means had the happiness to
form an acquaintance with one Mr. Polkenhorne, who, like himself, had
sundry schemes for obtaining money without toiling for it in the usual
vulgar way. {}Polkenhorne was a man of thirty-five, much of a
blackguard, but keen-witted, handsome, and tolerably educated; the son
of a Clerkenwell clockmaker, he had run through an inheritance of a few
thousand pounds, and made no secret of his history,---spoke of his
experiences, indeed, with a certain pride. Between these two a close
intimacy sprang up, one of those partnerships, beginning with mutual
deception, which are so common in the border-land of enterprise just
skirting the criminal courts. Polkenhorne resided at this time in
Kennington; he was married---or said that he was---to a young lady in
the theatrical profession, known to the public as Miss Grace Danver. To
Mrs. Polkenhorne, or Miss Danver, Joseph soon had the honour of being
presented, for she was just then playing at a London theatre; he found
her a pretty but consumptive-looking girl, not at all likely to achieve
great successes, earning enough, however, to support Mr. Polkenhorne
during this time of his misfortunes,---a most pleasant and natural
arrangement.

{}Polkenhorne's acquaintances were numerous, but, as he informed Joseph,
most of them were ``played out,'' that is to say, no further use could
be made of them from Polkenhorne's point of view. One, however, as yet
imperfectly known, promised to be useful, perchance as a victim, more
probably as an ally; his name was Scawthorne, and Polkenhorne had come
across him in consequence of a friendship existing between Grace Danver
and Mrs. Scawthorne,---at all events, a young lady thus known,---who was
preparing herself for the stage. This gentleman was ``something in the
City;'' he had rather a close look, but proved genial enough, and was
very ready to discuss things in general with Mr. Polkenhorne and his
capitalist friend Mr. Camden, just from the United States.

A word or two about Charles Henry Scawthorne, of the circumstances which
made him what you know, or what you conjecture. His father had a small
business as a dyer in Islington, and the boy, leaving school at
fourteen, was sent to become a copying-clerk in a {}solicitor's office;
his tastes were so stronglyintellectual that it seemed a pity to put him
to work he hated, and the clerkship was the best opening that could be
procured for him. Two years after, Mr. Scawthorne died; his wife tried
to keep on the business, but soon failed, and thenceforth her son had to
support her as well as himself. From sixteen to three-and-twenty was the
period of young Scawthorne's life which assured his future
advancement---and his moral ruin. A grave, gentle, somewhat effeminate
boy, with a great love of books and a wonderful power of application to
study, he suffered so much during those years of early maturity, that,
as in almost all such cases, his nature was corrupted. Pity that some
self-made intellectual man of our time has not flung in the world's
teeth a truthful autobiography. Scawthorne worked himself up to a
position which had at first seemed unattainable; what he paid for the
success was loss of all his pure ideals, of his sincerity, of his
disinterestedness, of the fine perceptions to which he was born.
{}Probably no one who is half-starved and overworked during those
critical years comes out of the trial with his moral nature uninjured;
to certain characters it is a wrong irreparable. To stab the root of a
young tree, to hang crushing burdens upon it, to rend off its early
branches,---that is not the treatment likely to result in growth such as
nature purposed. There will come of it a vicious formation, and the
principle applies also to the youth of men.

Scawthorne was fond of the theatre; as soon as his time of incessant
toil was over, he not only attended performances frequently, but managed
to make personal acquaintance with sundry theatrical people. Opportunity
for this was afforded by his becoming member of a club, consisting
chiefly of solicitors' clerks, which was frequently honoured by visits
from former associates who had taken to the stage; these happy beings
would condescend to recite at times, to give help in getting up a
dramatic entertainment, and soon, in this way, Scawthorne came to know
an old actor named {}Drake, who supported himself by instructing
novices, male and female, in his own profession; one of Mr. Drake's old
pupils was Miss Grace Danver, in whom, as soon as he met her, Scawthorne
recognised the Grace Rudd of earlier days. And it was not long after
this that he brought to Mr. Drake a young girl of interesting
appearance, but very imperfect education, who fancied she had a turn for
acting; he succeeded in arranging for her instruction, and a year and a
half later she obtained her first engagement at a theatre in Scotland.
The name she adopted was Clara Vale. Joseph Snowdon saw her once or
twice before she left London, and from Grace Danver he heard that Grace
and she had been schoolfellows in Clerkenwell. These facts revived in
his memory when he afterwards heard Clem speak of Clara Hewett.

Nothing came of the alliance between Polkenhorne and Joseph; when the
latter's money was exhausted, they naturally fell apart. Joseph made a
living in sundiy precarious ways, but at length sank into such {}straits
that he risked the step of going to Clerkenwell Close. Personal interest
in his child he had then none whatever; his short married life seemed an
episode in the remote past, recalled with indifference. But in spite of
his profound selfishness, it was not solely from the speculative point
of view that he regarded Jane, when he had had time to realise that she
was his daughter, and in a measure to appreciate her character. With the
merely base motives which led him to seek her affection and put him at
secret hostility with Sidney Kirkwood, there mingled before long a
strain of feeling which was natural and pure; he became a little jealous
of his father and of Sidney on other grounds than those of
self-interest. Intolerable as his home was, no wonder that he found it a
pleasant relief to spend an evening in Hanover Street; he never came
away without railing at himself for his imbecility in having married
Clem. For the present he had to plot with his wife and Mrs. Peckover,
but only let the chance for plotting \emph{against} them offer itself!
{}The opportunity might come. In the meantime, the great thing was to
postpone the marriage---he had no doubt it was contemplated---between
Jane and Sidney. That would be little less than a fatality.

The week that Jane spent in Essex was of course a time of desperate
anxiety with Joseph; immediately on her return he hastened to assure
himself that things remained as before. It seemed to him that Jane's
greeting had more warmth than she was wont to display when they met;
sundry other little changes in her demeanour struck him at the same
interview, and he was rather surprised that she had not so much
blitheness as before she went away. But his speculation on minutiæ such
as these was suddenly interrupted a day or two later by news which threw
him into a state of excitement; Jane sent word that her grandfather was
very unwell, that he appeared to have caught a chill in the journey
home, and could not at present leave his bed. For a week the old man
suffered from feverish symptoms, and, though he threw off the {}ailment,
it was in a state of much feebleness that he at length resumed the
ordinary tenor of his way. Jane had of course stayed at home to nurse
him; a fortnight, a month passed, and Michael still kept her from work.
Then it happened that, on Joseph's looking in one evening, the old man
said quietly, ``I think I'd rather Jane stayed at home in future. We've
had a long talk about it this afternoon.''

Joseph glanced at his daughter, who met the look very gravely. He had a
feeling that the girl was of a sudden grown older; when she spoke it was
in brief phrases, and with but little of her natural spontaneity;
noiseless as always in her movements, she walked with a staider gait,
held herself less girlishly, and on saying goodnight she let her cheek
rest for a moment against her father's, a thing she had never yet done.

The explanation of it all came a few minutes after Jane's retirement.
Michael, warned by his illness how unstable was the tenure on which he
henceforth held his life, had resolved to have an end of mystery and
explain to his son all that he had already {}made known to Sidney
Kirkwood. With Jane he had spoken a few hours ago, revealing to her the
power that was in his hands, the solemn significance he attached to it,
the responsibility with which her future was to be invested. To make the
same things known to Joseph was a task of more difficulty. He could not
here count on sympathetic intelligence; it was but too certain that his
son would listen with disappointment, if not with bitterness. In order
to mitigate the worst results, he began by making known the fact of his
wealth and asking if Joseph had any practical views which could be
furthered by a moderate sum put at his disposal.

``At my death,'' he added, ``you'll find that I haven't dealt unkindly
by you. But you're a man of middle age, and I should like to see you in
some fixed way of life before I go.''

Having heard all, Joseph promised to think over the proposal which
concerned himself It was in a strange state of mind that he returned to
the Close; one thing only he was clear upon, that to Clem and her mother
he {}would breathe no word of what had been told him. After a night
passed without a wink of sleep, struggling with the amazement, the
incredulity, the confusion of understanding caused by his father's
words, he betook himself to a familiar public-house, and there penned a
note to Scawthorne, requesting an interview as soon as possible. The
meeting took place that evening at the retreat behind Lincoln's Inn
Fields where the two had held colloquies on several occasions during the
last half-year. Scawthorne received with gravity what his acquaintance
had to communicate. Then he observed:

``The will was executed ten days ago.''

``It was? And what's he left me?''

``Seven thousand pounds---less legacy duty.''

``And thirty thousand to Jane?''

``Just so.''

Joseph drew in his breath; his teeth ground together for a moment; his
eyes grew very wide. With a smile Scawthorne proceeded to explain that
Jane's trustees were Mr. Percival, senior, and his son. Should she {}die
unmarried before attaining her twentyfirst birthday, the money
bequeathed to her was to be distributed among certain charities.

``It's my belief there's a crank in the old fellow,'' exclaimed Joseph.
``Is he really such a fool as to think Jane won't use the money for
herself? And what about Kirkwood? I tell you what it is; he's a deep
fellow, is Kirkwood. I wish you knew him.''

Scawthorne confessed that he had the same wish, but added that there was
no chance of its being realised; prudence forbade any move in that
direction.

``If he marries her,'' questioned Joseph, ``will the money be his?''

``No; it will be settled on her. But it comes to very much the same
thing; there's to be no restraint on her discretion in using it.''

``She might give her affectionate parent a hundred or so now and then,
if she chose?''

``If she chose.''

Scawthorne began a detailed inquiry into the humanitarian projects of
which Joseph had given but a rude and contemptuous {}explanation. The
finer qualities of his mind enabled him to see the matter in quite a
different light from that in which it presented itself to Jane's father;
he had once or twice had an opportunity of observing Michael Snowdon at
the office, and could realise in a measure the character which directed
its energies to such an ideal aim. Concerning Jane he asked many
questions; then the conversation turned once more to Sidney Kirkwood.

``I wish he'd married his old sweetheart,'' observed Joseph, watching
the other's face.

``Who was that?''

``A girl called Clara Hewett.''

Their looks met. Scawthorne, in spite of habitual self-command, betrayed
an extreme surprise.

``I wonder what's become of her?'' continued Joseph, still observing his
companion, and speaking with unmistakable significance.

``Just tell me something about this,'' said Scawthorne peremptorily.
Joseph complied, and ended his story with a few more hints.

{}``I never saw her myself,---at least I can't be sure that I did. There
was somebody of the same name---Clara---a friend of Polkenhorne's
wife.''

Scawthorne appeared to pay no attention; he mused with a wrinkled brow.

``If only I could put something between Kirkwood and the girl,''
remarked Joseph, as if absently. ``I shouldn't wonder if it could be
made worth some one's while to give a bit of help that way. Don't you
think so?''

In the tone of one turning to a different subject, Scawthorne asked
suddenly:

``What use are you going to make of your father's offer?''

``Well, I'm not quite sure. Shouldn't wonder if I go in for filters.''

``Filters?''

Joseph explained. In the capacity of ``commission agent''---denomination
which includes and apologises for such a vast variety of casual
pursuits---he had of late been helping to make known to the public a new
filter, which promised to be a commercial success. {}The owner of the
patent lacked capital, and a judicious investment might secure a share
in the business; Joseph thought of broaching the subject with him next
day.

``You won't make a fool of yourself?'' remarked Scawthorne.

``Trust me; I think I know my way about.''

For the present these gentlemen had nothing more to say to each other;
they emptied their glasses with deliberation, exchanged a look which
might mean either much or nothing, and so went their several ways.

The filter project was put into execution. When Joseph had communicated
it in detail to his father, the latter took the professional advice of
his friend Mr. Percival, and in the course of a few weeks Joseph found
himself regularly established in a business which had the---for
him---novel characteristic of serving the purposes of purity. The
manufactory was situated in a by-street on the north of Euston Road: a
small concern, but at all events a genuine one. On the window of the
office you read, ``Lake, Snowdon, \& Co.'' As it was necessary to
account for this {}achievement to Clem and Mrs. Peckover, Joseph made
known to them a part of the truth; of the will he said nothing, and, for
reasons of his own, he allowed these tender relatives to helieve that he
was in a fair way to inherit the greater part of Michaers possessions.
There was jubilation in Clerkenwell Close, but mother and daughter kept
stern watch upon Joseph's proceedings.

Another acquaintance of ours benefited by this event. Michael made it a
stipulation that some kind of work should be found at the factoiy for
John Hewett, who, since his wife's death, had been making a wretched
struggle to establish a more decent home for the children. The firm of
Lake, Snowdon, and Co. took Hewett into their employment as a porter,
and paid him twenty-five shillings a week,---of which sum, however, the
odd five shillings were privately made up by Michael. On receiving this
appointment, John drew the sigh of a man who finds himself in haven
after perilous beating about a lee shore. The kitchen in King's Cross
Eoad was abandoned, and with Sidney Kirkwood's aid the family {}found
much more satisfactory quarters. Friends of Sidney's, a man and wife of
middle age without children, happened to be looking for lodgings; it was
decided that they and John Hewett should join in the tenancy of a flat,
up on the fifth storey of the huge block of tenements called Farringdon
Road Buildings. By this arrangement the children would be looked after,
and the weekly twenty-five shillings could be made to go much further
than on the ordinary system. As soon as everything had been settled, and
when Mr. and Mrs. Eagles had already housed themselves in the one room
which was all they needed for their private accommodation, Hewett and
the children began to pack together their miserable sticks and rags for
removal. Just then Sidney Kirkwood looked in.

``Eagles wants to see you for a minute about something,'' he said.
``Just walk round with me, will you?''

John obeyed, in the silent, spiritless way now usual with him. It was
but a short distance to the buildings; they went up the {}winding stone
staircase, and Sidney gave a hollow-sounding knock at one of the two
doors that faced each other on the fifth storey. Mrs. Eagles opened, a
decent motherly woman, with a pleasant and rather curious smile on her
face. She led the way into one of the rooms which John had seen empty
only a few hours ago. How was this? Oil-cloth on the floor, a blind at
the window, a bedstead, a table, a chest of {drawers{{------}}.}

Mrs. Eagles withdrew, discreetly. Hewett stood with a look of uneasy
wonderment, and at length turned to his companion.

``Now, look here,'' he growled, in an unsteady voice, ``what's all this
about?''

``Somebody seems to have got here before you,'' replied Sidney, smiling.

``How the devil am I to keep any self-respect if you go on treatin' me
in this fashion?'' blustered John, hanging his head.

``It isn't my doing, Mr. Hewett.''

``Whose, then?''

``A friend's. Don't make a fuss. You shall know the person some day.''
