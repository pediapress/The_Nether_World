{}

{CHAPTER XI.}

THE FAMILY HISTORY PROGRESSES.

\textsc{What} could possess John Hewett that, after resting from the
day's work, he often left his comfortable room late in the evening and
rambled about the streets of that part of London which had surely least
interest for him, the streets which are thronged with idlers, with
carriages going homeward from the theatres, with those who can only come
forth to ply their business when darkness has fallen? Did he seek food
for his antagonism in observing the characteristics of the world in
which he was a stranger, the world which has its garners full and takes
its ease amid superfluity? It could scarcely be that, for since his
wife's death an indifference seemed to be settling upon him; he no
longer cared to visit the Green or his club on Sunday, and seldom spoke
on the subjects which formerly goaded {}him to madness. He appeared to
be drawn forth against his will, in spite of weariness, and his look as
he walked on was that of a man who is in search of some one. Yet whom
could he expect to meet in these highways of the West End?

Oxford Street, Regent Street, Piccadilly, the Strand, the ways about St.
James's Park; John Hewett was not the only father who has come forth
after nightfall from an obscure home to look darkly at the faces passing
on these broad pavements. At times he would shrink into a shadowed
corner, and peer thence at those who went by under the gaslight. When he
moved forward, it was with the uneasy gait of one who shuns observation;
you would have thought, perchance, that he watched an opportunity of
begging and was shamefaced; it happened now and then that he was
regarded suspiciously. A rough-looking man, with grizzled beard, with
eyes generally bloodshot, his shoulders stooping,---naturally the
miserable are always suspected where law is conscious of its injustice.

{}Two years ago he was beset for a time with the same restlessness, and
took night-walks in the same directions; the habit wore away, however.
Now it possessed him even more strongly. Between ten and eleven o'clock,
when the children were in bed, he fell into abstraction, and presently,
with an unexpected movement, looked up as if some one had spoken to
him,---just the look of one who hears a familiar voice; then he sighed,
and took his hat and went forth. It happened sometimes when he was
sitting with his friends Mr. and Mrs. Eagles; in that case he would make
some kind of excuse. The couple suspected that his business would take
him to the public-house, but John never came back with a sign about him
of having drunk; of that failing he had broken himself. He went
cautiously down the stone stairs, averting his face if any one met him;
then by cross-ways he reached Gray's Inn Road, and so westwards.

He had a well-ordered home, and his children were about him, but these
things did not compensate him for the greatest loss his life {}had
suffered. The children, in truth, had no very strong hold upon his
affections. Sometimes, when Amy sat and talked to him, he showed a
growing nervousness, an impatience, and at length turned away from her
as if to occupy himself in some manner. The voice was not that which had
ever power to soothe him when it spoke playfully. Memory brought back
the tones which had been so dear to him, and at times something more
than memory; he seemed really to hear them, as if from a distance. And
then it was that he went out to wander in the streets.

Of Bob in the meantime he saw scarcely anything. That young man
presented himself one Sunday shortly after his father had become settled
in the new home, but practically he was a stranger. John and he had no
interests in common; there even existed a slight antipathy on the
father's part of late years. Strangely enough this feeling expressed
itself one day in the form of a rebuke to Bob for neglecting
Pennyloaf,---Pennyloaf, whom John had always declined to recognise.

{}``I hear no good of your goin's on,'' remarked Hewett, on a casual
encounter in the street. ``A married man ought to give up the kind of
company as you keep.''

``I do no harm,'' replied Bob, bluntly.

``Has my wife been complaining to you?''

``I've nothing to do with her; it's what I'm told.''

``By Kirkwood, I suppose? You'd better not have made up with him again,
if he's only making mischief.''

``No, I didn't mean Kirkwood.''

And John went his way. Odd thing, was it not, that this embittered
leveller should himself practise the very intolerance which he reviled
in people of the upper world. For his refusal to recognise Pennyloaf he
had absolutely no grounds, save---I use the words advisedly---an
aristocratic prejudice. Bob had married deplorably beneath him; it was
unpardonable, let the character of the girl be what it might. Of course
you recognise the item in John Hewett's personality which serves to
explain this singular attitude. But, viewed {}generally, it was one of
those bits of human inconsistency over which the observer smiles, and
which should be recommended to good people in search of arguments for
the equality of men.

After that little dialogue. Bob went home in a disagreeable temper. To
begin with, his mood had been ruffled, for the landlady at his
lodgings---the fourth to which he had removed this year---was ``nasty''
about a week or two of unpaid rent, and a man on whom he had counted
this evening for the payment of a debt was keeping out of his way. He
found Pennyloaf sitting on the stairs with her two children, as usual;
poor Pennyloaf had not originality enough to discover new expressions of
misery, and that one bright idea of donning her best dress was a single
instance of ingenuity. In obedience to Jane Snowdon, she kept herself
and the babies and the room tolerably clean, but everything was done in
the most dispirited way.

``What are you kicking about here for?'' asked Bob impatiently. ``That's
how that kid gets its cold---of course it is!---{Ger} out!''

{}The last remark was addressed to the elder child, who caught at his
legs as he strode past. Bob was not actively unkind to the little
wretches for whose being he was responsible; he simply occupied the
natural position of unsophisticated man to children of that age, one of
indifference, or impatience. The infants were a nuisance; no one desired
their coming, and the older they grew the more expensive they were.

It was a cold evening of October; Pennyloaf had allowed the fire to get
very low (she knew not exactly where the next supply of coals was to
come from), and her husband growled as he made a vain endeavour to warm
his hands.

``Why haven't you got tea ready?'' he asked.

``I couldn't be sure as you was comin', Bob; how could I? But I'll soon
get the kettle boilin'.''

``Couldn't be sure as I was coming? Why, I've been back every night this
week---except two or three.''

{}It was Thursday, but Bob meant nothing jocose.

``Look here!'' he continued, fixing a surly eye upon her. ``What do you
mean by complaining about me to people? Just mind your own business.
When was that girl Jane Snowdon here last?''

``Yesterday, Bob.''

``I thought as much. Did she give you anything?'' He made this inquiry
in rather a shamefaced way.

``No, she didn't.''

``Well, I tell you what it is. I'm not going to have her coming about
the place, so understand that. When she comes next, you'll just tell her
she needn't come again.''

Pennyloaf looked at him with dismay. For the delivery of this command
Bob had seated himself on the corner of the table and crossed his arms.
But for the touch of blackguardism in his appearance, Bob would have
been a very good-looking fellow; his face was healthy, by no means
commonplace in its mould, and had the peculiar vividness which
{}indicates ability,---so impressive, because so rarely seen, in men of
his level. Unfortunately his hair was cropped all but to the scalp, in
the fashionable manner; it was greased, too, and curled up on one side
of his forehead with a peculiarly offensive perkishness. Poor Pennyloaf
was in a great degree responsible for the ills of her married life; not
only did she believe Bob to be the handsomest man who walked the earth,
but in her weakness she could not refrain from telling him as much. At
the present moment he was intensely self-conscious; with Pennyloafs eye
upon him, he posed for effect. The idea of forbidding future intercourse
with Jane had come to him quite suddenly; it was by no means his
intention to make his order permanent, for Jane had now and then brought
little presents which were useful, but just now he felt a satisfaction
in asserting authority. Jane should understand that he regarded her
censure of him with high displeasure.

``You don't mean that, Bob?'' murmured Pennyloaf.

{}``Of course I do. And let me catch you disobeying me! I should think
you might find better friends than a girl as used to be the Peckovers'
dirty little servant.''

Bob turned up his nose and sniffed the air. And Pennyloaf, in spite of
the keenest distress, actually felt that there was something in the
objection, thus framed! She herself had never been a servant,---never;
she had never sunk below working with the needle for sixteen hours a day
for a payment of ninepence. The work-girl regards a domestic slave as
very distinctly her inferior.

``But that's a long while ago,'' she ventured to urge, after reflection.

``That makes no difference. Do as I tell you, and don't argue.''

It was not often that visitors sought Bob at his home of an evening, but
whilst this dialogue was still going on an acquaintance made his arrival
known by a knock at the door. It was a lank and hungry individual, grimy
of face and hands, his clothing such as in the country would serve well
for a {}scarecrow. Who could have recognised in him the once spruce and
spirited Mr. Jack Bartley, distinguished by his chimney-pot hat at the
Crystal Palace on Bob's wedding-day? At the close of that same day, as
you remember, he and Bob engaged in terrific combat, the outcome of
earlier rivalry for the favour of Clem Peckover. Notwithstanding that
memory, the two were now on very friendly terms. You have heard from
Clem's lips that Jack Bartley, failing to win herself, ended by
espousing Miss Susan Jollop; also what was the result of that alliance.
Mr. Bartley was an unhappy man. His wife had a ferocious temper, was
reckless with money, and now drank steadily; the consequence was, that
Jack had lost all regular employment, and only earned occasional pence
in the most various ways. Broken in spirit, he himself first made
advances to his companion of former days, and Bob, flattered by the
other's humility, encouraged him as a hangeron.---Really, we shall soon
be coming to a conclusion that the differences between the {}nether and
the upper world are purely superficial.

Whenever Jack came to spend an hour with Mr. and Mrs. Hewett, he was
sure sooner or later to indulge the misery that preyed upon him and give
way to sheer weeping. He did so this evening, almost as soon as he
entered.

``I ain't had a mouthful past my lips since last night, I ain't!'' he
sobbed. ``It's 'ard on a feller as used to have his meals regular. I'll
murder Suke yet, see if I don't! I'll have her life! She met me last
night and gave me this black eye as you see,---she did! It's 'ard on a
feller.''

``You mean to say as she \emph{'it} you?'' cried Pennyloaf.

Bob chuckled, thrust his hands into his pockets, spread himself out. His
own superiority was so gloriously manifest.

``Suppose \emph{you} try it on with \emph{me}, Penny!'' he cried.

``You'd give me something as I should remember,'' she answered,
smirking, the good little slavey.

{}``Shouldn't wonder if I did,'' assented Bob.

Mr. Bartley's pressing hunger was satisfied with some bread and butter
and a cup of tea. Whilst taking a share of the meal, Bob brought a small
box on to the table; it had a sliding lid, and inside were certain
specimens of artistic work with which he was wont to amuse himself when
tired of roaming the streets in jovial company. Do you recollect that,
when we first made Bob's acquaintance, he showed Sidney Kirk wood a
medal of his own design and casting? His daily work at die-sinking had
of course supplied him with this suggestion, and he still found pleasure
in work of the same kind. In days before commercialism had divorced art
and the handicrafts, a man with Bob's distinct faculty would have found
encouragement to exercise it for serious ends; as it was, he remained at
the semi-conscious stage with regard to his own aptitudes, and cast
leaden medals just as a way of occupying his hands when a couple of
hours hung heavy on them. Partly with the thought of amusing the
dolorous Jack, yet more to win {}laudation, he brought forth now a
variety of casts and moulds and spread them on the table. His latest
piece of work was a medal in high relief bearing the heads of the Prince
and Princess of Wales surrounded with a wreath. Bob had no political
convictions; with complacency he drew these royal features, the sight of
which would have made his father foam at the mouth. True, he might have
found subjects artistically more satisfying, but he belonged to the
people, and the English people.

Jack Bartley, having dried his eyes and swallowed his bread and butter,
considered the medal with much attention. ``I say,'' he remarked at
length, ``will you give me this. Bob?''

``I don't mind. You can take it if you like.''

``Thanks!''

Jack wrapped it up and put it in his waistcoat pocket, and before long
rose to take leave of his friends.

``I only wish I'd got a wife like you,'' he observed at the door, as he
saw Pennyloaf bending over the two children, recently put to bed.
{}Pennyloaf's eyes gleamed at the compliment, and she turned them to her
husband.

``She's nothing to boast of,'' said Bob, judicially and masculinely.
``All women are pretty much alike.''

And Pennyloaf tried to smile at the snub.

Having devoted one evening to domestic quietude. Bob naturally felt
himself free to dispose of the next in a manner more to his taste. The
pleasures which sufficed to keep him from home had the same sordid
monotony which characterises life in general for the lower strata of
society. If he had money, there was the music-hall; if he had none,
there were the streets. Being in the latter condition to-night, he
joined a company of male and female intimates, and with them strolled
aimlessly from one familiar rendezvous to another. Would that it were
possible to set down a literal report of the conversation which passed
during hours thus spent. Much of it, of course, would be merely
revolting, but for the most part it would consist of such wearying, such
incredible imbecilities as no human {}patience could endure through five
minutes' perusal. Realise it, however, and you grasp the conditions of
what is called the social problem. As regards Robert Hewett in
particular, it would help you to understand the momentous change in his
life which was just coming to pass. On his reaching home at eleven
o'clock, Pennyloaf met him with the news that Jack Bartley had looked in
twice and seemed very anxious to see him. To-morrow being Saturday, Jack
would call again early in the afternoon. When the time came, he
presented himself, hungry and dirty as ever, but with an unwonted
liveliness in his eye.

``I've got something to say to you,'' he began, in a low voice, nodding
significantly towards Pennyloaf.

``Go and buy what you want for to-morrow,'' said Bob to his wife, giving
her some money out of his wages. ``Take the kids.''

Disappointed in being thus excluded from confidence, but obedient as
ever, Pennyloaf speedily prepared herself and the children, the
{}younger of whom she still had to carry. When she was gone Mr. Bartley
assumed a peculiar attitude and began to speak in an undertone.

``You know that medal as you gave me the other night?''

``What about it?''

``I sold it for fourpence to a chap I know. It got me a bed at the
lodgings in Pentonville Road.''

``Oh, you did! Well, what else?''

Jack was writhing in the most unaccountable way, peering hither and
thither out of the corners of his eyes, seeming to have an obstruction
in his throat.

``It was in a public-house as I sold it,---a chap I know. There was
another chap as I didn't know standing just by,---see? He kep' looking
at the medal, and he kep' looking at me. When I went out, the chap as I
didn't know followed behind me. I didn't see him at fu'st, but he come
up with me just at the top of Rosoman Street,---a red-haired chap,
looked like a corster. `Hollo!' says he. `Hollo!' says I. `Got any more
o' them {}medals?' he says, in a quiet way like. `What do you want to
know for?' I says,---'cos you see he was a bloke as I didn't know
nothing about, and there's no good being over-free with your talk. He
got me to walk on a bit with him, and kept talking. `You didn't buy that
nowhere,' he says, with a sort of wink. `What if I didn't?' I says.
`There's no harm, as I know.' Well, he kept on with his sort o' winks,
and then he says, `Got any \emph{queer} to put round?'''

At this point, Jack lowered his voice to a whisper and looked timorously
towards the door.

``You know what he meant. Bob?''

Bob nodded and became reflective.

``Well, I didn't say nothing,'' pursued Bartley, ``but the chap stuck to
me. `A fair price for a fair article,' he says. `You'll always find me
there of a Thursday night, if you've got any business going. Give me a
look round,' he says. `It ain't in my line,' I says. So he gave a grin
like, and kep' on talking. `If you want \emph{a four-half shiner},' he
says, `you {}know where to come. Reasonable with them as is reasonable.
Thursday night,' he says, and then he slung his hook round the corner.''

``What's a four-half shiner?'' inquired Bob, looking from under his
eyebrows.

``Well, I didn't know myself, just then; but I've found out. It's a
public-house pewter,---see?''

A flash of intelligence shot across Bob's face\ldots{}

When Pennyloaf returned she found her husband with his box of moulds and
medals on the table. He was turning over its contents, meditatively. On
the table there also lay a half-crown and a florin, as though Bob had
been examining these products of the Boyal Mint with a view to improving
the artistic quality of his amateur workmanship. He took up the coins
quietly as his wife entered and put them in his pocket.

``Mrs. Rendal's been at me again. Bob,'' Pennyloaf said, as she set down
her market-basket. ``You'll have to give her something to-day.''

{}He paid no attention, and Pennyloaf had a difficulty in bringing him
to discuss the subject of the landlady's demands. Ultimately, however,
he admitted with discontent the advisability of letting Mrs. Rendal have
something on account. Though it was Saturday night, he let hour after
hour go by and showed no disposition to leave home; to Pennyloaf's
surprise, he sat almost without moving by the fire, absorbed in thought.

Genuine respect for law is the result of possessing something which the
law exerts itself to guard. Should it happen that you possess nothing,
and that your education in metaphysics has been grievously neglected,
the strong probability is, that your mind will reduce the principle of
society to its naked formula: Get, by whatever means, so long as with
impunity. On that formula Bob Hewett was brooding; in the hours of this
Saturday evening he exerted his mind more strenuously than ever before
in the course of his life. And to a foregone result. Here is a man with
no moral convictions, with no conscious {}relations to society save
those which are hostile, with no personal affections; at the same time,
vaguely aware of certain faculties in himself for which life affords no
scope and encouraged in various kinds of conceit by the crass stupidity
of all with whom he associates. It is suggested to him all at once that
there is a very easy way of improving his circumstances, and that by
exercise of a certain craft with which he is perfectly familiar;---only,
the method happens to be criminal. ``Men who do this kind of thing are
constantly being caught and severely punished. Yes; men of a certain
kind; not Robert Hewett. Robert Hewett is altogether an exceptional
being; he is head and shoulders above the men with whom he mixes; he is
clever, he is remarkably good-looking. If any one in this world, of a
truth Robert Hewett may reckon on impunity when he sets his wits against
the law Why, his arrest and punishment is an altogether inconceivable
thing; he never in his life had a charge brought against him.''

{}Again and again it came back to that. Every novice in unimpassioned
crime has that thought, and the more self-conscious the man, the more
impressed with a sense of his own importance, so much the weightier is
its effect with him.

We know in what spirit John Hewett regarded rebels against the law. Do
not imagine that any impulse of that nature actuated his son. Clara
alone had inherited her father's instinct of revolt. Bob's temperament
was, in a certain measure, that of the artist; he felt without
reasoning; he let himself go whither his moods propelled him. Not a man
of evil propensities; entertain no such thought for a moment. Society
produces many a monster, but the mass of those whom, after creating
them, it pronounces bad are merely bad from the conventional point of
view; they are guilty of weaknesses, not of crimes. Bob was not
incapable of generosity; his marriage had, in fact, implied more of that
quality than you in the upper world can at all appreciate. He neglected
his wife, of {}course, for he had never loved her, and the burden of her
support was too great a trial for his selfishness. Weakness, vanity, a
sense that he has not satisfactions proportionate to his desert, a
strong temptation,---here are the data which, in ordinary cases, explain
a man's deliberate attempt to profit by criminality.

In a short time Pennyloaf began to be aware of peculiarities of
behaviour in her husband for which she could not account. Though there
appeared no necessity for the step, he insisted on their once more
seeking new lodgings, and, before the removal, he destroyed all his
medals and moulds.

``What's that for, Bob?'' Pennyloaf inquired.

``I'll tell you, and mind you hold your tongue about it. Somebody's been
saying as these things might get me into trouble. Just you be careful
not to mention to people that I used to make these kind of things.''

``But why should it get you into trouble?''

``Mind what I tell you, and don't ask questions. You're always too ready
at talking.''

{}His absences of an evening were nothing new, but his manner on
returning was such as Pennyloaf had never seen in him. He appeared to be
suffering from some intense excitement; his hands were unsteady; he
showed the strangest nervousness if there were any unusual sounds in the
house. Then he certainly obtained money of which his wife did not know
the source; he bought new articles of clothing, and in explanation said
that he had won bets. Pennyloaf remarked these things with uneasiness;
she had a fear during her lonely evenings for which she could give no
reason. Poor slow-witted mortal though she was, a devoted fidelity
attached her to her husband, and quickened wonderfully her apprehension
in everything that concerned him.

``Miss Snowdon came to-day. Bob,'' she had said, about a week after his
order with regard to Jane.

``Oh, she did? And did you tell her she'd better keep away?''

``Yes,'' was the dispirited answer.

{}``Glad to hear it.''

As for Jack Bartley, he never showed him self at the new lodgings.

Bob shortly became less regular in his attendance at the workshop. An
occasional Monday he had, to be sure, been in the habit of allowing
himself, but as the winter wore en he was more than once found straying
about the streets in mid-week. One morning towards the end of November,
as he strolled along High Holborn, a hand checked his progress; he gave
almost a leap, and tui'ned a face of terror upon the person who stopped
him. It was Clem,---Mrs. Snowdon. They had, of course, met casually
since Bob's marriage, and in progress of time the ferocious glances they
were wont to exchange had softened into a grin of half-friendly
recognition; Clem's behaviour at present was an unexpected revival of
familiarity. When he had got over his shock Bob felt surprised, and
expressed the feeling in a---``Well, what have \emph{you} got to say for
yourself?''

``You jumped as if I'd stuck a pin in you,'' {}replied Clem. ``Did you
think it was a copper?''

Bob looked at her with a surly smile. Though no one could have mistaken
the class she belonged to, Clem was dressed in a way which made her
companionship with Bob in his workman's clothing somewhat incongruous;
she wore a heavily trimmed brown hat, a long velveteen jacket, and
carried a little bag of imitation fur.

``Why ain't you at work?'' she added.

``Does Mrs. Pennyloaf Hewett know how you spend your time? ''

``Hasn't your husband taught you to mind your own business?''

Clem took the retort good-humouredly, and they walked on conversing. Not
altogether at his ease thus companioned, Bob turned out of the main
street, and presently they came within sight of the British Museum.

``Ever been in that place?'' Clem asked.

``Of course I have,'' he replied, with his air of superiority.

{}``I haven't. Is there anything to pay?---Let's go in for
half-an-honr.''

It was an odd freak, but Bob began to have a pleasure in this renewal of
intimacy; he wished he had been wearing his best suit. Years ago his
father had brought him on a public holiday to the Museum, and his
interest was chiefly excited by the collection of the Eoyal Seals. To
that quarter he first led his companion, and thence directed her towards
objects more likely to supply her with amusement; he talked freely, and
was himself surprised at the show of information his memory allowed him
to make,---desperately vague and often ludicrously wide of the mark, but
still a something of knowledge, retained from all sorts of chance
encounters by his capable mind. Had the British Museum been open to
visitors in the hours of the evening, or on Sundays, Bob Hewett would
possibly have been employing his leisure now-a-days in more profitable
pursuit\^{}. Possibly; one cannot say more than that; for the world to
which he belonged is above all {}a world of frustration, and only the
one man in half a million has fate for his friend.

Much Clem cared for antiquities; when she had wearied herself in
pretending interest, a seat in an unvisited corner gave her an
opportunity for more congenial dialogue.

``How's Mrs. Pennyloaf?'' she asked, with a smile of malice.

``How's Mr. What's-his-name Snowdon?'' was the reply.

``My husband's a gentleman. Good thing for me I had the sense to wait.''

``And for me too, I daresay.''

``Why ain't you at work? Got the sack?''

``I can take a day off if I like, can't I?''

``And you'll go 'ome and tell your wife as you've been working. I know
what you men are. What `ud Mrs. Pennyloaf say if she knew you was here
with me? You daren't tell her; you daren't!''

``I'm not doing any harm as I know of. I shall tell her if I choose, and
if I choose I shan't. I don't ask \emph{her} what I'm to do.''

``I daresay. And how does that mother of {}hers get on? And her brother
at the public? Nice relations for Mr. Bob Hewett. Do they come to tea on
a Sunday?''

Bob glared at her, and Clem laughed, showing all her teeth. From this
exchange of pleasantries the talk passed to various subjects,---the
affairs of Jack Bartley and his precious wife, changes in Clerkenwell
Close, then to Clem's own circumstances; she threw out hints of
brilliant things in store for her.

``Do you come here often?'' she asked at length.

``Can't say I do.''

``Thought p'raps you brought Mrs. Pennyloaf. When'll you be here
again?''

``Don't know,'' Bob replied, fidgeting and looking to a distance.

``I shouldn't wonder if I'm here this day next week,'' said Clem, after
a pause. ``You can bring Pennyloaf if you like.''

It was dinner-time, and they left the building together. At the end of
Museum Street they exchanged a careless nod and went their several ways.
