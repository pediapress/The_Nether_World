\chapter{The Bringer of Ill News}

\textsc{Knowing} the likelihood that Clara Hewett would go from home for
Bank-holiday, Sidney made it his request before he left Hanover Street
on Sunday night that Jane might be despatched on her errand at an early
hour next morning. At eight o'clock, accordingly, Snowdon went forth
with his grand-daughter, and, having discovered the street to which
Sidney had directed him, he waited at a distance whilst Jane went to
make her inquiries. In a few minutes the girl rejoined him.

``Miss Hewett has gone away,'' she reported.

``To spend the day, do you mean?'' was Snowdon's troubled question.

``No, she has left the house. She went yesterday, in the afternoon. It
was very {}sudden, the landlady says, and she doesn't know where she's
gone to.''

Jane had no understanding of what her information implied; seeing that
it was received as grave news, she stood regarding her grandfather
anxiously. Though Clara had passed out of her world since those first
days of illness, Jane held her in a memory which knew no motive of
retention so strong as gratitude. The thought of harm or sorrow coming
upon her protector had a twofold painfulness. Instantly she divined that
Clara was in some way the cause of Sidney Kirkwood's inability to go
into the country to-day. For a long time the two had been closely linked
in her reflections; Mrs. Peckover and Clem used constantly to exchange
remarks which made this inevitable. But not until now had Jane really
felt the significance of the bond. Of a sudden she had a throbbing at
her heart, and a confusion of mind which would not allow her to pursue
the direct train of thought naturally provoked by the visit she had just
paid. A turbid flood of ideas, of {}vague surmises, of apprehensions, of
forecasts, swept across her consciousness. The blood forsook her cheeks.
But that the old man began to move away, she could have remained thus
for many minutes, struggling with that new, half-understood thing which
was taking possession of her life.

The disappointment of the day was no longer simple, and such as a child
experiences. Nor ever from this hour onwards would Jane regard things as
she had been wont to do, with the simple feelings of childhood.

Snowdon walked on in silence until the street they had visited was far
behind them. Jane was accustomed to his long fits of musing, but now she
with difficulty refrained from questioning him. He said at length:

``Jane, I'm afraid we shall have to give up our day in the country.''

She assented readily, gladly; all the joy had gone out of the proposed
excursion, and she wished now to be by herself in quietness.

``I think I'll let you go home alone,'' Snowdon continued. ``I want to
see Mr. {}Kirkwood, and I daresay I shall find him in, if I walk on at
once.''

They went in different directions, and Snowdon made what speed he could
to Tysoe Street. Sidney had akeady been out, walking restlessly and
aimlessly for two or three hours. The news he now heard was the
half-incredible fulfilment of a dread that had been torturing him
through the night. No calamity is so difficult to realise when it
befalls as one which has haunted us in imagination.

``That means nothing! '' he exclaimed, as if resentfully. ``She was
dissatisfied with the lodging, that's all. Perhaps she's already got a
place. I daresay there's a note from her at home this morning.''

``Shall you go and see if there is?'' asked Snowdon, allowing, as usual,
a moment's silence to intervene.

Sidney hesitated, avoiding the other's look.

``I shall go to that house first of all, I think. Of course I shall hear
no more than they told Jane; but''{{------}}

{}He took a deep breath.

``Yes, go there,'' said Snowdon; ``but afterwards go to the Hewetts'. If
she \emph{hasn't} written to them, or let them have news of any kind,
her father oughtn't to be kept in ignorance for another hour.''

``He ought to have been told before this,'' replied Sidney in a thick
under-voice. ``He ought to have been told on Saturday. And the blame'll
be mine.''

It is an experience familiar to impulsive and self-confident men that a
moment's crisis may render scarcely intelligible a mode of thought or
course of action which till then one had deemed perfectly rational.
Sidney, hopeless in spite of the pretences he made, stood aghast at the
responsibility he had taken upon himself. It was so obvious to him now
that he ought to have communicated to John Hewett without loss of time
the news which Mrs. Hewett brought on Saturday mornino;. But could he be
sure that John was still in ignorance of Clara's movements? Was it not
all but certain that Mrs. Hewett {}must have broken the news before
this? If not, there lay before him a terrible duty.

The two went forth together, and another visit was paid to the
lodging-house. After that Sidney called upon Mrs. Tubbs, and made a
simple inquiry for Clara, with the anticipated result.

``You won't find her in this part of London, it's my belief,'' said the
woman significantly.

``She's left the lodgings as she took---so much I know. Never meant to
stay there, not she! You're a friend of her father's, mister?''

Sidney could not trust himself to make a reply. He rejoined Snowdon at a
little distance, and expressed his intention of going at once to
Clerkenwell Close.

``Let me see you again to-day,'' said the old man sadly.

Sidney promised, and they took leave of each other. It was now nearing
ten o'clock. In the Close an organ was giving delight to a great crowd
of children, some of them wearing holiday garb, but most clad in the
native rags which served them for all seasons and all days. {}The volume
of clanging melody fell with torture upon Kirkwood's ear, and when he
saw that the instrument was immediately before Mrs. Peckover's house, he
stood aside in gloomy impatience, waiting till it should move away. This
happened in a few minutes. The house door being open, he walked straight
upstairs.

On the landing he confronted Mrs. Hewett; she started on seeing him, and
whispered a question. The exchange of a few words apprised Sidney that
Hewett did not even know of Clara's having quitted Mrs. Tubbs'.

``Then I must tell him everything,'' he said. To put the task upon the
poor woman would have been simple cowardice. Merely in hearing his news
she was blanched with dread. She could only point to the door of the
front room,---the only one rented by the family since Jane Snowdon's
occupation of the other had taught them to be as economical in this
respect as their neighbours were.

Sidney knocked and entered. Two months had passed since his latest
visit, and he {}observed that in the meantime everything had become more
sqnalid. The floor, the window, the furniture, were not kept so clean as
formerly,---inevitable result of the overcrowding of a room; the air was
bad, the children looked untidy. The large bed had not been set in order
since last night; in it lay the baby, crying as always, ailing as it had
done from the day of its birth. John Hewett was engaged in mending one
of the chairs, of which the legs had become loose. He looked with
surprise at the visitor, and at once averted his face sullenly.

``Mr. Hewett,'' Kirkwood began, without form of greeting, ``on Saturday
morning I heard something that I believe I ought to have let you know at
once. I felt, though, that it was hardly my business; and somehow we
haven't been quite so open with each other just lately as we used to
be.''

His voice sank. Hewett had risen from his crouching attitude, and was
looking him full in the face with eyes which grew momently darker and
more hostile.

{}``Well? Why are you stopping? What have you got to say?''

The words come from a dry throat; the effort to pronounce them clearly
made the last all but violent.

``On Friday night,'' Sidney resumed, his own utterance uncertain,
``Clara left her place. She took a room not far from Upper Street, and I
saw her, spoke to her. She'd quarrelled with Mrs. Tubbs. I urged her to
come home, but she wouldn't listen to me. This morning I've been to try
and see her again, but they tell me she went away yesterday afternoon. I
can't find where she's living now.''

Hewett took a step forward. His face was so distorted, so fierce, that
Sidney involuntarily raised an arm, as if to defend himself.

``An' it's you as comes tellin' me this!'' John exclaimed, a note of
anguish blending with his fury. ``You have the face to stand there an'
speak like that to me, when you know it's all your own doing! Who was
the cause as the girl went away from 'ome? Who was it, I say? Haven't
been as friendly as we {}used to be, haven't we? An why? Haven't I seen
it plainer an' plainer what you was thinkin' when you told me to let her
have her own way? I spoke the truth then,---`cause I felt it; an' I was
fool enough, for all that, to try an' believe I was in the wrong. Now
you come an stand before me---why, I couldn't a' thought there was a man
had so little shame in him!''

Mrs. Hewett entered the room; the loud angry voice had reached her ears,
and in spite of terror she came to interpose between the two men.

``Do you know what he's come to tell me?'' cried her husband. ``Oh, you
do! He's been tryin' to talk you over, has he? You just answer to me,
an' tell the truth. Who was it persuaded me to let Clara go from 'ome?
Who was it come here an' talked an' talked till he got his way? He knew
what `ud be the end of it,---he knew, I tell you,---an' it's just what
he wanted. Hasn't he been drawin' away from us ever since the girl left?
I saw it all that night when he came here persuadin' {}me, an' I told it
him plain. He wanted to a' done with her, and to a' done with us. Am I
speakin' the truth or not?''

``Why should he think that way, John?'' pleaded the woman, faintly.
``You know very well as Clara `ud never listen to him. What need had he
to do such things?''

``Oh, yes, I'm wrong! Of course I'm wrong! You always did go against me
when there was anything to do with Clara. She'd never listen to him? No,
of course she wouldn't, an' he couldn't rest till he saw her come to
harm. What do you care, either? She's no child of yours. But I tell you
I'd see you an' all your children beg an' die in the streets rather than
a hair o' my own girl's head should be touched!''

Indulgence of his passion was making a madman of him. Never till now had
he uttered an unfeeling word to his wife, but the look with which he
accompanied this brutal speech was one of fiery hatred.

``Don't turn on \emph{her}'' cried Sidney, with bitterness. ``Say what
you like to me, and {}believe the worst you can of me; I shouldn't have
come here if I hadn't been ready to bear everything. It's no good
speaking reason to you now, but maybe you'll understand some day.''

``Who know's as she's come to harm?'' urged Mrs. Hewett. ``Nobody can
say it of her for certain, yet.''

``I'd have told him that, if he'd only listened to me and given me
credit for honesty,'' said Kirkwood. ``It is as likely as not she's gone
away just because I angered her on Saturday. Perhaps she said to herself
she'd have done with me once for all. It would be just her way.''

``Speak another word against my girl,'' Hewett shouted, misinterpreting
the last phrase, ``an' I'll do more than say what I think of you,---old
man though they call me! Take yourself out of this room; it was the
worst day of my life that ever you came into it. Never let me an you
come across each other again. I hate the sight of you, an' I hate the
sound of your voice!''

{}The animal in Sidney Kirkwood made it a terrible minute for him as he
turned away in silence before this savage injustice. The veins upon his
forehead were swollen; his clenched teeth gave an appearance of ferocity
to his spirited features. With head bent, and shoulders quivering as if
in supreme muscular exertion, he left the room without another word.

In a few minutes Hewett also quitted the house. He went to the
luncheon-bar in Upper Street, and heard for the first time Mrs. Tubbs's
rancorous surmises. He went to Clara's recent lodgings; a girl of ten
was the only person in the house, and she could say nothing more than
that Miss Hewett no longer lived there. Till midway in the afternoon
John walked about the streets of Islington, Highbury, Hoxton,
Clerkenwell, impelled by the unreasoning hope that he might see Clara,
but also because he could not rest in any place. He was half-conscious
now of the madness of his behaviour to Kirkwood, but this only confirmed
him in hostility to the {}young man; the thought of losing Clara was
anguish intolerable, yet with it mingled a bitter resentment of the
girl's cruelty to him. And all these sources of misery swelled the
current of rebellious feeling which had so often threatened to sweep his
life into wreckage. He was Clara's father, and the same impulse of
furious revolt which had driven the girl to recklessness, now inflamed
him with the rage of despair.

On a Bank-holiday only a few insignificant shops remain open even in the
poor districts of London; sweets you can purchase, and tobacco, but not
much else that is sold across an ordinary counter. The more noticeable
becomes the brisk trade of public-houses. At the gin-shop centres the
life of each street; here is a wide door and a noisy welcome, the more
attractive by contrast with the stretch of closed shutters on either
hand. At such a door, midway in the sultry afternoon, John Hewett
paused. To look at his stooping shoulders, his uncertain swaying this
way and that, his flushed, perspiring face, you might {}have taken him
for one who had already been drinking. No; it was only a struggle
between his despairing wretchedness and a lifelong habit of mind. Not
difficult to foresee which would prevail; the public-house always has
its doors open in expectation of such instances. With a gesture which
made him yet more like a drunken man, he turned from the pavement and
entered\ldots{}.

About nine o'clock in the evening, just when Mrs. Hewlett had put the
unwilling children to bed, and had given her baby a sleeping-dose,---it
had cried incessantly for eighteen hours,---the door of the room was
pushed open. Her husband came in. She stood looking at him,---unable to
credit the evidence of her eyes.

``John!''

She laid her hand upon him and stared into his face. The man shook her
off, without speaking, and moved staggeringly forward. Then he turned
round, waved his arm, and shouted:

{}``Let her go to the devil! She cares nothing for her father.''

He threw himself upon the bed, and soon sank into drunken sleep.
