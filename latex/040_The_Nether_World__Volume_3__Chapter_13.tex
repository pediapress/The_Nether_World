\hypertarget{headerContainer}{}
\hypertarget{navigationHeader}{}
\protect\hypertarget{headerprevious}{}{←\href{/wiki/The_Nether_World/Volume_3/Chapter_12}{Volume
3, Chapter 12}}

\textbf{\protect\hypertarget{header_title_text}{}{\href{/wiki/The_Nether_World}{The
Nether World}}} \emph{by
\href{/wiki/Author:George_Gissing}{\protect\hypertarget{header_author_text}{}{{George
Gissing}}}}\\
\protect\hypertarget{header_section_text}{}{Volume 3, Chapter 13}

\hypertarget{navigationNotes}{}

\hypertarget{ws-data}{}
\protect\hypertarget{ws-article-id}{}{2414259}\protect\hypertarget{ws-title}{}{\href{/wiki/The_Nether_World}{The
Nether World} --- \emph{Volume 3, Chapter
13}}\protect\hypertarget{ws-author}{}{George Gissing}

{\protect\hypertarget{285}{}{}}

{CHAPTER XIII.}

{JANE.}

\textsc{``The} poisoning business startled me. I shouldn't at all wonder
if I had a precious narrow squeak of something of the kind myself before
I took my departure; in fact, a sort of fear of the animal made me
settle things as sharp as I could. Let me know the result of the trial.
``Wonder whether there'll be any disagreeable remarks about a certain
acquaintance of yours, detained abroad on business? Better send me
newspapers,---same name and address. . . . But I've something
considerably more important to think about. . . . A big thing; I
scarcely dare tell you how big. I stand to win \$2,000,000 I . . . Not a
soul outside suspects the ring. When I tell you that R. S. N. is in it,
you'll see that I've struck the right ticket this time. . . . Let me
hear about Jane. If all goes well here, and you manage that little
business, you shall have \$100,000, just fur
{\protect\hypertarget{286}{}{}} house-furnishing, you know. I suppose
you'll have your partnership in a few months?"

Extracts from a letter, with an American stamp, which Mr. Scawthortie
read as he waited for his breakfast. It was the end of October, and cool
enough to make the crackling fire grateful. Having mused over the
epistle, our friend took up his morning paper and glanced at the report
of criminal trials. Whilst he was so engaged his landlady entered,
carrying a tray of appetising appearance.

``Good-morning, Mrs. Byass,'' he said, with much friendliness. Then, in
a lower voice, ``There's a fuller report here than there was in the
evening paper. Perhaps you looked at it?''

``Well, yes, sir; I thought you wouldn't mind,'' replied Bessie,
arranging the table.

``She'll be taken care of for three years, at all events.''

``If you'd seen her that day she came here after Miss Snowdon, you'd
understand how glad I feel that she's out of the way. I'm sure I've been
uneasy ever since. If ever there comes a rather loud knock at there I
begin to tremble; I do indeed. I don't think I shall ever get over it.''

{\protect\hypertarget{287}{}{}} ``I daresay Miss Snowdon will be easier
in mind?''

``I shouldn't wonder. But she won't say anything about it. She feels the
disgrace so much, and I know it's almost more than she can do to go to
work, just because she thinks they talk about her.''

``Oh, that'll very soon pass over. There's always something new
happening, and people quickly forget a case like this.''

Bessie withdrew, and her lodger addressed himself to his breakfast.

He had occupied the rooms on the first floor for about a year and a
half. Joseph Snowdon's proposal to make him acquainted with Jane had not
been carried out, Scawthorne deeming it impracticable; but when a year
had gone by, and Scawthorne, as Joseph's confidential correspondent, had
still to report that Jane maintained herself in independence, he one day
presented himself in Hanover Street, as a total stranger, and made
inquiry about the rooms which a card told him were to let. His improved
position allowed him to live somewhat more reputably than in the Chelsea
lodging, and Hanover Street would suit him
{\protect\hypertarget{288}{}{}} well enough uutil lie obtained the
promised partnership. Admitted as a friend to Mr. Percival's house in
Highbury, he had by this time made the acquaintance of Miss Lant, whom,
by the exercise of his agreeable qualities, he one day led to speak of
Jane Snowdon. Miss Lant continued to see Jane, at long intervals, and
was fervent in her praise as well as in compassionating the trials
through which she had gone. His position in Mr. Percival's office of
course made it natural that Scawthorne should have a knowledge of the
girl's story. When he had established himself in Mrs. Byass's rooms, he
mentioned the fact casually to his friends, making it appear that, in
seeking lodgings, he had come upon these by haphazard.

He could not but feel something of genuine interest in a girl who, for
whatever reason, declined a sufficient allowance and chose to work for
her living. The grounds upon which Jane took this decision were
altogether unknown to him until an explanation came from her father.
Joseph, when news of the matter reached him, was disposed to entertain
suspicions; with every care not to betray his own whereabouts, he wrote
to Jane, and in due {\protect\hypertarget{289}{}{}} time received a
reply, in which Jane told him truly her reasons for refusing the money.
These Joseph communicated to Scawthorne, and the latter's interest was
still more strongly awakened.

He was now on terms of personal acquaintance, almost of friendship, with
Jane. Miss Lant, he was convinced, did not speak of her too praisingly.
Not exactly a pretty girl, though far from displeasing in countenance;
very quiet, very gentle, with much natural refinement. Her air of
sadness---by no means forced upon the vulgar eye, but unmistakable when
you studied her---was indicative of faithful sensibilities. Scawthorne
had altogether lost sight of Sidney Kirkwood and of the Hewetts; he knew
they were all gone to a remote part of London, and more than this he had
no longer any care to discover. On excellent terms with his landlady, he
skilfully elicited from her now and then a confidential remark with
regard to Jane~; of late, indeed, he had established somethinsr like a
sentimental understanding with the good Bessie, so that, whenever he
mentioned Jane, she fell into a pleasant little flutter, feeling that
{\protect\hypertarget{290}{}{}} she understood what was in progress. . .
. Why not?---he kept asking himself. Joseph Suowdon (who addressed his
letters to Hanover Street in a feigned hand) seemed to have an
undeniable affection for the girl, and was constant in his promises of
providing a handsome dowry. The latter was not a point of such
importance as a few years ago, but the dollars would be acceptable. And
then, the truth was, Scawthorne felt himself more and more inclined to
put a certain question to Jane, dowry or none. . . .

Yes, she felt it as a disgrace, poor girl! When she saw the name
``Snowdon'' in the newspaper, in such a shameful and horrible
connection, her impulse was to flee, to hide herself. It was dreadful to
go to her work and hear the girls talking of this attempted murder. The
new misery came upon her just as she was regaining something of her
natural spirits, after long sorrow and depression which had affected her
health. But circumstances, now as ever, seemed to plot that at a
critical moment of her own experience she should be called out of
herself and constrained to become the consoler of others.

{\protect\hypertarget{291}{}{}} For some months the domestic peace of
Mr. and Mrs. By ass had been gravely disturbed. Unlike the household at
Crouch End, it was to prosperity that Sam and his wife owed their
troubles. Year after year Sam's position had improved; he was now in
receipt of a salary which made---or ought to have made---things at home
very comfortable. Though his children were now four in number, he could
supply their wants. He could buy Bessie a new gown without very grave
consideration, and could regard his own shiny top-hat, when he donned it
in the place of one that was really respectable enough, without twinges
of conscience.

But Sam was not remarkable for wisdom; indeed, had he been anything more
than a foolish calculating-machine, he would scarcely have thriven as he
did in the City. "When he had grown accustomed to rattling loose silver
in his pocket, the next thing, as a matter of course, was that he
accustomed himself to pay far too frequent visits to City bars. On
certain days in the week he invariably came home with a very red face
and a titubating walk; when Bessie received him angrily, he defended
himself on the great plea of business neces-
{\protect\hypertarget{292}{}{}} sities. As a town traveller there was no
possibility, he alleged, of declining invitations to refresh himself;
just as incumbent upon him was it to extend casual hospitality to those
with whom he had business.

``Business! Fiddle!'' cried Bessie. ``All you City fellows are the same.
You encourage each other in drink, drink, drinking whenever you have a
chance, and then you say it's all a matter of business. I won't have you
comiog home in that state, so there! I won t have a husband as drinks~!
Why, you can't stand straight.''

``Can't stand straight!'' echoed Sam, with vast scorn. ``Look here!''

And he shouldered the poker, with the result that one of the globes on
the chandelier came in shivers about his head. This was too much. Bessie
fumed, and for a couple of hours the quarrel was unappeasable.

Worse w\^{}as to come. Sam occasionally stayed out very late at night,
and on his return alleged a ``business appointment.'' Bessie at length
refused to accept these excuses~; she couldn't and wouldn't believe
them.

``Then don't!'' shouted Sam. "And understand that I shall come home just
when I like.

{\protect\hypertarget{293}{}{}} If you make a bother I wont come home at
all, so there you have it!"

``You're a bad husband and a beast!'' was Bessie's retort.

Shortly after that Bessie received information of such orrave misconduct
on her husband's part that she all but resolved to forsake the house,
and with the children seek refuge under her parents' roof at Woolwich.
Sam had been seen in indescribable company; no permissible words would
characterise the individuals with whom he had roamed shamelessly on the
pavement of Oxford Street. When he next met her, quite sober and with
exasperatingly innocent expression, Bessie refused to open her lips.
Neither that eveniog nor the next would she utter a word to him,---and
the effort it cost her was tremendous. The result was, that on the third
evening Sam did not appear.

It was a week after Clem's trial. Jane had been keeping to herself as
much as possible, but, having occasion to go down into the kitchen late
at night, she found Bessie in tears, utterly miserable.

``Don't bother about me!'' was the reply to her sympathetic question.
"You've got {\protect\hypertarget{294}{}{}} your own upsets to think of.
You might have come to speak to me before this---but never mind. It's
nothing to you."

It needed much coaxing to persuade her to detail Sam's enormities, but
she found much relief when she had done so, and wept more copiously than
ever.

``It's nearly twelve o'clock, and there's no sign of him. Perhaps he
won't come at all. He's in bad company, and if he stays away all night
I'll never speak to him again as long as I live. Oh, he's a beast of a
husband, is Sam!''

Sam came not. All through that night did Jane keep her friend company,
for Sam came not. In the morning a letter, addressed in his well-known
commercial hand. Bessie read it and screamed. Sam wrote to her that he
had accepted a position as country traveller, and perhaps he might be
able to look in at his home on that day month.

Jane could not go to work. The case had become very serious indeed;
Bessie was in hysterics; the four children made the roof ring with their
lamentations. At this juncture Jane put forth all her beneficent energy.
{\protect\hypertarget{295}{}{}} It happened that Bessie was just now
servant-less. There was Mr. Scawthorne's breakfast only half prepared;
Jane had to see to it herself, and herself take it upstairs. Then Bessie
must go to bed, or assuredly she would be so ill that unheard-of
calamities would befall the infants. Jane would have an eye to
everything; only let Jane be trusted.

The miserable day passed; after trying in vain to sleep, Bessie walked
about her sitting-room with tear-swollen face and rumpled gown, always
thinking it possible that Sam had only played a trick, and that he would
come. But he came not, and again it was night.

At eio\^{}ht o'clock Mr. Scawthorne's bell rang. Impossible for Bessie
to present herself; Jane would go. She ascended to the room which had
once---ah, once!---been her own parlour, knocked and entered.

``I---I wished to speak to Mrs. By ass,'' said Scawthorne, appearing for
some reason or other embarrassed by Jane's presenting herself.

``Mrs. Byass is not at all well, sir. But I'll let her know''------

``No, no~; on no account.''

``Can't I get you anything, sir?''

{\protect\hypertarget{296}{}{}} ``Miss Snowdon,---might I speak with you
for a few moments?''

Jane feared it might be a complaint. In a perfectly natural way she
walked forward. Scawthorne came in her direction, and---closed the door.

The interview lasted ten minutes, then Jane came forth and w\^{}ith a
light, quick step ran up to the floor above. She did not enter the room,
however, but stood with her hand on the door, in the darkness. A minute
or two, and with the same light, hurried step, she descended the stairs,
sprang past the lodger's room, sped down to the kitchen. Under other
circumstances Bessie must surely have noticed a strangeness in her look,
in her manner; but to-night Bessie had thought for nothing but her owm
calamities.

Another day, and no further news from Sam. The next morning, instead of
going to work (the loss of wages was most serious, but it couldn't be
helped), Jane privately betook herself to Sam's house of business. Mrs.
Byass was ill; would they let her know Mr. Byass's address, that he
might immediately be communicated with? The information was
{\protect\hypertarget{297}{}{}} readily supplied; Mr. Byass was no
farther away, at present, tlian St. Albans. Forth into the street again,
and in search of a policeman. ``Will you please to tell me what station
I have to go to for St. Albans?'' Why, Moorgate Street would do~; only a
few minutes' walk away. On she hastened. ``What is the cost of a return
ticket to St. Albans, please?'' Three-and-sevenpence. Back into the
street again; she must now look for a certain sign, indicating a certain
place of business. With some little trouble it is found; she enters a
dark passage, and comes before a counter, upon which she lays --- a
watch, her grand-father's old watch. ``How much?'' ``Four shillings,
please.'' She deposits a halfpenny, and receives four shillings,
together with a ticket. Now for St. Albans.

Sam~! Sam~! Ay, well might he turn red and stutter and look generally
foolish when that quiet little girl stood before him in his
``stock-room'' at the hotel. Her words were as quiet as her look. ``Til
write her a letter,'' he cries. "Stop~; you shall take it back. I can't
give up the job at once, but you may tell her I'm up to no harm. Where's
the pen? {\protect\hypertarget{298}{}{}} Where's the cursed ink?" And
she takes the letter.

``Why, you've lost a day's work, Jane! She gave you the money for the
journey, I suppose?''

``Yes, yes, of course.''

" Tell her she's not to make a fool of herself in future."

``No, I shan't say that, Mr. Byass. But I'm half-tempted to say it to
some one else!''

It was the old, happy smile, come back for a moment~; the voice that had
often made peace so merrily. The return journey seemed short, and with
glad heart-beating she hastened from the City to Hanover Street.

Well, well; of course it would all begin over again; Jane herself knew
it. But is not all life a struggle onward from compromise to compromise,
until the day of final pacification?

Through that winter she lived with a strange secret in her mind, a
secret which was the source of singularly varied feelings,---of
astonishment, of pain, of encouragement, of apprehension, of grief. To
no one could she speak of it; no one could divine its existence---no one
save the person to w\^{}hom she owed this surprising novelty in her
experience. She would have {\protect\hypertarget{299}{}{}} given much to
be rid of it; and yet, again, might she not legitimately accept that
pleasure which at times came of the thought?---the thought that, as a
woman, her qualities were of some account in the world.

She did her best to keep it out of her consciousness, and in truth had
so many other things to think about that it was seldom she really had
trouble with it. Life was not altogether easy; regular work was not
always to be kept; there was much need of planning and pinching, that
her independence might suffer no wound. Bessie Byass was always in arms
against that same independent spirit; she scoffed at it, assailed it
with treacherous blandishment, made direct attacks upon it.

``I must live in my own way, Mrs. Byass. I don't want to have to leave
you.''

And if ever life seemed a little too hard, if the image of the past grew
too mournfully persistent, she knew Avhere to go for consolation. Let us
follow\^{} her, one Saturday afternoon early in the year.

In a poor street in Clerkenwell was a certain poor little shop,---built
out as an afterthought from an irregular lump of houses; a shop with
{\protect\hypertarget{300}{}{}} a room behind it and a cellar below; no
more. Here was sold second-hand clothing, women's and children's. No
name over the front, but neighbours would have told you that it was kept
by one Mrs. Todd, a young widow with several children. Mrs. Todd, not
long ago, used to have only a stall in the street; but a lady named Miss
Lant helped her to start in a more regular way of business.

``And does she carry it on quite by herself?'' No; with her lived
another young woman, also a widow, who had one child. Mrs. Hewett, her
name. She did sewing in the room behind, or attended to the shop when
Mrs. Todd was away making purchases.

There Jane Snowdon entered. The clothing that hung in the window made it
very dark inside; she had to peer a little before she could distinguish
the person who sat behind the counter. ``Is Penny loaf in, Mrs. Todd?''

``Yes, Miss. Will you walk through?''

The room behind is lighted from the ceiling. It is heaped with the most
miscellaneous clothing. It contains two beds, some shelves with
crockery, a table, some chairs,---but it would have taken you a long
time to note all these {\protect\hypertarget{301}{}{}} details, so
huddled together was everything. Part of the general huddling were five
children, of various ages~; and among them, very busy, sat Pennyloaf.

``Everything going on well?'' was Jane's first question.

``Yes, Miss.''

``Then I know it isn't. Whenever you call me `Miss,' there's something
wrong~; I've learnt that.''

Pennyloaf smiled, sadly but with affection in her eyes. " Well, I have
been a bit low, an' that's the truth. It takes me sometimes, you know.
I've been thinkin', when I'd oughtn't."

``Same with me, Pennyloaf. We can't help thinking, can we~? What a good
thing if we'd nothing more to think about than these children! Where's
little Bob? Why, Bob, I thought you were old clothes; I did, really! You
may well laugh!''

The laughter was merry, and Jane encouraged it, inventing all sorts of
foolish jokes. ``Pennyloaf, I wish you'd ask me to stay to tea.''

``Then that I will. Miss Jane, an' gladly. Would you like it soon?''

"No; in an hour will do, won't it? Give {\protect\hypertarget{302}{}{}}
rue something that wants sewing, a really hard bit, something that'll
break needles. Yes, that'll do. Where's Mrs. Todd's thimble? Now we're
all going to be comfortable, and we'll have a good talk."

Pennyloaf found the dark thoughts slip away insensibly. And she talked,
she talked,---where was there such a talker as Pennyloaf now-a-days,
when she once began?

Mr. Byass was not very willing, after all, to give up his country
travelling. That his departure on that business befell at a moment of
domestic quarrel was merely chance; secretly he had made the arrangement
with his firm some weeks before. The penitence which affected him upon
Jane's appeal could not be of abiding result~; for, like all married men
at a certain point of their lives, he felt heartily tired of home aud
wished to see the world a little. Hanover Street heard endless
discussions of the point, between Sam and Bessie, between Bessie and
Jane, between Jane and Sam, between all three together. And the upshot
was, that \^{}h. Byass gained his point. For a time he would go on
country journeys. {\protect\hypertarget{303}{}{}} Bessie assented
sullenly, but, strange to say, she had never been in better spirits than
on the day after this decision had been arrived at.

On that day, however,---it was early in March,---an annoying incident
happened. Mr. Scawthorne, who always dined in town and seldom returned
to his lodgings till late in the evening, rang his bell about eight
o'clock and sent a message by the servant that he wished to see Mrs.
Byass. Bessie having come up, he announced to her with gravity that his
tenancy of the rooms would be at an end in a fortnight. Various
considerations necessitated his living in a different part of London.
Bessie frankly lamented~; she would never again find such an estimable
lodger. But, to be sure, Mr. Scawthorne had prepared her for this, three
months ago. Well, what must be, must be.

``Is Miss Snowdon in the house, Mrs. Byass?'' Scawthorne went on to
inquire.

``Miss Snowdon? Yes.''

``This letter from America, which I found on coming in, contains news
she must hear---disagreeable news, I'm sorry to say.''

``About her father?'' Bessie inquired anxiously.

{\protect\hypertarget{304}{}{}} Scawttiorne nodded a grave and
confidential affirmative. He had never given Mrs. Byass reason to
suppose that he knew anything of Joseph's whereabouts, but Bessie's
thoughts naturally turned in that direction.

The news comes to me by chance," he continued. ``I think I ought to
communicate it to Miss Snowdon privately, and leave her to let you know
what it is, as doubtless she will. Would it be inconvenient to you to
let me have the use of your parlour for five minutes?''

``I'll go and light the gas at once, and tell Miss Snowdon.''

``Thank you, Mrs. Byass.''

He was nervous, a most unusual thing with him. Till Bessie's return he
paced the room irregularly, chewing the ends of his moustache. When it
was announced to him that the parlour was ready he went down, the letter
in his hand. At the half-open door came a soft knock. Jane entered.

She showed signs of painful agitation.

"Will you sit down. Miss Snowdon? It happens that I have a correspondent
in the United States, who has lately had---had business relations with
Mr. Joseph Snowdon, your {\protect\hypertarget{305}{}{}} father. On
returning this evening I found a letter from my friend, in which there
is news of a distressing kind."

He paused. What he was about to say was---for once---the truth. The
letter, however, came from a stranger, a lawyer in Chicago.

``Your father, I understand, has lately been engaged in---in commercial
speculation on a great scale. His enterprises have proved unfortunate.
One of those fiu ancial crashes which are common in America caused his
total ruin.''

Jane drew a deep breath.

``I am sorry to say that is not all. The excitement of the days when his
fate was hanging in the balance led to illness---fatal illness. He died
on the sixth of February.''

Jane, with her eyes bent down, was motionless. After a pause, Scawthorne
continued~:

``I will speak of this with Mr. Percival to-morrow, and every inquiry
shall be made---on your behalf.''

``Thank you, sir.''

She rose, very pale, but with more self-command than on entering the
room. The latter part of his communication seemed to have affected her
as a relief.

{\protect\hypertarget{306}{}{}} ``Miss Snowdon,---if you would allow me
to say a few more words. You will remember I mentioned to you that
tliere was a prospect of my becoming a partner in tlie firm wbich I have
hitherto served as clerk. A certain examination had to be passed, that I
might be admitted a solicitor. That is over; in a few days my position
as a member of the firm will be assured.''

Jane waited, her eyes still cast down.

"I feel that it may seem to you an ill-chosen time; but the very fact
that I have just been the bearer of such sad news impels me to speak. I
cannot keep the promise that I would never revive the subject on which I
spoke to you not long ago. Forgive me; I must ask you again if you
cannot think of me as I wish? Miss Snowdon, will you let me devote
myself to making your life happy? It has always seemed to me that if I
could attain a position such as I now have, there would be little else
to ask for. I began life poor and half-educated, and you cannot imagine
the difficulties I have overcome. But if I go away from this house, and
leave you so lonely, living such a hard life, there will be very
{\protect\hypertarget{307}{}{}} little satisfaction for me in my
success. Let me try to make for you a happiness such as you merit. It
may seem as if we were very slightly acquainted, but I know you well
enough to esteem you more highly than any woman I ever met, and if you
could but think of me"---

He was sincere. Jane had brought out the best in him. With the death of
Snow- don all his disreputable past seemed swept away, and he had no
thought of anything but a decent rectitude, a cleanly enjoyment of
existence, for the future. But Jane was answering:

``I can't change what I said before, Mr. Scawthorne. I am very content
to live as I do now. I have friends I am very fond of. Thank you for
your kindness,---but I can't change.''

Without intending it, she ceased upon a word which to her hearer
conveyed a twofold meaning. He understood~; offer what he might, it
could not tempt her to forget the love which had been the best part of
her life. She was faithful to the past, and unchanging.

Mrs. Byass never suspected the second purpose for which her lodger had
desired to speak {\protect\hypertarget{308}{}{}} with Jane this evening.
Scawthorne in due time took his departure, with many expressions of
good-will, many assurances that nothing could please him better than to
be of service to Bessie and her husband.

``He wished me to say good-bye to you for him,'' said Bessie, when Jane
came back from her work.

So the romance in her life was over. Michael Snowdon's wealth had melted
away; with it was gone for ever the hope of realising his high projects.
All passed into the world of memory, of dream,---all save the spirit
which had ennobled him, the generous purpose bequeathed to those two
hearts which had loved him best.

To his memory all days were sacred; but one, that of his burial, marked
itself for Jane as the point in each year to which her life was
directed, the saddest, yet bringing with it her supreme solace.

A day in early spring, cloudy, cold. She left the workroom in the
dinner-hour, and did not return. But instead of going to Hanover Street,
she walked past Islington Green, all {\protect\hypertarget{309}{}{}}
along Essex Eoad, northward thence to Stoke Newington, and so came to
Abney Park Cemetery; a long way, but it did not weary her.

In the cemetery she turned her steps to a grave with a plain headstone.
Before leaving England, Joseph Snowdon had discharged this duty. The
inscription was simply a name, with dates of birth and death.

And, as she stood there, other footsteps approached the spot. She looked
up, with no surprise, and gave her hand for a moment. On the first
anniversary the meeting had been unanticipated; the same thought led her
and Sidney to the cemetery at the same hour. This was the third year,
and they met as if by understanding, though neither had spoken of it.

When they had stood in silence for a while, Jane told of her father's
death and its circumstances. She told him, too, of Pennyloafs humble
security.

``You have kept well all the year?'' he asked.

``And you too, I hope?''

Then they bade each other good-bye. . . .

In each life little for congratulation. He
{\protect\hypertarget{310}{}{}}with the ambitions of his youth
frustrated; neither an artist, nor a leader of men in the battle for
justice. She, no saviour of society by the force of a superb example; no
daughter of the people, holding wealth in trust for the people's needs.
Yet to both was their work given. Unmarked, unencouraged save by their
love of uprightness and mercy, they stood by the side of those more
hapless, brought some comfort to hearts less courageous than their own.
Where they abode it was not all dark. Sorrow certainly awaited them,
perchance defeat in even the humble aims that they had set themselves;
but at least their lives would remain a protest against those brute
forces of society which fill with wreck the abysses of the nether world.

~

THE END.
