{}

{CHAPTER VII.}

A VISION OF NOBLE THINGS.

\textsc{He} slept but for an hour or two, and even then with such
disturbance of fitful dreams that he could not be said to rest. At the
earliest sound of movements in the house he rose and went out into the
morning air. There had fallen a heavy shower just after sunrise, and the
glory of the east was still partly veiled with uncertain clouds.
Heedless of weather-signs, Sidney strode away at a great pace, urged by
his ungovernable thoughts. His state was that miserable one in which a
man repeats for the thousandth time something he has said, and torments
himself with devising possible and impossible interpretations thereof.
Through the night he had done nothing but imagine what significance Jane
might have attached to his words {}about Clara Hewett. Why had he spoken
of Clara at all? One moment he understood his reasons, and approved
them; the next he was at a loss to account for such needless revival of
a miserable story. How had Jane interpreted him? And was it right or
wrong to have paused when on the point of confessing that he loved her?

Rain caught him at a distance from home, and he returned to breakfast in
rather a cheerless plight. He found that Michael was not feeling quite
himself, and would not rise till midday. Jane had a look of anxiety, and
he fancied she behaved to him with a constraint hitherto unknown. The
fancy was dispelled, however, when, later in the morning, she persuaded
him to bring out his sketch-book, and suggested points of view for a
drawing of the farm that had been promised to Mr. Pammenter. Himself
unable to recover the tone of calm intimacy which till yesterday had
been natural between them, Sidney found himself studying the girl,
seeking to surprise some proof that she too was no longer the {}same,
and only affected this unconsciousness of change. There was, perhaps, a
little less readiness in her eyes to meet his, but she talked as
naturally as ever, and the spontaneousness of her good-humour was
assuredly not feigned.

On Monday the farmer had business in Maldon. Occasionally when he drove
over to that town he took one or other of his children with him to visit
a relative, and today he proposed that Jane should be of the party. They
started after an early dinner. Michael and Sidney stood together in the
road, watching the vehicle as it rolled away; then they walked in
silence to a familiar spot where they could sit in shadow. Sidney was
glad of Jane's departure for the afternoon. He found it impossible to
escape the restlessness into which he had fallen, and was resolved to
seek relief by opening his mind to the old man. There could be little
doubt that Michael already understood his thoughts, and no better
opportunity for such a conversation was likely to present itself. When
they {}had been seated for a minute or two, neither speaking, Sidney
turned to his companion with a grave look. At the same instant Michael
also had raised his eyes and seemed on the point of saying something of
importance. They regarded each other. The old man's face was set in an
expression of profound feeling, and his lips moved tremulously before
words rose to them.

``What were you going to say, Sidney?'' he asked, reading the other's
features.

``Something which I hope won't be displeasing to you. I was going to
speak of Jane. Since she has been living with you she has grown from a
child to a woman. When I was talking with her in the garden on Saturday
night I felt this change more distinctly than I had ever done before. I
understood that it had made a change in myself. I love her, Mr. Snowdon,
and it's my dearest hope that she may come to feel the same for me.''

Michael was more agitated than the speaker; he raised a hand to his
forehead and closed his eyes as if the light pained them. But the
{}smile with which he speedily answered Sidney's look of trouble was
full of reassurance.

``You couldn't have said anything that would give me more pleasure,'' he
replied, just above his breath. ``Does she know it? Did you speak to
her?''

``We were talking of years ago, and I mentioned Clara Hewett. I said
that I had forgotten all about her except that she'd befriended Jane.
But nothing more than that. I couldn't say what I was feeling just then.
Partly I thought that it was right to speak to' you first; and then---it
seemed to me almost as if I should be treating her unfairly. I'm so much
older,---she knows that it isn't the first time I{{------}}and she's
always thought of me just as a friend.''

``So much older?'' repeated Michael, wath a grave smile. ``Why, you're
both children to my sight. Wait and let me think a bit, Sidney. I too
have something I want to say. I'm glad you've spoken this afternoon,
when there's time for us to talk. Just wait a few minutes, and let me
think.''

{}Sidney had as good as forgotten that there was anything unusual in his
friend's circumstances; this last day or two he had thought of nothing
but Jane and his love for her. Now he recalled the
anticipation---originating he scarcely knew how---that some kind of
disclosure would before long be made to him. The trouble of his mind was
heightened; he waited with all but dread for the next words.

``I think I've told you,'' Michael resumed at length, steadying his
voice, ``that Joseph is my youngest son, and that I had three others.
Three others: Michael, Edward, and Robert,---all dead. Edward died when
he was a boy of fifteen; Robert was killed on the railway---he was a
porter---at three-and-twenty. The eldest went out to Australia; he took
a wife there, and had one child; the wife died when they'd been married
a year or two, and Michael and his boy were drowned, both together. I
was living with them at the time, as you know. But what I've never
spoken of, Sidney, is that my son had made his fortune. He left a deal
of land, and many thousands of {}pounds, behind him. There was no
finding any will; a lawyer in the nearest town, a man that had known him
a long time, said he felt sure there'd been no will made. So, as things
were, the law gave everything to his father.''

He related it with subdued voice, in a solemn and agitated tone. The
effect of the news upon Sidney was a painful constriction of the heart,
a rush of confused thought, an involvement of all his perceptions in a
sense of fear. The pallor of his cheeks and the pained parting of his
lips bore witness to how little he was prepared for such a story.

``I've begun with what ought by rights to have come last,'' pursued
Michael, after drawing a deep sigh, ``But it does me good to get it
told; it's been burdening me this long while. Now you must listen,
Sidney, whilst I show you why I've kept this a secret. I've no fear but
\emph{you}'ll understand me, though most people wouldn't. It's a secret
from everybody except a lawyer in London, who does business for me; a
right-hearted man {}he is, in most things, and I'm glad I met with him,
but he doesn't understand me as you will; he thinks I'm making a
mistake. My son knows nothing about it: at least, it's my hope and
belief he doesn't. He told me he hadn't heard of his brother's death. I
say I hope he doesn't know; it isn't selfishness, that; I needn't tell
you. I've never for a minute thought of myself as a rich man, Sidney;
I've never thought of the money as my own, never; and if Joseph proves
himself honest, I'm ready to give up to him the share of his brother's
property that it seems to me ought to be rightly his, though the law for
some reason looks at it in a different way. I'm ready, but I must know
that he's an honest man; I must prove him first.''

The eagerness of his thought impelled him to repetitions and emphasis.
His voice fell upon a note of feebleness, and with an effort he
recovered the tone in which he had begun.

``As soon as I knew that all this wealth had fallen to me, I decided at
once to come back to England. What could I do out {}there? I decided to
come to England, but I couldn't see further ahead than that. I sold all
the land; I had the business done for me by that lawyer I spoke of, that
had known my son, and he recommended me to a Mr. Percival in London. I
came back, and I found little Jane, and then bit by bit I began to
understand what my duty was. It got clear in my mind; I formed a
purpose, a plan, and it's as strong in me now as ever. Let me think
again for a little, Sidney. I want to make it as plain to you as it is
to me. You'll understand me best if I go back and tell you more than I
have done yet about my life before I left England. Let me think a
while.''

He was overcome with a fear that he might not be able to convey with
sufficient force the design which had wholly possessed him. So painful
was the struggle in him between enthusiasm and a consciousness of
failing faculties, that Sidney grasped his hand and begged him to speak
simply, without effort.

{}``Have no fear about my understanding you. We've talked a great deal
together, and I know veiy well what your strongest motives are. Trust me
to sympathise with you.''

``I do! If I hadn't that trust, Sidney, I couldn't have felt the joy I
did when you spoke to me of my Jane. You'll help me to carry out my
plan; you and Jane will; you and Jane! I've got to be such an old man
all at once, as it seems, and I dursn't have waited much longer without
telling you what I had in my mind. See now, I'll go back to when I was a
boy, as far back as I can remember. You know I was born in Clerkenwell,
and I've told you a little now and then of the hard times I went
through. My poor father and mother came out of the country, thinking to
better themselves; instead of that, they found nothing but cold and
hunger, and toil and moil. They were both dead by when I was between
thirteen and fourteen. They died in the same winter,---a cruel winter. I
used to go about begging bits of firewood from the neighbours. {}There
was a man in our house who kept dogs, and I remember once catching hold
of a bit of dirty meat---I can't call it meat---that one of them had
gnawed and left on the stairs; and I ate it, as if I'd been a dog
myself, I was that driven with hunger. Why, I feel the cold and the
hunger at this minute! It was a cruel winter, that, and it left me
alone. I had to get my own living as best I could.

``No teaching. I was nineteen before I could read the signs over shops,
or write my own name. Between nineteen and twenty I got all the
education I ever was to have, paying a man with what I could save out of
my earnings. The blessing was I had health and strength, and with hard
struggling I got into a regular employment. At five-andtwenty I could
earn my pound a week, pretty certain. When it got to five shillings
more, I must needs have a wife to share it with me. My poor girl came to
live with me in a room in Hill Street.

``I've never spoken to you of her, but you {}shall hear it all now, cost
me what it may in the telling. Of course she was out of a poor home, and
she'd known as well as me what it was to go cold and hungry. I sometimes
think, Sidney, I can see a look of her in Jane's face,---but she was
prettier than Jane; yes, yes, prettier than Jane. And to think a man
could treat a poor little thing like her in the way I did!---you don't
know what sort of a man Michael Snowdon was then; no, you don't know
what I was then. You're not to think I ill-used her in the common way; I
never raised my hand, thank God! and I never spoke a word a man should
be ashamed of. But I was a hard, self-willed, stubborn fool! How she
came to like me and to marry me I don't know; we were so different in
every way. Well, it was partly my nature and partly what I'd gone
through; we hadn't been married more than a month or two when I began to
find fault with her, and from that day on she could never please me. I
earned five-and-twenty shillings a week, and I'd made up my mind that we
must save out of it. {}I wouldn't let \emph{her} work; no, what
\emph{she} had to do was to keep the home on as little as possible, and
always have everything clean and straight when I got back at night. But
Jenny hadn't the same ideas about things as I had. She couldn't pinch
and pare, and our plans of saving came to nothing. It grew worse as the
children were born. The more need there was for carefulness, the more
heedless Jenny seemed to get. And it was my fault, mine from beginning
to end. Another man would have been gentle with her and showed her
kindly when she was wrong, and have been thankful for the love she gave
him, whatever her faults. That wasn't my way. I got angry, and made her
life a burden to her. I must have things done exactly as I wished; if
not, there was no end to my fault-finding. And yet, if you'll believe
it, I loved my wife as truly as man ever did. Jenny couldn't understand
that,---and how should she? At last she began to deceive me in all sorts
of little things; she got into debt with shop-people, she showed {}me
false accounts, she pawned things without my knowing. Last of all, she
began to drink. Our fourth child was born just at that time; Jenny had a
bad illness, and I believe it set her mind wrong. I lost all control of
her, and she used to say if it wasn't for the children she'd go and
leave me. One morning we quarrelled very badly, and I did as I'd
threatened to,---I walked about the streets all the night that followed,
never coming home. I went to work next day, but at dinner-time I got
frightened and ran home just to speak a word. Little Mike, the eldest,
was playing on the stairs, and he said his mother was asleep. I went
into the room, and saw Jenny lying on the bed dressed. There was
something queer in the way her arms were stretched out. When I got near
I saw she was dead. She'd taken poison.

``And it was I had killed her, just as much as if I'd put the poison to
her lips. All because I thought myself such a wise fellow, because I'd
resolved to live more prudently {}than other men of my kind did. I
wanted to save money for the future--- out of five-andtwenty shilhngs a
week. Many and many a day I starved myself to try and make up for
expenses of the home. Sidney, you remember that man we once went to hear
lecture, the man that talked of nothing but the thriftlessness of the
poor, and how it was their own fault they suffered? I was very near
telling you my story when we came away that night. Why, look; I myself
was just the kind of poor man that would have suited that lecturer. And
what came of it? If I'd let my poor Jenny go her own way from the first,
we should have had hard times now and then, but there'd have been our
love to help us, and we should have been happy enough. They talk about
thriftiness, and it just means that poor people are expected to practise
a self-denial that the rich can't even imagine, much less carry out! You
know now why this kind of talk always angers me.''

Michael brooded for a few moments, his eyes straying sadly over the
landscape before him.

{}``I was punished,'' he continued, ``and in the fittest way. The two of
my boys who showed most love for me, Edward and Robert, died young. The
eldest and youngest were a constant trouble to me. Michael was
quick-tempered and self-willed, like myself; I took the wrong way with
him, just like I had with his mother, and there was no peace till he
left home. Joseph was still harder to deal with; but he's the only one
left alive, and there is no need to bring up things against him. With
him I wasn't to blame, unless I treated him too kindly and spoilt him.
He was my favourite, was Jo, and he repaid me cruelly. When he married,
I only heard of it from other people; we'd been parted for a long time
already. And just about then I had a letter from Michael, asking me if I
was willing to go out and live with him in Australia. I hadn't heard
from him more than two or three times in twelve years, and when this
letter came to me I was living in Sheffield; I'd been there about five
years. He wrote to say he was {}doing well, and that he didn't like to
think of me being left to spend my old age alone. It was a kind letter,
and it warmed my heart. . Lonely I was; as lonely and sorrowful a man as
any in England. I wrote back to say that I'd come to him gladly if he
could promise to put me in the way of earning my own living. He agreed
to that, and I left the old country, little thinking I should ever see
it again. I didn't see Joseph before I went. All I knew of him was, that
he lived in Clerkenwell Close, married; and that was all I had to guide
me when I tried to find him a few years after. I was bitter against him,
and went without trying to say good-bye.

``My son's fortune seems to have been made chiefly out of horse-dealing
and what they call `land-grabbing,'---buying sheep-runs over the heads
of squatters, to be bought out again at a high profit. Well, you know
what my opinion is of trading at the best, and as far as I could
understand it, it was trading at about its worst that had filled
Michael's pockets. He'd had a partner for {}a time, and very ugly stones
were told me about the man. However, Michael gave me as kind a welcome
as his letter promised; prosperity had done him good, and he seemed only
anxious to make up for the years of unkindness that had gone by. Had I
been willing, I might have lived under his roof at my ease; but I held
him to his bargain, and worked like any other man who goes there without
money. It's a comfort to me to think of those few years spent in quiet
and goodwill with my eldest boy. His own lad would have given trouble,
I'm afraid, if he'd lived; Michael used to talk to me uneasily about
him, poor fellow! But they both came to their end before the world had
parted them.

``If I'd been a young man, I' dare say I should have felt different when
they told me how rich I was; it gave me no pleasure at first, and when
I'd had time to think about it I only grew worried. I even thought once
or twice of getting rid of the burden by giving all the money to a
hospital in Sydney or {}Melbourne. But then I remembered that the poor
in the old country had more claim on me, and when I'd got used to the
idea of being a wealthy man, I found myself recalling all sorts of
fancies and wishes that used to come into my head when I was working
hard for a poor living. It took some time to get all the lawyer's
business finished, and by when it was done I began to see a way before
me. First of all I must find my son in England, and see if he needed
help. I hadn't made any change in my way of living, and I came back from
Australia as a steerage passenger, wearing the same clothes that I'd
worked in. The lawyer laughed at me, but I'm sure I should have laughed
at myself if I'd dressed up as a gentleman and begun to play the fool in
my old age. The money wasn't to be used in that way. I'd got my ideas,
and they grew clearer during the voyage home.

``You know how I found Jane. Not long after, I put an advertisement in
the papers, asking my son, if he saw it, to communicate with Mr.
Percival,---that's the lawyer I was {}recommended to in London. There
was no answer; Joseph was in America at that time. I hadn't much reason
to like Mrs. Peckover and her daughter, but I kept up acquaintance with
them because I thought they might hear of Jo some day. And after a while
I sent Jane to learn a business. Do you know why I did that? Can you
think why I brought up the child as if I'd only had just enough to keep
us both, and never gave a sign that I could have made a rich lady of
her?''

In asking the question, he bent forward and laid his hand on Sidney's
shoulder. His eyes gleamed with that light which betrays the enthusiast,
the idealist. As he approached the explanation to which his story had
tended, the signs of age and weakness disappeared before the intensity
of his feeling. Sidney understood now why he had always been conscious
of something in the man's mind that was not revealed to him, of a
life-controlling purpose but vaguely indicated by the general tenor of
Michael's opinions. The latter's fervour affected him, and he replied
with emotion:

{}``You wish Jane to think of this money as you do yourself,---not to
regard it as wealth, but as the means of bringing help to the
miserable.''

``That is my thought, Sidney. It came to me in that form whilst I was
sitting by her bed, when she was ill at Mrs. Peckover's. I knew nothing
of her character then, and the idea I had might have come to nothing
through her turning out untrustworthy. But I thought to myself: Suppose
she grows up to be a good woman,---suppose I can teach her to look at
things in the same way as I do myself, train her to feel that no
happiness could be greater than the power to put an end to ever so
little of the want and wretchedness about her,---suppose when I die I
could have the certainty that all this money was going to be used for
the good of the poor by a woman who herself belonged to the poor? You
understand me? It would have been easy enough to leave it among
charities in the ordinary way; but my idea went beyond that. I might
have had Jane schooled and fashioned into a lady, and still have hoped
that she {}would use the money well; but my idea went beyond
\emph{that}. There's plenty of ladies now-a-days taking an interest in
the miserable, and spending their means unselfishly. What I hoped was to
raise up for the poor and the untaught a friend out of their own midst,
some one who had gone through all that \emph{they} suffer, who was
accustomed to earn her own living by the work of her hands as
\emph{they} do, who had never thought herself their better, who saw the
world as they see it and knew all their wants. A lady may do good, we
know that; but she can't be the friend of the poor as I understand it;
there's too great a distance between her world and theirs. Can you
picture to yourself how anxiously I've watched this child from the first
day she came to live with me? I've scarcely had a thought but about her.
I saw very soon that she had good feelings, and I set myself to
encourage them. I wanted her to be able to read and write, but there was
no need of any more education than that; it was the heart I cared about,
not the mind. Besides, I had {}always to keep saying to myself that
perhaps, after all, she wouldn't turn out the kind of woman I wished,
and in that case she mustn't be spoiled for an ordinary life. Sidney,
it's this money that has made me a weak old man when I might still have
been as strong as many at fifty; the care of it has worn me out; I
haven't slept quietly since it came into my hands. But the worst is
over. I shan't be disappointed. Jane will be the woman I've hoped for,
and however soon my own life comes to an end, I shall die knowing that
there's a true man by her side to help her to make my idea a reality.

``I've mentioned Mr. Percival, the lawyer, lie's an old man like myself,
and we've had many a long talk together. About a year and a half ago I
told him what I've told you now. Since I came back to England, he's been
managing the money for me; he's paid me the little we needed, and the
rest of the income has been used in charity by some people we could
trust. Well, Mr. Percival doesn't go with me in my plans for {}Jane. He
thinks I'm making a mistake, that I ought to have had the child educated
to fit her to live with rich people. It's no use; I can't get him to
feel what a grand thing it'll be for Jane to go about among her own
people and help them as nobody ever could. He said to me not long ago,
`And isn't the girl ever to have a husband?' It's my hope that she will,
I told him. `And do you suppose,' he went on, `that whoever marries her
will let her live in the way you talk of? Where are you going to find a
working man that'll be content never to touch this money,---to work on
for his weekly wages, when he might be living at his ease?' And I told
him that it wasn't as impossible as he thought. What do you think,
Sidney?''

The communication of a noble idea has the same effect upon the brains of
certain men---of one, let us say, in every hundred thousand---as a wine
that exalts and enraptures. As Sidney listened to the old man telling of
his wondrous vision, he became possessed with ardour such as he had
known {}but once or twice in his life. Idealism such as Michael Snowdon
had developed in these latter years is a form of genius; given the
susceptible hearer, it dazzles, inspires, raises to heroic contempt of
the facts of life. Had this story been related to him of some unknown
person, Sidney would have admired, but as one admires the nobly
impracticable; subject to the electric influence of a man who was great
enough to conceive and direct his life by such a project, who could
repose so supreme a faith in those he loved, all the primitive nobleness
of his character asserted itself, and he could accept with a throbbing
heart the superb challenge addressed to him.

``If Jane can think me worthy to be her husband,'' he replied, ``your
friend shall see that he has feared without cause.''

``I knew it, Sidney; I knew it!'' exclaimed the old man. ``How much
younger I feel, now that I have shared this burden with you!''

``And shall you now tellJane?'' the other inquired.

{}``Not yet; not just yet. She is very young; we must wait a little. But
there can be no reason why you shouldn't speak to her---of yourself.''

Sidney was descending from the clouds. As the flush of his humanitarian
enthusiasm passed away, and he thought of his personal relations to
Jane, a misgiving, a scruple began to make itself heard within him.
Worldly and commonplace the thought, but --- had he a right to ask the
girl to pledge herself to him under circumstances such as these? To be
sure, it was not as if Jane were an heiress in the ordinaiy way; for all
that, would it not be a proceeding of doubtful justice to woo her when
as yet she was wholly ignorant of the most important item in her
situation? His sincerity was unassailable, but---suppose, in fact, he
had to judge the conduct of another man thus placed? Upon the heated
pulsing of his blood succeeded a coolness, almost a chill; he felt as
though he had been on the verge of a precipice, and had been warned to
draw back {}only just in time. Every second showed him more distinctly
what his duty was. He experienced a sensation of thankfulness that he
had not spoken definitely on Saturday evening. His instinct had guided
him aright; Jane was still too young to be called upon solemnly to
decide her whole future.

``That, too, had better wait, Mr. Snowdon,'' he said, after a pause of a
minute. ``I should like her to know everything before I speak to her in
that way. In a year it will be time enough.''

Michael regarded him thoughtfully.

``Perhaps you are right. I wish you knew Mr. Percival; but there is
time, there is time. He still thinks I shall be persuaded to alter my
plans. That night you came to Hanover Street and found me away, he took
me to see a lady who works among the poor in Clerkenwell; she knew me by
name, because Mr. Percival had given her money from me to use, but we'd
never seen each other till then. He wants me to ask her opinion about
Jane.''

{}``Has he spoken of her to the lady, do you thmk?''

``Oh no!'' replied the other, with perfect confidence. ``He has promised
me to keep all that a secret as long as I wish. The lady---her name is
Miss Lant---seemed all that my friend said she was, and perhaps Jane
might do well to make her acquaintance some day; but that mustn't be
till Jane knows and approves the purpose of my life and hers. The one
thing that troubles me still, Sidney, is---her father. It's hard that I
can't be sure whether my son will be a help or a hindrance. I must wait,
and try to know him better.''

The conversation had so wearied Michael, that in returning to the house
he had to lean on his companion's arm. Sidney was silent, and yielded,
he scarce knew why, to a mood of depression. When Jane returned from
Maldon in the evening, and he heard her happy voice as the children ran
out to welcome her, there was a heaviness at his heart. Perhaps it came
only of hope deferred.
