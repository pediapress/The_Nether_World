\chapter{Sidney's Struggle}

\textsc{In} the dreary days when autumn is being choked by the first
fogs, Sidney Kirkwood had to bestir himself and to find new lodgings.
The cheerless task came upon him just when he had already more than
sufficient trouble, and to tear himself out of the abode in which he had
spent eight years caused him more than regret; he felt superstitiously
about it, and questioned fate as to what sorrows might be lurking for
him behind this corner in life's journey. Move he must; his landlady was
dead, and the house would perhaps be vacant for a long time. After
making search about Islington one rainy evening, he found himself at the
end of Hanover Street, and was drawn to the familiar house; not,
however, to visit the {}Snowdons, but to redeem a promise recently made
to Bessie Byass, who declared herself vastly indignant at the neglect
with which he treated her. So, instead of going up the steps to the
front door, he descended into the area. Bessie herself opened to him,
and after a shrewd glance, made as though she would close the door
again. ``Nothing for you! The idea of beggars coming down the
area-steps! Be off!''

``I'm worse than a beggar,'' replied Sidney. ``Housebreaking's more in
my line.''

And he attempted to force an entrance. Bessie struggled, but had to give
in, overcome with laughter. Samuel was enjoying a pipe in the front
kitchen; in spite of the dignity of keeping a servant (to whom the
back-kitchen was sacred), Mr. and Mrs. Byass frequently spent their
evenings below stairs in the same manner as of old.

The talk began with Sidney's immediate difficulties.

``Now if it had only happened half a year ago,'' said Bessie, ``I should
have got you into our first-floor rooms.''

{}``Shouldn't wonder if we have him there yet, some day,'' remarked Sam,
winking at his wife.

``Not him,'' was Bessie's rejoinder, with a meaning smile. ``He's a cool
hand, is Mr. Kirkwood. He knows how to wait. When \emph{something}
happens, we shall have him taking a house out at Highbury, you see if he
don't.''

Sidney turned upon her with anything but a jesting look.

``What do you mean by that, Mrs. Byass?'' he asked, sharply. ``When
\emph{what} happens? What are you hinting at?''

``Bless us and save us! '' cried Bessie. ``Here, Sam, he's going to
swallow me. What harm have I done? ''

``Please tell me what you meant?'' Sidney urged, his face expressing
strong annoyance. ``Why do you call me a `cool hand,' and say that `I
know how to wait'? What did you mean? I'm serious; I want you to
explain.''

Whilst he was speaking there came a knock at the kitchen-door. Bessie
cried, ``Come in,'' and Jane showed herself; she {}glanced in a startled
way at Sidney, murmured a ``good-evening'' to him, and made a request of
Bessie for some trifle she needed. Sidney, after just looking round,
kept his seat and paid no further attention to Jane, who speedily
retired.

Silence followed, and in the midst of it Kirkwood pushed his chair
impatiently.

``Bess,'' cried Samuel, with an affected jocoseness, ``you're called
upon to apologise. Don't make a fool of yourself again.''

``I don't see why he need be so snappish with me,'' replied his wife.
``I beg his pardon, if he wants me.''

But Sidney was laughing now, though not in a very natural way. He put an
end to the incident, and led off into talk of quite a different kind.
When supper-time was at hand he declared that it was impossible for him
to stay. The hour had been anything but a lively one, and when he was
gone his friends discussed at length this novel display of ill-humour on
Sidney's part.

He went home muttering to himself, and {}passed as bad a night as he had
ever known. Two days later his removal to new lodgings was effected;
notwithstanding his desire to get into a cleaner region, he had taken a
room at the top of a house in Red Lion Street, in the densest part of
Clerkenwell, where his neighbours under the same roof were craftsmen,
carrying on their business at home.

``It'll do well enough just for a time,'' he said to himself. ``Who can
say when I shall be really settled again, or whether I ever shall?''

Midway in an attempt to put his things in order, to nail his pictures on
the walls and bring forth his books again, he was seized with such utter
discouragement that he let a volume drop from his hand and threw himself
into a seat. A moan escaped his lips,---``That cursed money!''

Ever since the disclosure made to him by Michael Snowdon at Danbury he
had been sensible of a grave uneasiness respecting his relations with
Jane. At the moment he might imagine himself to share the old man's
{}enthusiasm, or dream, or craze,---whichever name were the most
appropriate,---but not an hour had passed before he began to lament that
such a romance as this should envelop the life which had so linked
itself with his own. Immediately there arose in him a struggle between
the idealist tendency, of which he had his share, and stubborn everyday
sense, supported by his knowledge of the world and of his own being, ---
a struggle to continue for months, thwarting the natural current of his
life, racking his intellect, embittering his heart's truest emotions.
Conscious of mystery in Snowdon's affairs, he had never dreamed of such
a solution as this; the probability was---so he had thought---that
Michael received an annuity under the will of his son who died in
Australia. No word of the old man's had ever hinted at wealth in his
possession; the complaints he frequently made of the ill use to which
wealthy people put their means seemed to imply a regret that he, with
his purer purposes, had no power of doing anything. There was no
explaining the manner of Jane's {}bringing-up, if it were not necessary
that she should be able to support herself; the idea on which Michael
acted was not such as would suggest itself, even to Sidney's mind.
Deliberately to withhold education from a girl who was to inherit any
property worth speaking of would \_be acting with such boldness of
originality that Sidney could not seriously have attributed it to his
friend. In fact, he did not know Michael until the revelation was made;
the depths of the man's character escaped him.

The struggle went all against idealism. It was a noble vision, that of
Michael's, but too certainly Jane Snowdon was not the person to make it
a reality; the fearful danger was, that all the possibilities of her
life might be sacrificed to a vain conscientiousness. Her character was
full of purity and sweetness and self-forgetful warmth, but it had not
the strength necessary for the carrying out of a purpose beset with
difficulties and perils. Michael, it was true, appeared to be aware of
this; it did not, however, gravely disturb {}him, and for the simple
reason that not to Jane alone did he look for the completion of his
design; destiny had brought him aid such as he could never have
anticipated; Jane's helpmate was at hand, in whom his trust was
unbounded.

What was in his way, that Sidney should not accept the responsibility?
Conscience from the first whispered against his doing so, and the
whisper was grown to so loud a voice that not an adverse argument could
get effective hearing. Temptations lurked for him and sprang out in
moments of his weakness, but as temptations they were at once
recognised. ``He had gone too far to retire; he would be guilty of sheer
treachery to Jane; he would break the old man's heart.'' All which meant
merely that he loved the girl, and that it would be like death to part
from her. But why part? What had conscience got hold of, that it made
all this clamour? Oh, it was simple enough; Sidney not only had no faith
in the practicability of such a life's work as Michael visioned, but he
had the profoundest distrust {}of his own moral strength if he should
allow himself to be committed to lifelong renunciation. ``I am no
hero,'' he said, ``no enthusiast. The time when my whole being could be
stirred by social questions has gone by. I am a man in love, and in
proportion as my love has strengthened, so has my old artist-self
revived in me, until now I can imagine no bliss so perfect as to marry
Jane Snowdon and go off to live with her amid fields and trees, where no
echo of the suffering world should ever reach us.'' To confess this was
to make it terribly certain that sooner or later the burden of
conscientiousness would become intolerable. Not from Jane would support
come in that event; she, poor child! would fall into miserable
perplexity, in conflict between love and duty, and her life would be
ruined.

Of course a man might have said, ``What matter how things arrange
themselves when Michael is past knowledge of them? I will marry the
woman I honestly desire, and together we will carry out this
humanitarian {}project so long as it be possible. When it ceases to be
so, well''{{------}}. But Sidney could not take that view. It shamed him
beyond endurance to think that he must ever avoid Jane's look, because
he had proved himself dishonest, and, what were worse, had tempted her
to become so.

The conflict between desire and scruple made every day a weariness.
Instead of looking forward eagerly to the evening in the week which he
spent with Michael and Jane, he dreaded its approach. Scarcely had he
met Jane's look since this trouble began; he knew that her voice when
she spoke to him expressed consciousness of something new in their
relations, and even whilst continuing to act his part he suffered
ceaselessly. Had Michael ever repeated to his granddaughter the
confession which Sidney would now have given anything to recall? It was
more than possible. Of Jane's feeling Sidney could not entertain a
serious doubt, and he knew that for a long time he had done his best to
encourage it. It was unpardonable to draw aloof {}from her just because
these circumstances had declared themselves, circumstances which brought
perplexity into her life and doubtless made her long for another kind of
support than Michael could afford her. The old man himself appeared to
be waiting anxiously; he had fallen back into his habit of long
silences, and often regarded Sidney in a way which the latter only too
well understood.

He tried to help himself through the time of indecision by saying that
there was no hurry. Jane was very young, and with the new order of
things her life had in truth only just begun. She must have a space to
look about her; all the better if she could form various acquaintances.
On that account he urged so strongly that she should be brought into
relation with Miss Lant, and, if possible, with certain of Miss Lant's
friends. All very well, had not the reasoning been utterly insincere. It
might have applied to another person; in Jane's case it was mere
sophistry. Her nature was home-keeping; to force her into alliance with
conscious {}philanthropists was to set her in the falsest position
conceivable; striving to mould herself to the desires of those she
loved, she would suffer patiently and in secret mourn for the time when
she had been obscure and happy. These things Sidney knew with a
certainty only less than that wherewith he judged his own sensations;
between Jane and himself the sympathy was perfect. And in despite of
scruple he would before long have obeyed the natural impulse of his
heart, had it not been that still graver complications declared
themselves, and by exasperating his over-sensitive pride made him
reckless of the pain he gave to others so long as his own self-torture
was made sufficiently acute.

With Joseph Snowdon he was doing his best to be on genial terms, but the
task was a hard one. The more he saw of Joseph, the less he liked him.
Of late the filter manufacturer had begun to strike notes in his
conversation which jarred on Sidney's sensibilities and made him
disagreeably suspicious that something more was meant than Joseph
{}cared to put into plain speech. Since his establishment in business
Joseph had become remarkably attentive to his father; he appeared to
enter with much zeal into all that concerned Jane; he conversed
privately with the old man for a couple of hours at a time, and these
dialogues, for some reason or other, he made a point of reporting to
Sidney. According to these reports---and Sidney did not wholly discredit
them---Michael was coming to have a far better opinion of his son than
formerly, was even disposed to speak with him gravely of his dearest
interests.

``We talked no end about you, Sidney, last night,'' said Joseph on one
occasion, with the smile whereby he meant to express the last degree of
friendly intelligence.

And Sidney, though anxiously desiring to know the gist of the
conversation, in this instance was not gratified. He could not bring
himself to put questions, and went away in a mood of vague annoyance
which Joseph had the especial power of exciting.

With the Byasses, Joseph was forming an {}intimacy; of this too Sidney
became aware, and it irritated him. The exact source of this irritation
he did not at first recognise, but it was disclosed at length
unmistakably enough, and that on the occasion of the visit recently
described. Bessie's pleasantry, which roused him in so unwonted a
manner, could bear, of course, but one meaning; as soon as he heard it,
Sidney saw as in a flash that one remaining aspect of his position which
had not as yet attracted his concern. The Byasses had learnt, or had
been put in the way of surmising, that Michael Snowdon was wealthy;
instantly they passed to the reflection that in marrying Jane their old
acquaintance would be doing an excellent stroke of business. They were
coarse-minded, and Bessie could even venture to jest with him on this
detestable view of his projects. But was it not very likely that they
derived their information from Joseph Snowdon? And if so, was it not all
but certain that Joseph had suggested to them this way of regarding
Sidney himself?

{}So when Jane's face appeared at the door he held himself in stubborn
disregard of her. A thing impossible to him, he would have said a few
minutes ago. He revenged himself upon Jane. Good; in this way he was
likely to make noble advances.

The next evening he was due at the Snowdons', and for the very first
time he voluntarily kept away. He posted a note to say that the business
of his removal had made him irregular; he would come next week, when
things were settled once more.

Thus it came to pass that he sat wretchedly in his unfamiliar room and
groaned about ``that accursed money.'' His only relief was in bursts of
anger. Why had he not the courage to go to Michael and say plainly what
he thought? ``You have formed a wild scheme, the project of a fanatic.
Its realisation would be a miracle, and in your heart you must know that
Jane's character contains no miraculous possibilities. You are playing
with people's lives, as fanatics always do. For Heaven's sake, bestow
your money on the {}practical folks who make a solid business of
relieving distress! Jane, I know, will bless you for making her as poor
as ever. Things are going on about you which you do not suspect. Your
son is plotting, plotting; I can see it. This money will be the cause of
endless suffering to those you really love, and will never be of as much
benefit to the unknown as if practical people dealt with it. Jane is a
simple girl, of infinite goodness; what possesses you that you want to
make her an impossible sort of social saint?'' Too hard to speak thus
frankly. Michael had no longer the mental pliancy of even six months
ago; his \emph{idea} was everything to him; as he became weaker, it
would gain the dire force of an hallucination. And in the meantime he,
Sidney, must submit to be slandered by that fellow who had his own ends
to gain.

To marry Jane, and, at the old man's death, resign every farthing of the
money to her trustees, for charitable uses?---But the old pang of
conscience; the life-long wound to Jane's tender heart.

{}A day of headache and incapacity, during which it was all he could do
to attend to his mechanical work, and again the miserable loneliness of
his attic. It rained, it rained. He had half a mind to seek refuge at
some theatre, but the energy to walk so far was lacking. And whilst he
stood stupidly abstracted there came a knock at his door.

``I thought I'd just see if you'd got straight,'' said Joseph Snowdon,
entering with his genial smile.

Sidney made no reply, but turned as if to stir the fire. Hands in
pockets, Joseph sauntered to a seat.

``Think you'll be comfortable here?'' he went on. ``Well, well; of
course it's only temporary.''

``I don't know about that,'' returned Sidney.

``I may stay here as long as I was at the last place,---eight years.''

Joseph laughed, with exceeding good-nature.

``Oh yes; I shouldn't wonder,'' he said, entering into the joke.
``Still''---becoming {}serious---``I wish you'd found a pleasanter
place. With the winter coming on, you see''{{------}}

Sidney broke in with splenetic perversity.

``I don't know that I shall pass the winter here. My arrangements are
all temporary---all of them.''

After glancing at him the other crossed his legs and seemed to dispose
himself for a stay of some duration.

``You didn't turn up the other night---in Hanover Street.''

``No.''

``I was there. We talked about you. My father has a notion you haven't
been quite well lately. I dare say you're worrying a little, eh?''

Sidney remained standing by the fireplace, turned so that his face was
in shadow.

``Worry? Oh, I don't know,'' he replied, idly.

``Well, Pm worried a good deal, Sidney, and that's the fact.''

``What about?''

{}``All sorts of things. I've meant to have a long talk with you; but
then I don't quite know how to begin. Well, see, it's chiefly about
Jane.''

Sidney neither moved nor spoke.

``After all, Sidney,'' resumed the other, softening his voice, " I
\emph{am} her father, you see. A precious bad one I've been, that
there's no denying, and dash it if I don't sometimes feel ashamed of
myself. I do when she speaks to me in that pleasant way she has,---you
know what I mean. For all that, I am her father, and I think it's only
right I should do my best to make her happy. You agree with that, I
know.''

``Certainly I do.''

``You won't take it ill if I ask whether---in fact, whether you've ever
asked her---you know what I mean.''

``I have not,'' Sidney replied, in a clear, unmoved tone, changing his
position at the same time so as to look his interlocutor in the face.

Joseph seemed relieved.

{}``Still,'' he continued, ``you've given her to understand---eh? I
suppose there's no secret about that?''

``I've often spoken to her very intimately, but I have used no words
such as you are thinking of. It's quite true that my way of behaving has
meant more than ordinary friendship.''

``Yes, yes; you're not offended at me bringing this subject up, old man?
You see, I'm her father, after all, and I think we ought to understand
each other.''

``You are quite right.''

``Well, now, see.'' He fidgeted a little.

``Has my father ever told you that his friend the lawyer, Percival,
altogether went against that way of bringing up Jane?''

``Yes, I know that.''

``You do?'' Joseph paused before proceeding. ``To tell you the truth, I
don't much care about Percival. I had a talk with him, you know, when my
business was being settled. No, I don't quite take to him, so to say.
Now, you won't be offended? The fact {}of the matter is, he asked some
rather queer questions about you,---or, at all events, if they weren't
exactly questions, they---they came to the same thing.''

Sidney was beginning to glare under his brows. Common-sense told him how
very unlikely it was that a respectable solicitor should compromise
himself in talk with a stranger, and that such a man as J. J. Snowdon;
yet, whether the story were true or not, it meant that Joseph was
plotting in some vile way, and thus confirmed his suspicions. He
inquired, briefly and indifferently, what Mr. Percival's insinuations
had been.

``Well, I told you I don't much care for the fellow. He didn't say as
much, mind, but he seemed to be hinting-like that, as Jane's father, I
should do well to---to keep an eye on you---ha, ha! It came to that, I
thought,---though, of course, I may have been mistaken. It shows how
little he knows about you and father. I fancy he'd got it into his head
that it was \emph{you} set father on those plans about Jane,---though
\emph{why} I'd like to know.''

{}He paused. Sidney kept his eyes down, and said nothing.

``Well, there's quite enough of that; too much. Still, I thought I'd
tell you, you see. It's well to know when we've got enemies behind our
backs. But see, Sidney; to speak seriously, between ourselves.'' He
leaned forward in the confidential attitude. ``You say you've gone just
a bit further than friendship with our Janey. Well, I don't know a
better man, and that's the truth,---but don't you think we might put
this off for a year or two? Look now, here's this lady. Miss Lant,
taking up the girl, and it's an advantage to her; you won't deny that. I
sympathise with my good old dad; I do, honestly; but I can't help
thinking that Janey, in her position, ought to see a little of the
world. There's no secrets between us; you know what she'll have as well
as I do. I should be a brute if I grudged it her, after all she's
suffered from my neglect. But don't you think we might leave her free
for a year or two?''

``Yes, I agree with you.''

{}``You do? I thought you and I could understand each other, if we only
got really talking. Look here, Sidney; I don't mind just whispering to
you. For anything I know, Percival is saying disagreeable things to the
old man; but don't you worry about that. It don't matter a scrap, you
see, so long as you and I keep friendly, eh? I'm talking very open to
you, but it's all for Janey's sake. If you went and told father I'd been
saying anything against Percival---well, it would make things nasty for
me. I've put myself in your hands, but I know the kind of man you are.
It's only right you should hear of what's said. Don't worry; we'll just
wait a little, that's all. I mean it all for the little girl's sake. It
wouldn't be nice if you married her and then she was told---eh?''

Sidney looked at the speaker steadily, then stirred the fire and moved
about for a few moments. As he kept absolute silence, Joseph, after
throwing out a few vague assurances of good- will and trust, rose to
take his leave. Kirkwood shook hands with him, but spoke {}not a word.
Late the same night Sidney penned a letter to Michael Snowdon. In the
morning he read it over, and instead of putting it into an envelope,
locked it away in one of his drawers.

When the evening for his visit to Hanover Street again came round he
again absented himself, this time just calling to leave word with the
servant that business kept him away. The business was that of walking
aimlessly about Clerkenwell, in mud and fog. About ten o'clock he came
to Farringdon Road Buildings, and with a glance up towards the Hewetts'
window he was passing by when a hand clutched at him. Turning, he saw
the face of John Hewett, painfully disturbed, strained in some wild
emotion.

``Sidney! Come this way; I want to speak to you.''

``Why, what's wrong?''

``Come over here. Sidney,---I've found my girl,---I've found Clara!''
