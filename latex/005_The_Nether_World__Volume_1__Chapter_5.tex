\chapter {Jane is Visited}

At ten o'clock next morning Mrs. Peckover reached home. She was a tall,
big-boned woman of fifty, with an arm like a coal-heaver's. She had dark
hair, which shone and was odorous with unguents; a sallow, uncomely
face, and a handsome moustache. Her countenance was more difficult to
read than Clem's; a coarse, and most likely brutal, nature was plain
enough in its lines, but there was also a suggestion of self-restraint,
of sagacity, at all events of cunning, qualities which were decidedly
not inherited by her daughter. With her came the relative whose presence
had been desired at the funeral to-day. This was Mrs. Gully, a stout
person with a very red nose and bleared eyes. The
{\protect\hypertarget{93}{}{}}credit of the family demanded that as many
relatives as possible should follow the hearse, and Mrs. Peckover's
reason for conducting Mrs. Gully hither was a justifiable fear lest, if
she came alone, the latter would arrive in too manifest a state of
insobriety. A certain amount of stimulant had been permitted on the way,
just enough to assist a genteel loquacity, for which Mrs. Gully had a
reputation. She had given her word to abstain from further imbibing
until after the funeral.

The news which greeted her arrival was anything but welcome to Mrs.
Peckover. In the first place, there would be far more work than usual to
be performed in the house today, and Jane could be ill spared. Worse
than that, however. Clara Hewett, who was losing half a day's work on
Jane's account, made a very emphatic statement as to the origin of the
illness, and said that if anything happened to Jane, there would be
disagreeable facts forthcoming at a coroner's inquest. Having looked at
the sick child, Mrs. Peckover went downstairs and shut
{\protect\hypertarget{94}{}{}}herself up with Clem. There was a stormy
interview.

``So you thought you'd have yer fling, did you, just because I wasn't
'ere? You must go makin trouble, just to suit yer own fancies! I'll pay
you, my lady! Gr-r-r!''

Whereupon followed the smack of a large hand on a fleshy cheek, so
vigorous and unexpected a blow that even the sturdy Clem staggered back.

``You leave me alone, will you?'' she roared out, her smitten cheek in a
flame. ``Do that again, an' I'll give you somethin' for yerself! See if
I don't! You just try it on!''

The room rang with uproarious abuse, with disgusting language, with the
terrific threats which are such common flowers of rhetoric in that
world, and generally mean nothing whatever. The end of it all was that
Clem went to fetch a doctor, one in whom Mrs. Peckover could repose
confidence. The man was, in fact, a druggist, with a shop in an obscure
street over towards St. Luke's; in {\protect\hypertarget{95}{}{}}his
window was exhibited a card which stated that a certain medical man
could be consulted here daily. The said medical man had, in fact, so
much more business than he could attend to---his name appearing in many
shops---that the druggist was deputed to act as his assistant, and was
considerately supplied with death- certificates, already signed, and
only needing to be filled in with details. Summoned by Mrs. Peckover,
whose old acquaintance he was, the druggist left the shop in care of his
son, aged fifteen, and sped to Clerkenwell Close. He made light of
Jane's ailment. ``A little fever, that was all---soon pull her round.
Any wounds, by the bye? No? Oh, soon pull her round. Send for
medicines.''

``We'll have her down in the back-kitchen as soon as the corffin's
away,'' said Mrs. Peckover to Mrs. Hewett. "Don't you upset yerself
about it, my dear; you've got quite enough to think about, Yer. 'usband
got anythink yet? Dear, dear! Don't you put yerself out. I'm sure it was
a great {\protect\hypertarget{96}{}{}}kindness of you to let the
troublesome thing lay 'ere all night."

Funeral guests were beginning to assemble. On arriving, they were
conducted first of all into the front-room on the ground-floor, the
Peckovers' parlour. It was richly furnished. In the centre stood a round
table, which left small space for moving about, and was at present
covered with refreshments. A polished sideboard supported a row of
dessert plates propped on their edges, and a number of glass vessels,
probably meant for ornament alone, as they could not possibly have been
put to any use. A low cupboard in a recess was surmounted by a frosted
cardboard model of St. Paul's under a glass case, behind which was
reared an oval tray painted with flowers. Over the mantelpiece was the
regulation mirror, its gilt frame enveloped in coarse yellow gauze; the
mantelpiece itself bore a ``wealth'' of embellishments in glass and
crockery. On each side of it hung a framed silhouette, portraits of
ancestors. Other pictures there were many, the most
{\protect\hypertarget{97}{}{}}impressive being an ancient oil-painting,
of which the canvas bulged forth from the frame; the subject appeared to
be a ship, but was just as likely a view of the Alps. Several German
prints conveyed instruction as well as delight; one represented the
trial of Strafford in Westminster Hall; another, the trial of William
Lord Russell at the Old Bailey. There was also a group of engraved
portraits, the Royal Family of England early in the reign of Queen
Victoria; and finally, ``The Destruction of Nineveh,'' by John Martin.
Along the window-sill were disposed flower-plots containing artificial
plants; one or other was always being knocked down by the curtains or
blinds.

Each guest having taken a quaff of ale or spirits or what was called
wine, with perhaps a mouthful of more solid sustenance, was then led
down into the back-kitchen to view the coffin and the corpse. I mention
the coffin first, because in every one's view this was the main point of
interest. Could Mrs. Peckover have buried the old woman
{\protect\hypertarget{98}{}{}}in an orange-crate, she would gladly have
done so for the saving of expense; but with relatives and neighbours to
consider, she drew a great deal of virtue out of necessity, and dealt so
very handsomely with the undertaker, that this burial would be the talk
of the Close for some weeks. The coffin was inspected inside and out,
was admired and appraised, Mrs. Peckover being at hand to check the
estimates. At the same time every most revolting detail of the dead
woman's last illness was related and discussed and mused over and
exclaimed upon. ``A lovely corpse, considerin' her years,'' was the
general opinion. Then all went upstairs again, and once more refreshed
themselves. The house smelt like a bar-room.

``Everythink most respectable, I'm sure!'' remarked the female mourners
to each other, as they crowded together in the parlour.

``An' so it had ought to be!'' exclaimed one, in an indignant tone, such
as is reserved for the expression of offence among educated people, but
among the poor---the London {\protect\hypertarget{99}{}{}}poor, least
original and least articulate beings within the confines of
civilisation---has also to do duty for friendly emphasis. ``If Mrs.
Peckover can't afford to do things respectable, who can?''

And the speaker looked defiantly about her, as if daring contradiction.
But only approving murmurs replied. Mrs. Peckover had, in fact, the
reputation of being wealthy; she was always inheriting, always
accumulating what her friends called ``interess,'' never expending as
other people needs must. The lodgings she let enabled her to live
rent-free and rate-free. Clem's earnings at an artificial-flower factory
more than paid for that young lady's board and clothing; and all other
outlay was not worth mentioning as a deduction from the income created
by her sundry investments. Her husband---ten years deceased---had been a
``moulder;'' he earned on an average between three and four pounds a
week, and was so prudently disposed that, for the last decade of his
life, he made it a rule never to spend a farthing of
{\protect\hypertarget{100}{}{}}his wages. Mrs. Peckover at that time
kept a small beer-shop in Rosoman Street,---small and unpretending in
appearance, but through it there ran a beery Pactolus. By selling the
business shortly after her husband's death, Mrs. Peckover realised a
handsome capital. She retired into private life, having a strong sense
of personal dignity, and feeling it necessary to devote herself to the
moral training of her only child.

At half-past eleven Mrs. Peckover was arrayed in her mourning
robes,---new, dark-glistening. During her absence, Clem had kept guard
over Mrs. Gully, whom it was very difficult indeed to restrain from the
bottles and decanters; the elder lady coming to relieve, Clem could rush
away and don her own solemn garments. The undertaker with his men
arrived; the hearse and coaches drove up; the Close was in a state of
excitement. ``Now that's what I call a respectable turn out!'' was the
phrase passed from mouth to mouth in the crowd gathering near the door.
Children in great numbers had {\protect\hypertarget{101}{}{}}absented
themselves from school for the purpose of beholding this procession. ``I
do like to see spirited 'orses at a funeral!'' remarked one of the
mourners, who had squeezed his way to the parlour window. ``It puts the
finishin' touch, as you may say, don't it?''

When the coffin was borne forth, there was such a press in the street
that the men with difficulty reached the hearse. As the female mourners
stepped across the pavement with handkerchiefs held to their mouths, a
sigh of satisfaction was audible throughout the crowd; the males were
less sympathetically received, and some jocose comments from a
costermonger, whose business was temporarily interrupted, excited
indulgent smiles.

The procession moved slowly away, and the crowd, unwilling to disperse
immediately, looked about for some new source of entertainment. They
were fortunate, for at this moment came round the corner an individual
notorious throughout Clerkenwell as ``Mad Jack.'' Mad he presumably
was---at all events, an idiot. A lanky, raw-boned,
{\protect\hypertarget{102}{}{}}red-headed man, perhaps forty years old;
not clad, hut hung over with the filthiest rags; hatless, shoeless. He
supported himself by singing in the streets, generally psalms, and with
eccentric modulations of the voice which always occasioned mirth in
hearers. Sometimes he stood at a corner and began the delivery of a
passage of Scripture in French; how, where, or when he could have
acquired this knowledge was a mystery, and Jack would throw no light on
his own past. At present, having watched the funeral coaches pass away,
he lifted up his voice in a terrfic blare, singing, ``A11 ye works of
the Lord, bless ye the Lord, praise Him and magnify Him for ever.''
Instantly he was assailed by the juvenile portion of the throng, was
pelted with anything that came to hand, mocked mercilessly, buffeted
from behind. For a while he persisted in his psalmody, but at length,
without warning, he rushed upon his tormentors, and with angry shrieks
endeavoured to take revenge. The uproar continued till a policeman came
and cleared {\protect\hypertarget{103}{}{}}the way. Then Jack went off
again, singing, ``All ye works of the Lord.'' With his voice blended
that of the costermonger, ``Penny a bundill!''

Up in the Hewetts' back-room lay Jane Snowdon, now seemingly asleep, now
delirious. When she talked, a name was constantly upon her lips; she
kept calling for ``Mr. Kirkwood.'' Amy was at school; Annie and Tom
frequently went into the room and gazed curiously at the sick girl. Mrs.
Hewett felt so ill to-day that she could only lie on the bed and try to
silence her baby's crying.

The house-door was left wide open between the departure and return of
the mourners; a superstition of the people demands this. The Peckovers
brought back with them some half a dozen relatives and friends, invited
to a late dinner. The meal had been in preparation at an eating-house
close by, and was now speedily made ready in the parlour. A liberal
supply of various ales was furnished by the agency of a pot-boy (Jane's
absence being {\protect\hypertarget{104}{}{}}much felt), and in the
course of half an hour or so the company were sufficiently restored to
address themselves anew to the bottles and decanters. Mrs. Gully was now
permitted to obey her instincts; the natural result could be attributed
to over-strung feelings.

Just when the mourners had grown noisily hilarious, testifying thereby
to the respectability with which things were being conducted to the very
end, Mrs. Peckover became aware of a knocking at the front-door. She
bade her daughter go and see who it was. Clem, speedily returning,
beckoned her mother from among the guests.

``It's somebody wants to know if there ain't somebody called Snowdon
livin' 'ere,'' she whispered in a tone of alarm. ``An old man.''

Mrs. Peckover never drank more than was consistent with the perfect
clearness of her brain. At present she had very red cheeks, and her
cat-like eyes gleamed noticeably, but any kind of business would have
found her as shrewdly competent as ever.

{\protect\hypertarget{105}{}{}}``What did you say?'' she whispered
savagely.

``Said I'd come an' ask.''

``You stay 'ere. Don't say nothink.''

Mrs. Peckover left the room, closed the door behind her, and went along
the passage. On the doorstep stood a man with white hair, wearing an
unusual kind of cloak and a strange hat. He looked at the landlady
without speaking.

``What was you wantin', mister? ''

``I have been told,'' replied the man in a clear, grave voice, ``that a
child of the name of Snowdon lives in your house, ma'am.''

``Eh? Who told you that?''

``The people next door but one. I've been asking at many houses in the
neighbourhood. There used to be relations of mine lived somewhere here;
I don't know the house, nor the street exactly. The name isn't so very
common. If you don't mind, I should like to ask you who the child's
parents was.''

Mrs. Peckover's eyes were searching the speaker with the utmost
closeness.

{\protect\hypertarget{106}{}{}}``I don't mind tellin' you,'' she said,
``that there is a child of that name in the 'ouse, a young girl, at
least. Though I don't rightly know her age, I take her for fourteen or
fifteen.''

The old man seemed to consult his recollections.

``If it's any one I'm thinking of,'' he said slowly, ``she can't be
quite as old as that.''

The woman's face changed; she looked away for a moment.

``Well, as I was sayin', I don't rightly know her age. Any way, I'm
responsible for her. I've been a mother to her, an' a good
mother---though I say it myself---these six years or more. I look on her
now as a child o' my own. I don't know who you may be, mister. P'r'aps
you've come from abroad?''

"Yes, I have. There's no reason why I shouldn't tell you that I'm trying
to find any of my kin that are still alive. There was a married son of
mine that once lived somewhere about here. His name was Joseph James
Snowdon. When I last heard of him, {\protect\hypertarget{107}{}{}}he was
working at a 'lectroplater's in Clerkenwell. That was thirteen years
ago. I deal openly with you; I shall thank you if you'll do the like
with me."

``See, will you just come in? I've got a few friends in the front-room;
there's been a death in the `ouse, an' there's sickness, an' we're out
of order a bit. I'll ask you to come downstairs.''

It was late in the afternoon, and though lights were not yet required in
the upper rooms, the kitchen would have been all but dark save for the
fire. Mrs. Peckover lit a lamp and bade her visitor be seated. Then she
re-examined his face, his attire, his hands. Everything about him told
of a life spent in mechanical labour. His speech was that of an untaught
man, yet differed greatly from the tongue prevailing in Clerkenwell; he
was probably not a Londoner by birth, and---a point of more moment---he
expressed himself in the tone of one who is habitually thoughtful, who,
if the aid of books has been denied to him, still has won from life
{\protect\hypertarget{108}{}{}}the kind of knowledge which develops
character. Mrs. Peckover had small experience of faces which bear the
stamp of simple sincerity. This man's countenance put her out. As a
matter of course, he wished to over-reach her in some way, but he was
obviously very deep indeed. And then she found it so difficult to guess
his purposes. How would he proceed if she gave him details of Jane's
history, admitting that she was the child of Joseph James Snowdon? What,
again, had he been told by the people of whom he had made inquiries? She
needed time to review her position.

``As I was sayin', she resumed, poking the fire, ''I've been a mother to
her these six years or more, an' I feel I done the right thing by her.
She was left on my `ands by them as promised to pay for her keep; an' a
few months, I may say a few weeks, was all as ever I got. Another woman
would a sent the child to the `Ouse; but that's always the way with me;
I'm always actin' against my own interesses."

{\protect\hypertarget{109}{}{}}``You say that her parents went away and
left her?'' asked the old man, knitting his brows.

``Her father did. Her mother, she died in this very `ouse, an' she was
buried from it. He gave her a respectable burial, I'll say that much for
him. An' I shouldn't have allowed anything but one as was respectable to
leave this `ouse; I'd sooner a paid money out o' my own pocket. That's
always the way with me. Mr. Willis, he's my undertaker; you'll find him
at Number 17 Green Passage. He buried my 'usband; though that wasn't
from the Close; but I never knew a job turned out more respectable. He
was 'ere to day; we've only just buried my 'usband's mother. That's why
I ain't quite myself,---see?''

Mrs. Peckover was not won't to be gossippy. She became so at present,
partly in consequence of the stimulants she had taken to support her
through a trying ceremony, partly as a means of obtaining time to
reflect. Jane's unlucky illness made an especial
{\protect\hypertarget{110}{}{}}difficulty in her calculations. She felt
that the longer she delayed mention of the fact, the more likely was she
to excite suspicion; on the other hand, she could not devise the
suitable terms in which to reveal it. The steady gaze of the old man was
disconcerting. Not that he searched her face with a cunning scrutiny,
such as her own eyes expressed; she would have found that less
troublesome, as being familiar. The anxiety, the troubled anticipation,
which her words had aroused in him, were wholly free from shadow of
ignoble motive; he was pained, and the frequent turning away of his look
betrayed that part of the feeling was caused by observation of the woman
herself, but every movement visible on his features was subdued by
patience and mildness. Suffering was a life's habit with him, and its
fruit in this instance that which (spite of moral commonplace) it least
often bears,---self-conquest.

``You haven't told me yet,'' he said, with quiet disregard of her
irrelevancies, ``whether or not her father's name was Joseph Snowdon.''

{\protect\hypertarget{111}{}{}}``There's no call to hide it. That was
his name. I've got letters of his writin'. `J. J. Snowdon' stands at the
end, plain enough. And he was your son, was he?''

``He was. But have you any reason to think he's dead?''

"Dead! I never heard as he was. But then I never heard as he was livin'
neither. When his wife went, poor thing,---an' it was a chill on the
liver, they said; it took her very sudden,---he says to me, `Mrs.
Peckover,' he says, `I know you for a motherly woman,'---just like
that,---see?---'I know you for a motherly woman,' he says, `an' the idea
I have in my `ed is as I should like to leave Janey in your care,
'cause,' he says, `I've got work in Birmingham, an' I don't see how I'm
to take her with me. Understand me?' he says. `Oh!' I says,---not
feelin' quite sure what I'd ought to do,---see? `Oh!' I says. `Yes,' he
says; `an' between you an' me,' he says, `there won't be no
misunderstanding. If you'll keep Janey with you,'---an' she was goin' to
school at the time, 'cause {\protect\hypertarget{112}{}{}}she went to
the same as my own Clem,---that's Clemintiner,---understand?---'if
you'll keep Janey with you,' he says, `for a year, or maybe two years,
or maybe three years,---'cause that depends on
cirkinstances,'---understand?---'I'm ready,' he says, `to pay you what
it's right that pay I should, an' I'm sure,' he says, `as we shouldn't
misunderstand one another.' Well, of course I had my own girl to bring
up, an' my own son to look after too. A nice sort o' son; just when he
was beginnin' to do well, an' ought to a paid me back for all the
expense I was at in puttin' him to a business, what must he do but take
his 'ook to Australia."

Her scrutiny discerned something in the listener's face which led her to
ask:

``Perhaps you've been in Australia yourself, mister? ''

``I have.''

The woman paused, speculation at work in her eyes.

``Do you know in what part of the country your son is?'' inquired the
old man absently.

{\protect\hypertarget{113}{}{}}``He's wrote me two letters, an' the
last, as come more than a year ago, was from a place called
Maryborough.''

The other still preserved an absent expression; his eyes travelled about
the room.

``I always said,'' pursued Mrs. Peckover, ``as it was Snowdon as put
Australia into the boy's `ed. He used to tell us he'd got a brother
there, doin' well. P'r'aps it wasn't true.''

``Yes, it was true,'' replied the old man coldly. ``But you haven't told
me what came to pass about the child.''

An exact report of all that Mrs. Peckover had to say on this subject
would occupy more space than it merits. The gist of it was that for less
than a year she had received certain stipulated sums irregularly; that
at length no money at all was forthcoming; that in the tenderness of her
heart she had still entertained the child, sent her to school, privately
instructed her in the domestic virtues, trusting that such humanity
would not lack even its material reward, and that either Joseph
{\protect\hypertarget{114}{}{}}Snowdon or some one akin to him would
ultimately make good to her the expenses she had not grudged.

"She's a child as pays you back for all the trouble you take, so much I
\emph{will} say for her," observed the matron in conclusion. "Not as it
hasn't been a little `ard to teach her tidiness, but she's only a young
thing still. I shouldn't wonder but she's felt her position a little now
an' then; it's only natural in a growin' girl, do what you can to
prevent it. Still, she's willin'; that nobody can deny, an' I'm sure
\emph{I} should never wish to. Her cirkinstances has been peculiar; that
you'll understand, I'm sure. But I done my best to take the place of the
mother as is gone to a better world. An' now that she's layin' ill, I'm
sure no mother could feel it more{{------}}"

``Ill? Why didn't you mention that before?''

"Didn't I say as she was ill? Why, I thought it was the first word I
spoke as soon as you got into the 'ouse. You can't a
{\protect\hypertarget{115}{}{}}noticed it, or else it was me as is so
put about. What with havin' a burial"

``Where is she?'' asked the old man anxiously.

"Where? Why, you don't think as I'd a sent her to be looked after by
strangers? She's lay in' in Mrs. Hewett's room---that's one o' the
lodgers---all for the sake o' comfort. A better an kinder woman than
Mrs. Hewett you wouldn't find, not if you was to{{------}}"

With difficulty the stranger obtained a few details of the origin and
course of the illness,---details wholly misleading, but devised to
reassure. When he desired to see Jane, Mrs. Peckover assumed an air of
perfect willingness, but reminded him that she had nothing save his word
to prove that he had indeed a legitimate interest in the girl.

``I can do no more than tell you that Joseph James Snowdon was my
younger son,'' replied the old man simply. "I've come back to spend my
last years in England, and I hoped---I hope still---to find my son
{\protect\hypertarget{116}{}{}}I wish to take his child into my own
care; as he left her to strangers---perhaps he didn't do it willingly;
he may be dead---he could have nothing to say against me giving her the
care of a parent. You've been at expense{{------}}"

Mrs. Peckover waited with eagerness, but the sentence remained
incomplete. Again the old man's eyes strayed about the room. The current
of his thoughts seemed to change, and he said:

``You could show me those letters you spoke of---of my son's writing?''

``Of course I could,'' was the reply, in the tone of coarse resentment
whereby the scheming vulgar are wont to testify to their dishonesty.

``Afterwards---afterwards. I should like to see Jane, if you'll be so
good.''

The mild voice, though often diffident, now and then fell upon a note of
quiet authority which suited well with the speaker's grave, pure
countenance. As he spoke thus, Mrs. Peckover rose and said she would
first go upstairs just to see how things were. She was
{\protect\hypertarget{117}{}{}}absent ten minutes, then a little
girl---Amy Hewett---came into the kitchen and asked the stranger to
follow her.

Jane had been rapidly transferred from the mattress to the bedstead, and
the room had been put into such order as was possible. A whisper from
Mrs. Peckover to Mrs. Hewett, promising remission of half a week's rent,
had sufficed to obtain for the former complete freedom in her movements.
The child, excited by this disturbance, had begun to moan and talk
inarticulately. Mrs. Peckover listened for a moment, but heard nothing
dangerous. She bade the old man enter noiselessly, and herself went
about on tiptoe, speaking only in a hoarse whisper.

The visitor had just reached the bedside, and was gazing with deep,
compassionate interest at the unconscious face, when Jane, as if
startled, half rose and cried painfully, ``Mr. Kirkwood! oh, Mr.
Kirkwood!'' and she stretched her hand out, appearing to believe that
the friend she called upon was near her.

{\protect\hypertarget{118}{}{}}``Who is that?'' inquired the old man,
turning to his companion.

``Only a friend of ours,'' answered Mrs. Peck over, herself puzzled and
uneasy.

Again the sick girl called ``Mr Kirkwood!'' but without other words.
Mrs. Peckover urged the danger of this excitement, and speedily led the
way downstairs.
