{}

{CHAPTER X.}

THE LAST COMBAT.

\textsc{During} these summer months Sidney Kirkwood's visits to the
house in Clerkenwell Close were comparatively rare. It was not his own
wish to relax in any degree the close friendship so long subsisting
between the Hewitts and himself, but from the day of Clara's engagement
with Mrs. Tubbs John Hewett began to alter in his treatment of him. At
first there was nothing more than found its natural explanation in
regret of what had happened, a tendency to muteness, to troubled
brooding; but before long John made it unmistakable that the young man's
presence was irksome to him. If, on coming home, he found Sidney with
Mrs. Hewett and the children, a cold nod was the only greeting he
offered; then followed signs of ill-humour, {}such as Sidney could not
in the end fail to interpret as unfavourable to himself. He never heard
Clara's name on her father's lips, and himself never uttered it when
John was in hearing.

``She told him what passed between us that night,'' Sidney argued
inwardly. But it was not so. Hewett had merely abandoned himself to an
unreasonable resentment. Notwithstanding his concessions, he blamed
Sidney for the girl's leaving home, and, as his mood grew more
irritable, the more hopeless it seemed that Clara would return, he
nursed the suspicion of treacherous behaviour on Sidney's part. He would
not take into account any such thing as pride which could forbid the
young man to urge a rejected suit. Sidney had grown tired of Clara, that
was the truth, and gladly caught at any means of excusing himself. He
had made new friends. Mrs. Peckover reported that he was a constant
visitor at the old man Snowdon's lodgings; she expressed her belief that
Snowdon had come back from Australia with a little store {}of money, and
if Kirkwood had knowledge of that, would it not explain his interest in
Jane Snowdon?

``For shame to listen to such things!'' cried Mrs. Hewett, angrily, when
her husband once repeated the landlady's words. ``I'd be ashamed of
myself, John! If you don't know him no better than that, you ought to by
this time.''

And John did, in fact, take to himself no little shame, but his
unsatisfied affection turned all the old feelings to bitterness. In
spite of himself, he blundered along the path of perversity. Sidney,
too, had his promptings of obstinate humour. When he distinctly
recognised Hewett's feeling it galled him; he was being treated with
gross injustice, and temper suggested reprisals which could answer no
purpose but to torment him with self-condemnation. However, he must
needs consult his own dignity; he could not keep defending himself
against ignoble charges. For the present, there was no choice but to
accept John's hints, and hold apart as much as was {}possible without
absolute breach of friendly relations. Nor could he bring himself to
approach Clara. It was often in his mind to write to her; had he obeyed
the voice of his desire, he would have penned such letters as only the
self-abasement of a passionate lover can dictate. But herein, too, the
strain of sternness that marked his character made its influence felt.
He said to himself that the only hope of Clara's respecting him lay in
his preservation of the attitude he had adopted, and as the months went
on he found a bitter satisfaction in adhering so firmly to his purpose.
The self-flattery with which no man can dispense whispered assurance
that Clara only thought the more of him the longer he held aloof. When
the end of July came, he definitely prescribed to his patience a trial
of yet one more month. Then he would write Clara a long letter, telling
her what it had cost him to keep silence, and declaring the constancy he
devoted to her.

This resolve he registered whilst at work one morning. The triumphant
sunshine, {}refusing to be excluded even from London workshops, gleamed
upon his tools and on the scraps of jewellery before him; he looked up
to the blue sky, and thought with heavy heart of many a lane in Surrey
and in Essex where he might be wandering but for this ceaseless
necessity of earning the week's wage. A fly buzzed loudly against the
grimy window, and by one of those associations which time and change
cannot affect, he mused himself back into boyhood. The glimpse before
him of St. John's Arch aided the revival of old impressions; his hand
ceased from its mechanical activity, and he was absorbed in a waking
dream, when a voice called to him and said that he was wanted. He went
down to the entrance, and there found Mrs. Hewett. Her coming at all was
enough to signal some disaster, and the trouble on her face caused
Sidney to regard her with silent interrogation.

``I couldn't help comin' to you,'' she began, gazing at him fixedly. ``I
know you can't do anything, but I had to speak to somebody, {}an' I know
nobody better than you. It's about Clara.''

``What about her?''

``She's left Mrs. Tubbs. They had words about Bank-holiday last night,
an' Clara went off at once. Mrs. Tubbs thought she'd come 'ome, but this
mornin' her box was sent for, an' it was to be took to a house in
Islington. An' then Mrs. Tubbs came an' told me. An' there's worse than
that, Sidney. She's been goin' about to the theatre an' such places with
a man as she got to know at the bar, an' Mrs. Tubbs says she believes
it's him has tempted her away.''

She spoke the last sentences in a low voice, painfully watching their
effect.

``And why hasn't Mrs. Tubbs spoken about this before?'' Sidney asked,
also in a subdued voice, but without other show of agitation.

``That's just what I said to her myself. The girl was in her charge, an'
it was her duty to let us know if things went wrong. But how am I to
tell her father? I dursn't do it, Sidney; for my life, I dursn't! I'd go
{}an' see her where she's lodging,---see, I've got the address wrote
down here,---but I should do more harm than good; she'd never pay any
heed to me at the best of times, an' it isn't likely she would now.''

``Look here! If she's made no attempt to hide away, you may be quite
sure there's no truth in what Mrs. Tubbs says. They've quarrelled, and
of course the woman makes Clara as black as she can. Tell her father
everything as soon as he comes home; you've no choice.''

Mrs. Hewett averted her face in profound dejection. Sidney learnt at
length what her desire had been in coming to him; she hoped he would see
Clara and persuade her to return home.

``I dursn't tell her father,'' she kept repeating. ``But perhaps it
isn't true what Mrs. Tubbs says. Do go an' speak to her before it's too
late. Say we won't ask her to come 'ome, if only she'll let us know what
she's goin' to do.''

In the end he promised to perform this {}service, and to communicate the
result that evening. It was Saturday; at half-past one he left the
workroom, hastened home to prepare himself for the visit, and without
thinking of dinner, set out to find the address Mrs. Hewett had given
him. His steps were directed to a dull street on the north of
Pentonville Road; the house at which he made inquiry was occupied by a
drum-manufacturer. Miss Hewett, he learnt, was not at home; she had gone
forth two hours ago, and nothing was known of her movements. Sidney
turned away and began to walk up and down the shadowed side of the
street; there was no breath of air stirring, and from the open windows
radiated stuffy odours. A quarter of an hour sufficed to exasperate him
with anxiety and physical malaise. He suffered from his inability to do
anything at once, from conflict with himself as to whether or not it
behoved him to speak with John Hewett; of Clara he thought with anger
rather than fear, for her behaviour seemed to prove that nothing had
happened save the {}inevitable breach with Mrs. Tubbs. Just as he had
said to himself that it was no use waiting about all the afternoon, he
saw Clara approaching. At sight of him she manifested neither surprise
nor annoyance, but came forward with eyes carelessly averted. Not having
seen her for so long, Sidney was startled by the change in her features;
her cheeks had sunk, her eyes were unnaturally dark, there was something
worse than the familiar self-will about her lips.

``I've been waiting to see you,'' he said.

``Will you walk along here for a minute or two?''

``What do you want to say? I'm tired.''

``Mrs. Tubbs has told your mother what has happened, and she came to me.
Your father doesn't know yet.''

``It's nothing to me whether he knows or not. I've left the place,
that's all, and I'm going to live here till I've got another.''

``Why not go home?''

``Because I don't choose to. I don't see that it concerns you, Mr.
Kirkwood.''

{}Their eyes met, and Sidney felt how little fitted he was to reason
with the girl, even would she consent to hear him. His mood was the
wrong one; the torrid sunshine seemed to kindle an evil fire in him, and
with difficulty he kept back words of angry unreason; he
even---strangest of inconsistencies---experienced a kind of brutal
pleasure in her obvious misery. Already she was reaping the fruit of
obstinate folly, Clara read what his eyes expressed; she trembled with
responsive hostility.

``No, it doesn't concern me,'' Sidney replied, half turning away. ``But
it's perhaps as well you should know that Mrs. Tubbs is doing her best
to take away your good name. However little we are to each other, it's
my duty to tell you that, and put you on your guard. I hope your father
mayn't hear these stories before you have spoken to him yourself.''

Clara listened with a contemptuous smile.

``What has she been saying?''

``I shan't repeat it.''

{}As he gazed at her, the haggardness of her countenance smote like a
sword-edge through all the black humours about his heart, piercing the
very core of love and pity. He spoke in a voice of passionate appeal.

``Clara, come home before it is too late! Come with me---now---come at
once! Thank heaven you have got out of that place! Come home, and stay
there quietly till we can find you something better.''

``I'll die rather than go home!'' was her answer, flung at him as if in
hatred, ``Tell my father that, and tell him anything else you like. I
want no one to take any thought for me; and I wouldn't do as you wish,
not to save my soul!''

How often, in passing along the streets, one catches a few phrases of
discord such as this! The poor can seldom command privacy; their scenes
alike of tenderness and of anger must for the most part be enacted on
the peopled ways. It is one of their misfortunes, one of the many
necessities which blunt feeling, which balk reconciliation, {}which
enhance the risks of dialogue at best semi-articulate.

Clara, having uttered the rancour which had so long poisoned her mind,
straightway crossed the street and entered the house where she was
lodging. She had just returned from making several applications for
employment,---futile, as so many were likely to be, if she persevered in
her search for a better place than the last. The wages due to her for
the present week she had of course sacrificed; her purchases of
clothing---essential and superfluous---had left only a small sum out of
her earnings. Food, fortunately, would cost her little; the difficulty,
indeed, was to eat anything at all.

She was exhausted after her long walk, and the scene with Sidney had
made her tremulous. In thrusting open the windows, as soon as she
entered, she broke a pane which was aleady cracked; the glass cut into
her palm, and blood streamed forth. For a moment she watched the red
drops falling to the floor, then began to sob miserably, almost as a
child {}might have done. The exertion necessary for binding the wound
seemed beyond her strength; sobbing and moaning, she stood in the same
attitude until the blood began to congeal. The tears, too, she let dry
unheeded upon her eyelashes and her cheeks; the mist with which for a
time they obscured her vision was nothing amid that cloud of misery
which blackened about her spirit as she brooded. The access of self-pity
was followed, as always, by a persistent sense of intolerable wrong, and
that again by a fierce desire to plunge herself into ruin, as though by
such act she could satiate her instincts of defiance. It is a phase of
exasperated egotism common enough in original natures frustrated by
circumstance,---never so pronounced as in those who suffer from the
social disease. Such mood perverts everything to cause of bitterness.
The very force of sincerity, which Clara could not but recognise in
Kirkwood's appeal, inflamed the resentment she nourished against him;
she felt that to yield would be salvation and happiness, yet yield she
might not, and {}upon him she visited the anger due to the evil impulses
in her own heart. He spoke of her father, and in so doing struck the
only nerve in her which conveyed an emotion of tenderness; instantly the
feeling begot self-reproach, and of self-reproach was born as quickly
the harsh self-justification with which her pride ever answered blame.
She had made her father's life even more unhappy than it need have been,
and to be reminded of that only drove her more resolutely upon the
recklessness which would complete her ingratitude.

The afternoon wore away, the evening, a great part of the night. She ate
a few mouthfuls of bread, but could not exert herself to make tea. It
would be necessary to light a fire, and already the air of the room was
stifling.

After a night of sleeplessness, she could only lie on her bed through
the Sunday morning, wretched in a sense of abandonment. And then began
to assail her that last and subtlest of temptations, the thought that
already she had taken an irrevocable {}step, that an endeavour to return
would only be trouble spent in vain, that the easy course was in truth
the only one now open to her. Mrs. Tubbs was busy circulating calumnies;
that they were nothing more than calumnies could never be proved; all
who heard them would readily enough believe. Why should she struggle
uselessly to justify herself in the eyes of people predisposed to
condemn her? Fate was busy in all that had happened during the last two
days. Why had she quitted her situation at a moment's notice? Why on
this occasion rather than fifty times previously? It was not her own
doing; something impelled her, and the same force---call it chance or
destiny---would direct the issue once more. All she could foresee was
the keeping of her appointment with Scawthorne to-morrow morning; what
use to try and look further, when assuredly a succession of
circumstances impossible to calculate would in the end constrain her?
The best would be if she could sleep out the interval.

At mid-day she rose, ate and drank {}mechanically, then contemplated the
hours that must somehow be killed. There was sunlight in the sky, but to
what purpose should she go out? She went to the window, and surveyed the
portion of street that was visible. On the opposite pavement, at a
little distance, a man was standing; it was Sidney Kirkwood. The sight
of him roused her from apathy; her blood tingled, rushed into her cheeks
and throbbed at her temples. So, for all she had said, he was daring to
act the spy! He suspected her; he was lurking to surprise visitors, to
watch her outgoing and coming in. Very well; at least he had provided
her with occupation.

Five minutes later she saw that he had gone away. Thereupon---having in
the meantime clad herself---she left the house and walked at a quick
step towards a region of north London with which she had no
acquaintance. In an hours time she had found another lodging, which she
took by the day only. Then back again to Islington. She told her
landlady that a sudden necessity compelled {}her to leave; she would
have a cab and remove her box at once. There was the hazard that Sidney
might return just as she was leaving; she braved it, and in another ten
minutes was out of reach\ldots{}.

Let his be the blame. She had warned him, and he chose to disregard her
wish. Now she had cut the last bond that fretted her, and the hours
rushed on like a stormwind, driving her whither they would.

Her mind was relieved from the stress of conflict; despair had given
place to something that made her laugh at all the old scruples. So far
from dreading the judgments that would follow her disappearance, she
felt a pride in evil repute. Let them talk of her! If she dared
everything, it would be well understood that she had not done so without
a prospect worthy of herself If she broke away from the obligations of a
life that could never be other than poor and commonplace, those who knew
her would estimate the compensation she had found. Sidney Kirkwood was
aware of her ambitions; for his {}own sake he had hoped to keep her on
the low level to which she was born; now let him recognise his folly!
Some day she would present herself before him:---``Very sorry that I
could not oblige you, my dear sir, but you see that my lot was to be
rather different from that you kindly planned for me.'' Let them gossip
and envy!

It was a strange night that followed. Between one and two o'clock the
heavens began to be overflashed with summer lightning; there was no
thunder, no rain. The blue gleams kept illuminating the room for more
than an hour. Clara could not lie in bed. The activity of her brain
became all but delirium; along her nerves, through all the courses of
her blood, seemed to run fires which excited her with an indescribable
mingling of delight and torment. She walked to and fro, often speaking
aloud, throwing up her arms. She leaned from the open window and let the
lightning play freely upon her face; she fancied it had the effect of
restoring her wasted health. Whatever the cause, she felt {}stronger and
more free from pain than for many months.

At dawn she slept. The striking of a church-clock woke her at nine,
giving her just time to dress with care and set forth to keep her
appointment.
