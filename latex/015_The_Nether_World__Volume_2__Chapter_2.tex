\chapter{Sunlight in Dreary Places}

\textsc{Among} the by-ways of Clerkenwell you might, with some
difficulty, have discovered an establishment known in its neighbourhood
as ``Whitehead's.'' It was an artificial-flower factory, and the rooms
of which it consisted were only to be reached by traversing a
timber-yard and then mounting a wooden staircase outside a saw-mill.
Here at busy seasons worked some threescore women and girls, who, owing
to the nature of their occupation, were spoken of by the jocose youth of
the locality as ``Whitehead's pastepots.''

Naturally they varied much in age and aspect. There was the child who
had newly left school and was now invited to consider the question of
how to keep herself alive; there was the woman of uncertain age, who
{}had spent long years of long days in the atmosphere of workrooms, and
showed the result in her parchmenty cheek and lack-lustre eye; and
between these extremes came all the various types of the London
craftsgirl: she who is young enough to hope that disappointments may yet
be made up for by the future; she who is already tasting such scanty
good as life had in store for her; she who has outlived her illusions
and no longer cares to look beyond the close of the week. If regularly
engaged as time-workers, they made themselves easy in the prospect of
wages that allowed them to sleep under a roof and eat at certain
intervals of the day; if employed on piece-work, they might at any
moment find themselves wageless, but this, being a familiar state of
things, did not trouble them. With few exceptions, they were clad
neatly; on the whole, they plied their task in wonderful contentment.
The general tone of conversation among them was not high; moralists
unfamiliar with the ways of the nether world would probably have applied
a term other {}than negative to the laughing discussions which now and
then enlivened this or that group; but it was very seldom indeed that a
child newly arriving heard anything with which she was not already
perfectly familiar.

One afternoon at the end of May there penetrated into the largest of the
workrooms that rarest of visitants, a stray sunbeam. Only if the sun
happened to shine at given moments could any of its light fall directly
into the room I speak of; this afternoon, however, all circumstances
were favourable, and behold the floor chequered with uncertain gleam.
The workers were arranged in groups of three, called ``parties,''
consisting of a learner, an improver, and a hand. All sat with sleeves
pushed up to their elbows, and had a habit of rocking to and fro as they
plied their mechanical industry. Owing to the movement of a cloud, the
sunlight spread gradually towards one of these groups; it touched the
skirt, the arms, the head of one of the girls, who, as if gladdened by
the {}kindly warmth, looked round and smiled. A smile you would have
been pleased to observe,---unconscious, gently thoughtful, rich in
possibilities of happiness. She was quite a young girl, certainly not
seventeen, and wore a smooth grey dress, with a white linen collar; her
brown hair was closely plaited, her head well-shaped, the bend of her
neck very graceful. From her bare arms it could be seen that she was
anything but robustly made, yet her general appearance was not one of
ill-health, and she held herself, even thus late in the day, far more
uprightly than most of her companions. Had you watched her for a while,
you would have noticed that her eyes occasionally strayed beyond the
work-table, and, perhaps unconsciously, fixed themselves for some
moments on one or other of the girls near her; when she remembered
herself and looked down again upon her task, there rose to her face a
smile of the subtlest meaning, the outcome of busy reflection.

By her side was a little girl just beginning to learn the work, whose
employment it was {}to paper wires and make ``centres.'' This toil
always results in blistered fingers, and frequent was the child's appeal
to her neighbour for sympathy.

``It'll be easier soon,'' said the latter, on one of these occasions,
bending her head to speak in a low voice. ``You should have seen what
blisters I had when I began.''

``It's all very well to say that. I can't do no more, so there! Oh,
when'll it be five o'clock?''

``It's a quarter to. Try and go on, Annie.''

Five clock did come at length, and with it twenty minutes' rest for tea.
The rule at Whitehead's was, that you could either bring your own tea,
sugar, and eatables, or purchase them here from a forewoman; most of the
workers chose to provide themselves. It was customary for each ``party''
to club together, emptying their several contributions of tea out of
little twists of newspaper into one teapot. Wholesome bustle and
confusion succeeded to the former silence. One of the {}learners, whose
turn it was to run on errands, was overwhelmed with commissions to a
chandler's shop close by; a wry-faced, stupid little girl she was, and
they called her, because of her slowness, the ``funeral horse.'' She had
strange habits, which made laughter for those who knew of them; for
instance, it was her custom in the dinner-hour to go apart and eat her
poor scraps on a doorstep close by a cook-shop; she confided to a
companion that the odour of baked joints seemed to give her food a
relish. From her present errand she returned with a strange variety of
dainties,---for it was early in the week, and the girls still had
coppers in their pockets; for two or three she had purchased a
farthing's-worth of jam, which she carried in paper. A bite of this and
a taste of that rewarded her for her trouble.

The quiet-mannered girl whom we were observing took her cup of tea from
the pot in which she had a share, and from her bag produced some folded
pieces of bread and butter. She had begun her meal, when there came
{}and sat down by her a young woman of very different appearance,---our
friend, Miss Peckover. They were old acquaintances; but when we first
saw them together it would have been difficult to imagine that they
would ever sit and converse as at present, apparently in all
friendliness. Strange to say, it was Clem who, during the past three
years, had been the active one in seeking to obliterate disagreeable
memories. The younger girl had never repelled her, but was long in
overcoming the dread excited by Clem's proximity. Even now she never
looked straight into Miss Peckover's face, as she did when speaking with
others; there was reserve in her manner, reserve unmistakable, though
clothed with her pleasant smile and amiable voice.

``I've got something to tell you, Jane,'' Clem began, in a tone
inaudible to those who were sitting near. ``Something as'll surprise
you.''

``What is it, I wonder?''

``You must swear you won't tell nobody.''

{}Jane nodded. Then the other brought her head a little nearer, and
whispered:

``I'm goin' to be married!''

``Are you really?''

``In a week. Who do you think it is? Somebody as you know of, but if you
guessed till next Christmas you'd never come right.''

Nor had Clem any intention of revealing the name, but she laughed
consumedly, as if her reticence covered the most amusing situation
conceivable.

``It'll be the biggest surprise you ever had in your life. You've swore
you won't speak about it. I don't think I shall come to work after this
week,---but you'll have to come an' see us. You'll promise to, won't
you?''

Still convulsed with mirth, Clem went off to another part of the room.
From Jane's countenance the look of amusement which she had perforce
summoned soon passed; it was succeeded by a shadow almost of pain, and
not till she had been at work again for nearly an hour was the former
placidity restored to her.

{}When final release came, Jane was among the first to hasten down the
wooden staircase and get clear of the timber-yard. By the direct way, it
took her twenty minutes to walk from Whitehead's to her home in Hanover
Street, but this evening she had an object in turning aside. The visit
she wished to pay took her into a disagreeable quarter, a street of
squalid houses, swarming with yet more squalid children. On all the
doorsteps sat little girls, themselves only just out of infancy, nursing
or neglecting bald, red-eyed, doughy-limbed abortions in every stage of
babyhood, hapless spawn of diseased humanity, born to embitter and
brutalise yet further the lot of those who unwillingly gave them life.
With wide, pitiful eyes Jane looked at each group she passed. Three
years ago she would have seen nothing but the ordinary and the
inevitable in such spectacles, but since then her moral and intellectual
being had grown on rare nourishment; there was indignation as well as
heartache in the feeling with which she had learnt to regard the {}world
of her familiarity. To enter the house at which she paused it was
necessaiy to squeeze through a conglomerate of dirty little bodies. At
the head of the first flight of stairs she came upon a girl sitting in a
weary attitude on the top step and beating the wood listlessly with the
last remnant of a hearth-brush; on her lap was one more specimen of the
infinitely-multiplied baby, and a child of two years sprawled behind her
on the landing.

``Waiting for him to come home, Pennyloaf?'' said Jane.

``Oh, is that you, Miss Snowdon!'' exclaimed the other, returning to
consciousness and manifesting some shame at being discovered in this
position. Hastily she drew together the front of her dress, which for
the baby's sake had been wide open, and rose to her feet. Pennyloaf was
not a bit more womanly in figure than on the day of her marriage; her
voice was still an immature treble; the same rueful uTesponsibility
marked her features; but all her poor prettiness was {}Wasted under the
disfigurement of pains and cares. Incongruously enough, she wore a gown
of briglit-pattemed calico, and about her neck had a collar of
pretentious lace; her hair was dressed as if for a holiday, and a daub
recently made on her cheeks by the baby's fingers lent emphasis to the
fact that she had but a little while ago washed herself with much care.

``I can't stop,'' said Jane, ``but I thought I'd just look in and speak
a word. How have you been getting on?''

``Oh, do come in for just a minute!'' pleaded Pennyloaf, moving
backwards to an open door, whither Jane followed. They entered a
room,---much like other rooms that we have looked into from time to
time. Following the nomadic custom of their kind, Bob Hewett and his
wife had lived in six or seven different lodgings since their honeymoon
in Shooter's Gardens. Mrs. Candy first of all made a change necessary,
as might have been anticipated, and the restlessness of domestic
ill-being subsequently drove them {}from place to place. ``Come in `ere,
Johnny,'' she called to the child lying on the landing. ``What's the
good o' yvashin yon, I'd like to know! Just see, Miss Snowdon, he's made
his face all white with the milk as the boy spilt on the stairs! Take
this brush an' play with it, do! I \emph{can't} keep 'em clean, Miss
Snowdon, so it's no use talkin'.''

``Are you going somewhere to-night?''

Jane inquired, with a glance at the strange costume. Pennyloaf looked up
and down in a shamefaced way.

``I only did it just because I thought he might like to see me. He
promised me faithful as he'd come 'ome to-night, and I thought---it's
only somethink as got into my 'ed to-day, Miss Snowdon.''

``But hasn't he been coming home since I saw you last?''

"He did just once, an' then it was all the old ways again. I did what
you told me; I did, as sure as I'm a-standin `ere! I made the room so
clean you wouldn't have {}believed; I scrubbed the floor an' the table,
an' I washed the winders,---you can see they ain't dirty yet. An' he'd
never a' paid a bit o' notice if I hadn't told him. He was jolly enough
for one night, just like he can be when he likes. But I knew as it
wouldn't last, an' the next night he was off with a lot o' fellers an'
girls, same as ever. I didn't make no row when he came 'ome; I wish I
may die if I said a word to set his back up! An' I've gone on just the
same all the week; we haven't had not the least bit of a row; so you see
I kep' my promise. But it's no good; he won't come 'ome; he's always got
fellers an' girls to go round with. He took his hoath as he'd come back
to-night, an' then it come into my 'ed as I'd put my best things on,
just to---you know what I mean, Miss Snowdon. But he won't come before
twelve o'clock; I know he won't. An' I get that low sittin' `ere, you
can't think! I can't go nowhere, because o' the children. If it wasn't
for them I could go to work again, an' I'd be that glad; I feel as if my
'ed would {}drop off sometimes! I \emph{ham} so glad you just come in!''

Jane had tried so many forms of encouragement, of consolation, on
previous occasions that she knew not how to repeat herself. She was
ashamed to speak words which sounded so hollow and profitless. This
silence was only too significant to Pennyloaf, and in a moment she
exclaimed with querulous energy:

``I know what'll be the hend of it! I'll go an' do like mother does,---I
will! I will! I'll put my ring away, an' I'll go an' sit all night in
the public-`ouse! It's what all the others does, an' I'll do the same. I
often feel I'm a fool to go on like this. I don't know what I live for.
P'r'aps he'll be sorry when I get run in like mother.''

``Don't talk like that, Pennyloaf!'' cried Jane, stamping her foot. (It
was odd how completely difference of character had reversed their
natural relations to each other; Pennyloaf was the child, Jane the
mature woman.) ``You know better, and you've no {}right to give way to
such thoughts. I was going to say I'd come and be with you all Saturday
afternoon, but I don't know whether I shall now. And I'd been thinking
you might like to come and see me on Sunday, but I can't have people
that go to the publichouse, so we won't say anything more about it. I
shall have to be off; good-bye!''

She stepped to the door.

``Miss Snowdon!''

Jane turned, and after an instant of mock severity, broke into a laugh
which seemed to fill the wretched den with sunlight. Words, too, she
found; words of soothing influence such as leap from the heart to the
tongue in spite of the heavy thoughts that try to check them. Pennyloaf
was learning to depend upon these words for strength in her desolation.
They did not excite her to much hopefulness, but there was a sustaining
power in their sweet sincerity which made all the difference between
despair tending to evil and the sigh of renewed effort. ``I don't
care,'' Pennyloaf had got into the habit of thinking, {}after her
friend's departure; ``I won't give up as long as she looks in now and
then.''

Out from the swarm of babies Jane hmTied homewards. She had a reason for
wishing to be back in good time to-night; it was Wednesday, and on
Wednesday evening there was wont to come a visitor, who sat for a couple
of hours in her grandfather's room and talked, talked,---the most
interesting talk Jane had ever heard or could imagine. A latch-key
admitted her; she ran up to the second floor. A voice from the front
room caught her ear; certainly not \emph{his} voice,---it was too
early,---but that of some unusual visitor. She was on the point of
entering her own chamber, when the other door opened, and somebody
exclaimed, ``Ah, here she is!''

The speaker was an old gentleman, dressed in black, bald, with small and
rather rugged features; his voice was pleasant. A gold chain and a bunch
of seals shone against his waistcoat, also a pair of eyeglasses. A
professional man, obviously. Jane remembered {}that she had seen him
once before, about a year ago, when he had talked with her for a few
minutes, very kindly.

``Will you come in here, Jane? '' her grandfather's voice called to her.

Snowdon had changed much. Old age was heavy upon his shoulders, and had
even produced a slight tremulousness in his hands; his voice told the
same story of enfeeblement. Even more noticeable was the ageing of his
countenance. Something more, however, than the progress of time seemed
to be here at work. He looked strangely careworn; his forehead was set
in lines of anxiety; his mouth expressed a nervousness of which formerly
there had been no trace. One would have said that some harassing
preoccupation must have seized his mind. His eyes were no longer merely
sad and absent, but restless with fatiguing thought. As Jane entered the
room he fixed his gaze upon her,---a gaze that appeared to reveal
worrying apprehension.

``You remember Mr. Percival, Jane,'' he said.

{}The old gentleman thus presented held out his hand with something of
fatherly geniality.

``Miss Snowdon, I hope to have the pleasure of seeing you again hefore
long, but just now I am carrying off your grandfather for a couple of
hours, and indeed we mustn't linger that number of minutes. You look
well, I think?''

He stood and examined her intently, then cried:

``Come, my dear sir, come! we shall be late.''

Snowdon was already prepared for walking. He spoke a few words to Jane,
then followed Mr. Percival downstairs.

Flurried by the encounter, Jane stood looking about her. Then came a
rush of disappointment, as she reflected that the visitor of Wednesday
evenings would call in vain. Hearing that her grandfather was absent,
doubtless he would take his leave at once. Or, would he{{------}}?

In a minute or two she ran downstairs to exchange a word with Mrs.
Byass. On {}entering the kitchen, she was surprised to see Bessie
sitting idly by the fire. At this hour it was usual for Mr. Byass to
have returned, and there was generally an uproar of laughing talk. This
evening, dead silence, and a noticeable something in the air which told
of trouble. The baby--- of course a new baby---lay in a bassinette near
its mother, seemingly asleep; the other child was sitting in a high
chair by the table, clattering ``bricks.''

Bessie did not even look round.

``Is Mr. Byass late?'' inquired Jane, in an apprehensive voice.

``He's somewhere in the house, I believe,'' was the answer, in monotone.

Oh dear! Jane recognised a situation which had aleady come under her
notice once or twice during the last six months. She drew near, and
asked in a low voice:

``What's happened, Mrs. Byass?''

``He's a beast! If he doesn't mind I shall go and leave him. I mean
it!''

Bessie was in a genuine fit of sullenness. One of her hands was clenched
below her {}chin; her pretty lips were not pretty at all; her brow was
rumpled. Jane began to seek for the cause of dissension, to put
affectionate questions, to use her voice soothingly.

``He's a beast!'' was Bessie's reiterated observation; but by degrees
she added phrases more explanatory. ``How can I help it if he cuts
himself when he's shaving?---Serve him right!---What for? Why, for
saying that babies was nothing but a nuisance, and that my baby was the
ugliest and noisiest ever born!''

``Did she cry in the night?'' inquired Jane, with sympathy.

``Of course she did! Hasn't she a right to?''

``And then Mr. By ass cut himself with his razor?``

``Yes. And he said it was because he was woke so often, and it made him
nervous, and his hand shook. And then I told him he'd better cut himself
on the other side, and it wouldn't matter. And then he complained
because he had to wait for breakfast. And {}he said there'd been no
comfort in the house since we'd had children. And I cared nothing about
him, he said, and only about the baby and Ernest. And he went on like a
beast, as he is! I hate him!''

``Oh no, not a bit of it!'' said Jane, seeing the opportunity for a
transition to jest.

``I do! And you may go upstairs and tell him so.''

``All right; I will.''

Jane ran upstairs and knocked at the door of the parlour. A gruff voice
bade her enter, but the room was nearly in darkness.

``Will you have a light, Mr. Byass?''

``No--- thank you.''

``Mr. Byass, Mrs. Byass says I'm to say she hates you.''

``All right. Tell her I've known it a long time. She needn't trouble
about me; I'm going out to enjoy myself.''

Jane ran back to the kitchen.

``Mr. Byass says he's known it a long time,'' she reported, with much
gravity. ``And he's going out to enjoy himself.''

{}Bessie remained mute.

``What message shall I take back, Mrs. Byass?''

``Tell him if he dares to leave the house, I'll go to mother's the first
thing to-morrow, and let them know how he's treating me.''

``Tell her,'' was Mr. Byass's reply, ``that I don't see what it matters
to her whether I'm at home or away. And tell her she's a cruel wife to
me.''

Something like the sound of a snivel came out of the darkness as he
concluded. Jane, in reporting his speech, added that she thought he was
shedding tears. Thereupon Bessie gave a sob, quite in earnest.

``So am I,'' she said, chokingly. ``Go and tell him, Jane.''

``Mr. Byass, Mrs. Byass is crying,'' whispered Jane at the parlour-door.
``Don't you think you'd better go downstairs?''

Hearing a movement, she ran to be out of the way. Samuel left the dark
room, and with slow step descended to the kitchen. Then Jane knew that
it was all right, and {}tripped up to her room humming a song of
contentment.

Had she, then, wholly outgrown the bitter experiences of her childhood?
Had the cruelty which tortured her during the years when the soul is
being fashioned left upon her no brand of slavish vice, nor the baseness
of those early associations affected her with any irremovable taint? As
far as human observation could probe her, Jane Snowdon had no spot of
uncleanness in her being; she had been rescued while it was yet time,
and the subsequent period of fostering had enabled features of her
character, which no one could have discerned in the helpless child, to
expand with singular richness. Two effects of the time of her bondage
were, however, clearly to be distinguished. Though nature had endowed
her with a good intelligence, she could only with extreme labour acquire
that elementary book-knowledge which vulgar children get easily enough;
it seemed as if the bodily overstrain at a critical period of life had
affected her memory, and her power of mental application {}generally. In
spite of ceaseless endeavour, she could not yet spell words of the least
difficulty; she could not do the easiest sums with accuracy;
geographical names were her despair. The second point in which she had
suffered harm was of more serious nature. She was subject to fits of
hysteria, preceded and followed by the most painful collapse of that
buoyant courage which was her supreme charm and the source of her
influence. Without warning, an inexplicable terror would fall upon her;
like the weakest child, she craved protection from a dread inspired
solely by her imagination, and solace for an anguish of wTetchedness to
which she could give no form in words. Happily this illness afflicted
her only at long inters\^{}als, and her steadily improving health gave
waiTant for hoping that in time it would altogether pass away.

Whenever an opportunity had offered for struggling successfully with
some form of evil,---were it poor Pennyloaf's dangerous despair, or the
veiy- human difficulties between Bessie and her husband,---Jane lived at
her highest {}reach of spiritual joy. For all that there was a
disappointment on her mind, she felt this joy to-night, and went about
her pursuits in happy self-absorption. So it befell that she did not
hear a knock at the house-door. Mrs. Byass answered it, and not knowing
that Mr. Snowdon was from home, bade his usual visitor go upstairs. The
visitor did so, and announced his presence at the door of the room.

``Oh, Mr. Kirkwood,'' said Jane, ``I'm so sorry, but grandfather had to
go out with a gentleman.''

And she waited, looking at him, a gentle warmth on her face.
