\hypertarget{headerContainer}{}
\hypertarget{navigationHeader}{}
\protect\hypertarget{headerprevious}{}{←\href{/wiki/The_Nether_World/Volume_2/Chapter_2}{Volume
2, Chapter 2}}

\textbf{\protect\hypertarget{header_title_text}{}{\href{/wiki/The_Nether_World}{The
Nether World}}} \emph{by
\href{/wiki/Author:George_Gissing}{\protect\hypertarget{header_author_text}{}{{George
Gissing}}}}\\
\protect\hypertarget{header_section_text}{}{Volume 2, Chapter 3}

\protect\hypertarget{headernext}{}{\href{/wiki/The_Nether_World/Volume_2/Chapter_4}{Volume
2, Chpater 4}→}

\hypertarget{navigationNotes}{}

\hypertarget{ws-data}{}
\protect\hypertarget{ws-article-id}{}{1877027}\protect\hypertarget{ws-title}{}{\href{/wiki/The_Nether_World}{The
Nether World} --- \emph{Volume 2, Chapter
3}}\protect\hypertarget{ws-author}{}{George Gissing}

{\protect\hypertarget{42}{}{}}

{CHAPTER III.}

DIALOGUE AND COMMENT.

\textsc{``Will} it be late before he comes back?'' asked Sidney, his
smile of greeting shadowed with disappointment.

``Not later than half-past ten, he said.''

Sidney turned his face to the stairs. The homeward prospect was dreary
after that glimpse of the familiar room through the doorway. The breach
of habit discomposed him, and something more positive strengthened his
reluctance to be gone. It was not his custom to hang in hesitancy and
court chance by indirectness of speech; recognising and admitting his
motives, he said simply:

``I should like to stay a little, if you will let me,---if I shan't be
in your way?''

{\protect\hypertarget{43}{}{}}``Oh no! Please come in. I'm only
sewing.''

There were two round-backed wooden chairs in the room; one stood on each
side of the fireplace, and between them, beside the table, Jane always
had her place on a small chair of the ordinary comfortless kind. She
seated herself as usual, and Sidney took his familiar position, with the
vacant chair opposite. Snowdon and he were accustomed to smoke their
pipes whilst conversing, but this evening Sidney dispensed with tobacco.

It was very quiet here. On the floor below dwelt at present two sisters
who kept themselves alive (it is quite inaccurate to use any other
phrase in such instances) by doing all manner of skilful needlework;
they were middle-aged women, gentle-natured and so thoroughly subdued to
the hopelessness of their lot that scarcely ever could even their
footfall be heard as they went up and down stairs; their voices were
always sunk to a soft murmur. Just now no infant wailing came from the
Byasses' regions. Kirkwood {\protect\hypertarget{44}{}{}}enjoyed a sense
of restfulness, intenser, perhaps, for the momentary disappointment he
had encountered. He had no desire to talk; enough for a few minutes to
sit and watch Jane's hand as it moved backwards and forwards with the
needle.

``I went to see Pennyloaf as I came back from work,'' Jane said at
length, just looking up.

``Did you? Do things seem to be any better?''

``Not much, I'm afraid. Mr. Kirkwood, don't you think you might do
something? If you tried again with her husband?''

``The fact is,'' replied Sidney, ``I'm so afraid of doing more harm than
good.''

``You think{{------}}? But then perhaps that's just what \emph{I'm}
doing?''

Jane let her hand fall on the sewing and regarded him anxiously.

``No, no! I'm quite sure \emph{you} can't do harm. Pennyloaf can get
nothing but good from having you as a friend. She likes you; she misses
you when you happen not to have {\protect\hypertarget{45}{}{}}seen her
for a few days. I'm sorry to say it's quite a different thing with Bob
and me. We're friendly enough,---as friendly as ever,---but I haven't a
scrap of influence with him like you have with his wife. It was all very
well to get hold of him once, and try to make him understand, in a
half-joking way, that he wasn't behaving as well as he might. He didn't
take it amiss---just that once. But you can't think how difficult it is
for one man to begin preaching to another. The natural thought is: Mind
your own business. If I was the parson of the parish''{{------}}

He paused, and in the same instant their eyes met. The suggestion was
irresistible; Jane began to laugh merrily.

What sweet laughter it was! How unlike the shrill discord whereby the
ordinary work-girl expresses her foolish mirth! For years Sidney
Kirkwood had been unused to utter any sound of merriment; even his
smiling was done sadly. But of late he had grown conscious of the
element of joy in Jane's character, had accustomed himself to look for
{\protect\hypertarget{46}{}{}}its manifestations,---to observe the
brightening of her eyes which foretold a smile, the moving of her lips
which suggested inward laughter,---and he knew that herein, as in many
another matter, a profound sympathy was transforming him. Sorrow such as
he had suffered will leave its mark upon the countenance long after time
has done its kindly healing, and in Sidney's case there was more than
the mere personal affliction tending to confirm his life in sadness.
With the ripening of his intellect, he saw only more and more reason to
condemn and execrate those social disorders of which his own wretched
experience was but an illustration. From the first, his friendship with
Snowdon had exercised upon him a subduing influence; the old man was
stern enough in his criticism of society, but he did not belong to the
same school as John Hewett, and the sober authority of his character
made appeal to much in Sidney that had found no satisfaction amid the
uproar of Clerkenwell Green. For all that, Kirkwood could not become
other than himself; his vehemence was
{\protect\hypertarget{47}{}{}}moderated, but he never affected to be at
one with Snowdon in that grave enthusiasm of far-off hope which at times
made the old man's speech that of an exhorting prophet. Their natural
parts were reversed; the young eyes declared that they could see nothing
but an horizon of blackest cloud, whilst those enfeebled by years bore
ceaseless witness to the raying forth of dawn.

And so it was with a sensation of surprise that Sidney first became
aware of lightheartedness in the young girl who was a silent hearer of
so many lugubrious discussions. Ridiculous as it may sound,---as Sidney
felt it to be,---he almost resented this evidence of happiness; to him,
only just recovering from a shock which would leave its mark upon his
life to the end, his youth wronged by bitter necessities, forced into
brooding over problems of ill when nature would have bidden him enjoy,
it seemed for the moment a sign of shallowness that Jane could look and
speak cheerfully. This extreme of morbid feeling proved its own cure;
{\protect\hypertarget{48}{}{}}even in reflecting upon it, Sidney was
constrained to laugh contemptuously at himself. And therewith opened for
him a new world of thought. He began to study the girl. Of course he had
already occupied himself much with the peculiarities of her position,
but of Jane herself he knew very little; she was still, in his
imagination, the fearful and miserable child over whose shoulders he had
thrown his coat one bitter night; his impulse towards her was one of
compassion merely, justified now by what he heard of her mental
slowness, her bodily sufferings. It would take very long to analyse the
process whereby this mode of feeling was changed, until it became the
sense of ever-deepening sympathy which so possessed him this evening.
Little by little Jane's happiness justified itself to him, and in so
doing began subtly to modify his own temper. With wonder he recognised
that the poor little serf of former days had been meant by nature for
one of the most joyous among children. What must that heart have
suffered, so scorned and trampled {\protect\hypertarget{49}{}{}}upon!
But now that the days of misery were over, behold nature having its way
after all. If the thousands are never rescued from oppression, if they
perish abortive in their wretchedness, is that a reason for refusing to
rejoice with the one whom fate has blest? Sidney knew too much of Jane
by this time to judge her shallow-hearted. This instinct of gladness had
a very different significance from the animal vitality which prompted
the constant laughter of Bessie Byass; it was but one manifestation of a
moral force which made itself nobly felt in many another way. In himself
Sidney was experiencing its pure effects, and it was owing to his
conviction of Jane's power for good that he had made her acquainted with
Bob Hewett's wife. Snowdon warmly approved of this; the suggestion led
him to speak expressly of Jane, a thing he very seldom did, and to utter
a strong wish that she should begin to concern herself with the sorrows
she might in some measure relieve.

Sidney joined in the laughter he had
{\protect\hypertarget{50}{}{}}excited by picturing himself the parson of
the parish. But the topic under discussion was a serious one, and Jane
speedily recovered her gravity.

``Yes, I see how hard it is,'' she said.

``But it's a cruel thing for him to neglect poor Pennyloaf as he does.
She never gave him any cause.''

``Not knowingly, I quite believe,'' replied Kirkwood. ``But what a
miserable home it is!''

``Yes.'' Jane shook her head. ``She doesn't seem to know how to keep
things in order. She doesn't seem even to understand me when I try to
show her how it might be different.''

``There's the root of the trouble, Jane. What chance had Pennyloaf of
ever learning how to keep a decent home, and bring up her children
properly? How was \emph{she} brought up? The wonder is that there's so
much downright good in her; I feel the same wonder about people every
day. Suppose Pennyloaf behaved as badly as her mother
{\protect\hypertarget{51}{}{}}does, who on earth would have the right to
blame her? But we can't expect miracles; so long as she lives decently,
it's the most that can be looked for. iind there you are; that isn't
enough to keep a fellow like Bob Hewett in order. I doubt whether any
wife would manage it, but as for poor Pennyloaf{{------}}!''

``I shall speak to him myself,'' said Jane quietly. ``Do! There's much
more hope in that than in anything I could say. Bob isn't a bad fellow;
the worst thing I know of him is his conceit. He's good-looking, and
he's clever in all sorts of ways, and unfortunately he can't think of
anything but his own merits. Of course he'd no business to marry at all
whilst he was nothing but a boy.''

Jane plied her needle, musing.

``Do you know whether he ever goes to see his father?'' Sidney inquired
presently.

``No, I don't,'' Jane answered, looking at him, but immediately dropping
her eyes.

``If he doesn't I should think worse of him.
{\protect\hypertarget{52}{}{}}Nobody ever had a kinder father, and
there's many a reason why he should be careful to pay the debt he
owes.''

Jane waited a moment, then again raised her eyes to him. It seemed as
though she would ask a question, and Sidney's grave attentiveness
indicated a surmise of what she was about to say. But her thought
remained unuttered, and there was a prolongation of silence.

Of course they were both thinking of Clara. That name had never been
spoken by either of them in the other's presence, but as often as
conversation turned upon the Hewetts, it was impossible for them not to
supplement their spoken words by a silent colloquy of which Clara was
the subject. From her grandfather Jane knew that, to this day, nothing
had been heard of Hewett's daughter; what people said at the time of the
girl's disappearance she had learned fully enough from Clem Peckover,
who even yet found it pleasant to revive the scandal, and by
contemptuous comments revenge herself for Clara's haughty
{\protect\hypertarget{53}{}{}}usage in old days. Time had not impaired
Jane's vivid recollection of that Bank-holiday morning when she herself
was the first to make it known that Clara had gone away. Many a time
since then she had visited the street whither Snowdon led her,---had
turned aside from her wonted paths in the thought that it was not
impossible she might meet Clara, though whether with more hope or fear
of such a meeting she could not have said. When two years had gone by,
her grandfather one day led the talk to that subject; he was then
beginning to change in certain respects the tone he had hitherto used
with her, and to address her as one who had outgrown childhood. He
explained to her how it came about that Sidney could no longer be even
on terms of acquaintance with John Hewett. The conversation originated
in Jane's bringing the news that Hewett and his family had at length
left Mrs. Peckover's house. For two years things had gone miserably with
them, their only piece of good fortune being the death of the youngest
child. John was confirmed in a {\protect\hypertarget{54}{}{}}habit of
drinking. Not that he had become a brutal sot; sometimes for as much as
a month he would keep sober, and even when he gave way to temptation he
never behaved with violence to his wife and children. Still, the
character of his life had once more suffered a degradation, and he
possessed no friends who could be of the least use to him. Snowdon, for
some reason of his own, maintained a slight intercourse with the
Peckovers, and through them he endeavoured to establish an intimacy with
Hewett; but the project utterly failed. Probably on Kirkwood's account,
John met the old man's advances with something more than coldness.
Sternly he had forbidden his wife and the little ones to exchange a word
of any kind with Sidney, or with any friend of his. He appeared to
nourish incessantly the bitter resentment to which he gave expression
when Sidney and he last met.

There was no topic on which Sidney was more desirous of speaking with
Jane than this which now occupied both their minds. How far she
understood Clara's story, and his {\protect\hypertarget{55}{}{}}part in
it, he had no knowledge; for between Snowdon and himself there had long
been absolute silence on that matter. It was not improbable that Jane
had been instructed in the truth; he hoped she had not been left to
gather what she could from Clem Peckover's gossip. Yet the difficulty
with which he found himself beset, now that an obvious opportunity
offered for frank speech was so great that, after a few struggles, he
fell back on the reflection with which he was wont to soothe himself:
Jane was still so young, and the progress of time, by confirming her
knowledge of him, would make it all the simpler to explain the miserable
past. Had he, in fact, any right to relate this story, to seek her
sympathy in that direct way? It was one aspect of a very grave question
which occupied more and more of Sidney's thought.

With an effort, he turned the dialogue into quite a new direction, and
Jane, though a little absent for some minutes, seemed at length to
forget the abruptness of the change. Sidney had of late been resuming
his old {\protect\hypertarget{56}{}{}}interest in pencil-work; two or
three of his drawings hung on these walls, and he spoke of making new
sketches when he next went into the country. Years ago, one of his
favourite excursions,---of the longer ones which he now and then allowed
himself,---was to Danbury Hill, some five miles to the east of
Chelmsford, one of the few pieces of rising ground in Essex, famous for
its view over Maldon and the estuary of the Blackwater. Thither Snowdon
and Jane accompanied him during the last summer but one, and the former
found so much pleasure in the place that he took lodgings with certain
old friends of Sidney's, and gave his granddaughter a week of healthful
holiday. In the summer that followed, the lodgings were again taken for
a week, and this year the same expedition was in view. Sidney had as
good as promised that he would join his friends for the whole time of
their absence, and now he talked with Jane of memories and
anticipations. Neither was sensible how the quarters and the half-hours
went by in such chatting. Sidney {\protect\hypertarget{57}{}{}}abandoned
himself to the enjoyment of peace such as he had never known save in
this room, to a delicious restfulness such as was always inspired in him
by the girl's gentle voice, by her laughter, by her occasional quiet
movements. The same influence was affecting his whole life. To Jane he
owed the gradual transition from tumultuous politics and social
bitterness to the mood which could find pleasure as of old in nature and
art. This was his truer self, emancipated from the distorting effect of
the evil amid which he perforce lived. He was recovering somewhat of his
spontaneous boyhood; at the same time, reaching after a new ideal of
existence which only ripened manhood could appreciate.

Snowdon returned at eleven; it alarmed Sidney to find how late he had
allowed himself to remain, and he began shaping apologies. But the old
man had nothing but the familiar smile and friendly words.

``Haven't you given Mr. Kirkwood any supper?'' he asked of Jane, looking
at the table.

{\protect\hypertarget{58}{}{}}``I really forgot all about it,
grandfather,'' was the laughing reply.

Then Snowdon laughed, and Sidney joined in the merriment; but he would
not be persuaded to stay longer.
