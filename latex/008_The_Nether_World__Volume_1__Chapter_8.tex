\chapter{Pennyloaf Candy}

In the social classification of the nether world---a subject which so
eminently adapts itself to the sportive and gracefully picturesque mode
of treatment---it will be convenient to distinguish broadly, and with
reference to males alone, the two great sections of those who do, and
those who do not, wear collars. Each of these orders would, it is
obvious, offer much scope to an analyst delighting in subtle gradation.
Taking the collarless, how shrewdly might one discriminate between the
many kinds of neckcloth which our climate renders necessary as a
substitute for the nobler article of attire! The navvy, the scaffolder,
the costermonger, the cab-tout,---innumerable would be the varieties of
texture, of fold, of knot, observed in the ranks of
{\protect\hypertarget{167}{}{}}unskilled labour. And among those whose
higher station is indicated by the linen or paper symbol, what a gap
between the mechanic with collar attached to a flannel shirt and just
visible along the top of a black tie, and the shopman whose pride it is
to adorn himself with the very ugliest neck-encloser put in vogue by
aristocratic sanction! For such attractive disquisition I have,
unfortunately, no space; it must suffice that I indicate the two genera.
And I was led to do so in thinking of Bob Hewett.

Bob wore a collar. In the die-sinking establishment which employed him
there were, it is true, two men who belonged to the collarless; but
their business was down in the basement of the building, where they kept
up a furnace, worked huge stamping-machines, and so on. Bob's workshop
was upstairs, and the companions with whom he sat, without exception,
had something white and stiff round their necks; in fact, they were
every bit as respectable as Sidney Kirkwood, and such as he, who bent
over a {\protect\hypertarget{168}{}{}}jeweller's table. To John Hewett
it was no slight gratification that he had been able to apprentice his
son to a craft which permitted him always to wear a collar. I would not
imply that John thought of the matter in these terms, but his
reflections bore this significance. Bob was raised for ever above the
rank of those who depend merely upon their muscles, even as Clara was
saved from the dismal destiny of the women who can do nothing but sew.

There was, on the whole, some reason why John Hewett should feel pride
in his eldest son. Like Sidney Kirkwood, Bob had early shown a faculty
for draughtsmanship; when at school, he made decidedly clever
caricatures of such persons as displeased him, and he drew such
wonderful horses (on the racecourse or pulling cabs), such laughable
donkeys in costers' carts, such perfect dogs, that on several occasions
some friend had purchased with a veritable shilling a specimen of his
work. ``Put him to the die-sinking,'' said an acquaintance of the
family, himself so {\protect\hypertarget{169}{}{}}employed; ``he'll find
a use for this kind of thing some day.'' Die-sinking is not the craft it
once was; cheap methods, vulgarising here as everywhere, have diminished
the opportunities of capable men; but a fair living was promised the lad
if he stuck to his work, and at the age of nineteen he was already
earning his pound a week. Then he was clever in a good many other ways.
He had an ear for music, played (nothing else was within his reach) the
concertina, sang a lively song with uncommon melodiousness---a gift much
appreciated at the meetings of a certain Mutual Benefit Club, to which
his father had paid a weekly subscription, without fail, through all
adversities. In the regular departments of learning, Bob had never shown
any particular aptitude; he wrote and read decently, but his speech, as
you have had occasion for observing, was not marked by refinement, and
for books he had no liking. His father, unfortunately, had spoilt him,
just as he had spoilt Clara. Being of the nobly independent sex, between
{\protect\hypertarget{170}{}{}}fifteen and sixteen he practically freed
himself from parental control. The use he made of his liberty was not
altogether pleasing to John, but the time for restraint and training had
hopelessly gone by. The lad was selfish, that there was no denying; he
grudged the money demanded of him for his support; but in other matters
he always showed himself so easy-tempered, so disposed to a genial
understanding, that the great fault had to be blinked. Many failings
might have been forgiven him in consideration of the fact that he had
never yet drunk too much, and indeed cared little for liquor.

Men of talent, as you are aware, not seldom exhibit low tastes in their
choice of companionship. Bob was a case in point; he did not
sufficiently appreciate social distinctions. He, who wore a collar,
seemed to prefer associating with the collarless. There was Jack---more
properly ``Jeck''---Bartley, for instance, his bosom friend until they
began to cool in consequence of a common interest in Miss Peckover. Jack
never wore a collar {\protect\hypertarget{171}{}{}}in his life, not even
on Sundays, and was closely allied with all sorts of blackguards, who
somehow made a living on the outskirts of turf-land. And there was Eli
Snape, compared with whom Jack was a person of refinement and culture.
Eli dealt surreptitiously in dogs and rats, and the mere odour of him
was intolerable to ordinary nostrils; yet he was a species of hero in
Bob's regard, such invaluable information could he supply with regard to
``events'' in which young Hewett took a profound interest. Perhaps a
more serious aspect of Bob's disregard for social standing was revealed
in his relations with the other sex. Susceptible from his tender youth,
he showed no ambition in the bestowal of his amorous homage. At the age
of sixteen did he not declare his resolve to wed the daughter of old
Sally Budge, who went about selling watercress? and was there not a
desperate conflict at home before this project could be driven from his
head? It was but the first of many such instances. Had he been left to
his own devices, he would {\protect\hypertarget{172}{}{}}already, like
numbers of his coevals, have been supporting (or declining to support) a
wife and two or three children. At present, he was ``engaged'' to Clem
Peckover; that was an understood thing. His father did not approve it,
but this connection was undeniably better than those he had previously
declared or concealed. Bob, it seemed evident, was fated to make a
\emph{mésalliance};---a pity, seeing his parts and prospects. He might
have aspired to a wife who had scarcely any difficulty with her
\emph{h}'s; whose bringing-up enabled her to look with compassion on
girls who could not play the piano; who counted among her relatives not
one collarless individual.

Clem, as we have seen, had already found, or imagined, cause for
dissatisfaction with her betrothed. She was well enough acquainted with
Bob's repute, and her temper made it improbable, to say the least, that
the course of wooing would in this case run very smoothly. At present,
various little signs were beginning to convince her that she had a
rival, and the {\protect\hypertarget{173}{}{}}hints of her rejected
admirer, Jack Bartley, fixed her suspicions upon an acquaintance whom
she had hitherto regarded merely with contempt. This was Pennyloaf
Candy, formerly, with her parents, a lodger in Mrs. Peckover's house.
The family had been ousted some eighteen months ago on account of
failure to pay their rent and of the frequent intoxication of Mrs.
Candy. Pennyloaf's legal name was Penelope, which, being pronounced as a
trisyllable, transformed itself by further corruption into a sound at
all events conveying some meaning; applied in the first instance
jocosely, the title grew inseparable from her, and was the one she
herself always used. Her employment was the making of shirts for export;
she earned on an average tenpence a day, and frequently worked fifteen
hours between leaving and returning to her home. That Bob Hewett could
interest himself, with whatever motive, in a person of this description,
Miss Peckover at first declined to believe. A hint, however, was quite
enough to excite her jealous temperament; as proof
{\protect\hypertarget{174}{}{}}accumulated, cunning and ferocity wrought
in her for the devising of such a declaration of war as should speedily
scare Pennyloaf from the field. Jane Snowdon's removal had caused her no
little irritation; the hours of evening were heavy on her hands, and
this new emotion was not unwelcome as a temporary resource.

As he came home from work one Monday towards the end of April, Bob
encountered Pennyloaf; she had a bundle in her hands and was walking
hurriedly.

``Hallo! that you?'' he exclaimed, catching her by the arm. ``Where are
you going?''

``I can't stop now. I've got some things to put away, an' it's nearly
eight.''

``Come round to the Passage to-night. Be there at ten.''

``I can't give no promise. There's been such rows at `ome. You know
mother summonsed father this mornin'?''

``Yes, I've heard. All right! come if you can; I'll be there.''

Pennyloaf hastened on. She was a meagre,
{\protect\hypertarget{175}{}{}}hollow-eyed, bloodless girl of seventeen,
yet her features had a certain charm, that dolorous kind of prettiness
which is often enough seen in the London needle-slave. Her habitual look
was one of meaningless surprise; whatever she gazed upon seemed a source
of astonishment to her, and when she laughed, which was not very often,
her eyes grew wider than ever. Her attire was miserable, but there were
signs that she tried to keep it in order; the boots upon her feet were
sewn and patched into shapelessness; her limp straw hat had just
received a new binding.

By saying that she had things ``to put away,'' she meant that her
business was with the pawnbroker, who could not receive pledges after
eight o'clock. It wanted some ten minutes of the hour when she entered a
side-doorway, and, by an inner door, passed into one of a series of
compartments constructed before the pawnbroker's counter. She deposited
her bundle, and looked about for some one to attend to her. Two young
men were in sight, both transacting business;
{\protect\hypertarget{176}{}{}}one was conversing facetiously with a
customer on the subject of a pledge. Two or three gas-jets lighted the
interior of the shop, but the boxes were in shadow. There was a strong
musty odour; the gloom, the narrow compartments, the low tones of
conversation, suggested stealth and shame.

Pennyloaf waited with many signs of impatience, until one of the
assistants approached, a smartly attired youth, with black hair greased
into the discipline he deemed becoming, with an aquiline nose, a coarse
mouth, a large horse-shoe pin adorning his neck-tie, and rings on his
fingers. He caught hold of the packet and threw it open; it consisted of
a petticoat and the skirt of an old dress.

``Well, what is it?'' he asked, rubbing his tongue along his upper lip
before and after speaking.

``Three an six, please, sir.''

He rolled the things up again with a practised turn of the hand, and
said indifferently, glancing towards another box, ``Eighteen-pence.''

{\protect\hypertarget{177}{}{}}``Oh, sir, we had two shillin's on the
skirt not so long ago,'' pleaded Pennyloaf, with a subservient voice.
``Make it two shillin's,---please do, sir!''

The young man paid no attention; he was curling his moustache and
exchanging a smile of intelligence with his counter-companion with
respect to a piece of business the latter had in hand. Of a sudden he
turned and said sharply:

``Well, are you goin' to take it or not?''

Pennyloaf sighed and nodded.

``Got a 'apenny?'' he asked.

``No.''

He fetched a cloth, rolled the articles in it very tightly, and pinned
them up; then he made out ticket and duplicate, handling his pen with
facile flourish, and having blotted the little piece of card on a box of
sand (a custom which survives in this conservative profession), he threw
it to the customer. Lastly, he counted out one shilling and fivepence
halfpenny. The coins were sandy, greasy, and of scratched surface.

{\protect\hypertarget{178}{}{}}Pennyloaf sped homewards. She lived in
Shooter's Gardens, a picturesque locality which demolition and
rebuilding have of late transformed. It was a winding alley, with paving
raised a foot above the level of the street whence was its main
approach. To enter from the obscurer end, you descended a flight of
steps, under a low archway, in a court itself not easily discovered.
From without, only a glimpse of the Gardens was obtainable; the houses
curved out of sight after the first few yards, and left surmise to busy
itself with the characteristics of the hidden portion. A stranger bold
enough to explore would have discovered that the Gardens had a blind
offshoot, known simply as ``The Court.'' Needless to burden description
with further detail; the slum was like any other slum; filth,
rottenness, evil odours, possessed these dens of superfluous mankind and
made them gruesome to the peering imagination. The inhabitants of course
felt nothing of the sort; a room in Shooter's Gardens was the only kind
of home that {\protect\hypertarget{179}{}{}}most of them knew or
desired. The majority preferred it, on all grounds, to that offered them
in a block of model-lodgings not very far away; here was independence,
that is to say, the liberty to be as vile as they pleased. How they came
to love vileness, well, that is quite another matter, and shall not for
the present concern us.

Pennyloaf ran into the jaws of this black horror with the indifference
of habit; it had never occurred to her that the Gardens were fearful in
the night's gloom, nor even that better lighting would have been a
convenience. Did it happen that she awoke from her first sleep with the
ring of ghastly shrieking in her ears, that was an incident of too
common occurrence to cause her more than a brief curiosity; she could
wait till the morning to hear who had half killed whom. Four days ago it
was her own mother's turn to be pounded into insensibility; her father
(a journeyman baker, often working nineteen hours out of the
twenty-four, which probably did not improve his temper), maddened by
{\protect\hypertarget{180}{}{}}his wife's persistent drunkenness, was
stopped just on the safe side of murder. To the amazement and
indignation of the Gardens, Mrs. Candy prosecuted her sovereign lord;
the case had been heard to-day, and Candy had been cast in a fine. The
money was paid, and the baker went his way, remarking that his family
were to ``expect him back when they saw him.'' Mrs. Candy, on her
return, was hooted through all the length of the Gardens, a
demonstration of public feeling probably rather of base than of worthy
significance.

As Pennyloaf drew near to the house, a wild, discordant voice suddenly
broke forth somewhere in the darkness, singing in a high key, ``All ye
works of the Lord, bless ye the Lord, praise Him and magnify Him for
ever!'' It was Mad Jack, who had his dwelling in the Court, and at all
hours was wont to practise the psalmody which made him notorious
throughout Clerkenwell. A burst of laughter followed from a group of men
and boys gathered near the archway. Unheeding,
{\protect\hypertarget{181}{}{}}the girl passed in at an open door and
felt her way up a staircase; the air was noisome, notwithstanding a
fierce draught which swept down the stairs. She entered a room lighted
by a small metal lamp hanging on the wall---a precaution of Pennyloaf's
own contrivance. There was no bed, but one mattress lay with a few rags
of bed-clothing spread upon it, and two others were rolled up in a
corner. This chamber accommodated, under ordinary circumstances, four
persons: Mr. and Mrs. Candy, Pennyloaf, and a son named Stephen, whose
years were eighteen. (Stephen pursued the occupation of a potman; his
hours were from eight in the morning to midnight on week-days, and on
Sunday the time during which a public-house is permitted to be open;
once a month he was allowed freedom after six o'clock.) Against the
window was hung an old shawl pierced with many rents. By the fire sat
Mrs. Candy; she leaned forward, her head, which was bound in linen
swathes, resting upon her hands.

``What have you got?'' she asked, in {\protect\hypertarget{182}{}{}}the
thick voice of a drunkard, without moving.

``Eighteenpence; it's all they'd give me.''

The woman cursed in her throat, but exhibited no anger with Pennyloaf.

``Go an' get some tea an' milk,'' she said, after a pause. ``There is
sugar. An' bring seven o' coals; there's only a dust.''

She pointed to a deal box which stood by the hearth. Pennyloaf went out
again.

Over the fireplace, the stained wall bore certain singular ornaments.
These were five coloured cards, such as are signed by one who takes a
pledge of total abstinence; each presented the signature, ``Maria
Candy,'' and it was noticeable that at each progressive date the
handwriting had become more unsteady. Yes, five times had Maria Candy
``promised, with the help of God, to abstain,'' \&c. \&c.; each time she
was in earnest. But it appeared that the help of God availed little
against the views of one Mrs. Green, who kept the beer-shop in Rosoman
Street, once Mrs. Peckover's, and who could on no account
{\protect\hypertarget{183}{}{}}afford to lose so good a customer. For
many years that house, licensed for the sale of non-spirituous liquors,
had been working Mrs. Candy's ruin; not a particle of her frame but was
vitiated by the drugs retailed there under the approving smile of
civilisation. Spirits would have been harmless in comparison. The
advantage of Mrs. Green's ale was that the very first half-pint gave
conscience its bemuddling sop; for a penny you forgot all the cares of
existence; for threepence you became a yelling maniac.

Poor, poor creature! She was sober tonight, sitting over the fire with
her face battered into shapelessness; and now that her fury had had its
way, she bitterly repented invoking the help of the law against her
husband. What use? what use? Perhaps he had now abandoned her for good,
and it was certain that the fear of him was the only thing that ever
checked her on the ruinous road she would so willingly have quitted. But
for the harm to himself, the only pity was he had not taken her life
outright. She knew all the {\protect\hypertarget{184}{}{}}hatefulness of
her existence; she knew also that only the grave would rescue her from
it. The struggle was too unequal between Mrs. Candy with her appeal to
Providence, and Mrs. Green with the forces of civilisation at her back.

Pennyloaf speedily returned with a hap'orth of milk, a pennyworth of
tea, and seven pounds (also price one penny) of coals in an apron. It
was very seldom indeed that the Candys had more of anything in their
room than would last them for the current day. There being no kettle,
water was put on to boil in a tin saucepan; the tea was made in a jug.
Pennyloaf had always been a good girl to her mother; she tended her as
well as she could to-night; but there was no word of affection from
either. Kindly speech was stifled by the atmosphere of Shooter's
Gardens.

Having drunk her tea, Mrs. Candy lay down, as she was, on the already
extended mattress and drew the ragged coverings about her. In half an
hour she slept.

{\protect\hypertarget{185}{}{}}Pennyloaf then put on her hat and jacket
again and left the house. She walked away from the denser regions of
Clerkenwell, came to Sadler's Wells Theatre (gloomy in its profitless
recollection of the last worthy manager that London knew), and there
turned into Myddelton Passage. It is a narrow paved walk between brick
walls seven feet high; on the one hand lies the New River Head, on the
other are small gardens behind Myddelton Square. The branches of a few
trees hang over; there are doors, seemingly never opened, belonging one
to each garden; a couple of gas-lamps shed feeble light. Pennyloaf paced
the length of the Passage several times, meeting no one. Then a
policeman came along with echoing tread, and eyed her suspiciously. She
had to wait more than a quarter of an hour before Bob Hewett made his
appearance. Greeting her with a nod and a laugh, he took up a leaning
position against the wall, and began to put questions concerning the
state of things at her home.

{\protect\hypertarget{186}{}{}}``And what'll your mother do if the old
man don't give her nothing to live on?'' he inquired, when he had
listened good-naturedly to the recital of domestic difficulties.

``Don't know,'' replied the girl, shaking her head, the habitual
surprise of her countenance becoming a blank interrogation of destiny.
Bob kept kicking the wall, first with one heel, then with the other. He
whistled a few bars of the last song he had learnt at the music-hall.

``Say, Penny,'' he remarked at length, with something of shamefacedness,
``there's a namesake of mine here as I shan't miss, if you can do any
good with it.''

He held a shilling towards her under his hand. Pennyloaf turned away,
casting down her eyes and looking troubled.

``We can get on for a bit,'' she said indistinctly. Bob returned the
coin to his pocket. He whistled again for a moment, then asked abruptly:

{\protect\hypertarget{187}{}{}}``Say! have you seen Clem again?''

``No,'' replied the girl, examining him with sudden acuteness. ``What
about her?''

``Nothing much. She's got her back up a bit, that's all.''

``About me?'' Pennyloaf asked anxiously.

Bob nodded. As he was making some further remarks on the subject, a
man's figure appeared at a little distance, and almost immediately
withdrew again round a winding of the Passage. A moment after there
sounded from that direction a shrill whistle. Bob and the girl regarded
each other.

``Who was that?'' said the former suspiciously. ``I half believe it was
Jeck Bartley. If Jeck is up to any of his larks, I'll make him remember
it. You wait here a minute!''

He walked at a sharp pace towards the suspected quarter. Scarcely had he
gone half a dozen yards, when there came running from the other end of
the Passage a girl whom Pennyloaf at once recognised. It was Clem
Peckover; with some friend's assistance she
{\protect\hypertarget{188}{}{}}had evidently tracked the couple and was
now springing out of ambush. She rushed upon Pennyloaf, who for very
alarm could not flee, and attacked her with clenched fists. A scream of
terror and pain caused Bob to turn and run back. Pennyloaf could not
even ward off the blows that descended upon her head; she was pinned
against the wall, her hat was torn away, her hair began to fly in
disorder. But Bob effected a speedy rescue. He gripped Clem's muscular
arms, and forced them behind her back as if he meant to dismember her.
Even then it was with no slight effort that he restrained the girl's
fury.

``You run off 'ome!'' he shouted to Pennyloaf. ``If she tries this on
again, I'll murder her!''

Pennyloaf's hysterical cries and the frantic invectives of her assailant
made the Passage ring. Again Bob roared to the former to be off, and was
at length obeyed. When Pennyloaf was out of sight he released Clem. Her
twisted arms caused her such pain that she threw herself against the
wall, mingling {\protect\hypertarget{189}{}{}}maledictions with groans.
Bob burst into scornful laughter.

Clem went home vowing vengeance. In the nether world this trifling
dissension might have been expected to bear its crop of violent language
and straightway pass into oblivion; but Miss Peckover's malevolence was
of no common stamp, and the scene of to-night originated a feud which in
the end concerned many more people than those immediately interested.
