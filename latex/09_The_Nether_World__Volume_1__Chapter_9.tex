\hypertarget{headerContainer}{}
\hypertarget{navigationHeader}{}
\protect\hypertarget{headerprevious}{}{←\href{/wiki/The_Nether_World/Volume_1/Chapter_8}{Chapter
8}}

\textbf{\protect\hypertarget{header_title_text}{}{\href{/w/index.php?title=The_Nether_World/Volume_1\&action=edit\&redlink=1}{The
Nether World/Volume 1}}} \emph{by
\href{/wiki/Author:George_Gissing}{\protect\hypertarget{header_author_text}{}{{George
Gissing}}}}\\
\protect\hypertarget{header_section_text}{}{Chapter 9}

\protect\hypertarget{headernext}{}{\href{/wiki/The_Nether_World/Volume_1/Chapter_10}{Chapter
10}→}

\hypertarget{navigationNotes}{}

\hypertarget{ws-data}{}
\protect\hypertarget{ws-article-id}{}{1766234}\protect\hypertarget{ws-title}{}{\href{/w/index.php?title=The_Nether_World/Volume_1\&action=edit\&redlink=1}{The
Nether World/Volume 1} --- \emph{Chapter
9}}\protect\hypertarget{ws-author}{}{George Gissing}

{\protect\hypertarget{190}{}{}}

{CHAPTER IX.}

PATHOLOGICAL.

\textsc{Through} the day and through the evening, Clara Hewett had her
place behind Mrs. Tubbs's bar. For daylight wear, the dress which had
formerly been her best was deemed sufficient; it was simple, but not
badly made and became her figure. Her evening attire was provided by
Mrs. Tubbs who recouped herself by withholding the promised wages for a
certain number of weeks. When Clara had surveyed this garment in the bar
mirror, she turned away contemptuously; the material was cheap, the mode
vulgar. It must be borne with for the present, like other indignities
which she found to be inseparable from her position. As soon as her
employer's claim was satisfied, and the weekly five shillings began to
be paid, Clara remembered {\protect\hypertarget{191}{}{}}the promise she
had volunteered to her father. But John was once more at work; for the
present there really seemed no need to give him any of her money, and
she herself, on the other hand, lacked so many things. This dress
plainly would not be suitable for the better kind of engagement she had
in view; it behoved her first of all to have one made in accordance with
her own taste. A mantle too, a silk umbrella, gloves{{------}} It would
be unjust to herself to share her scanty earnings with those at home.

Yes; but you must try to understand this girl of the people, with her
unfortunate endowment of brains and defect of tenderness. That smile of
hers, which touched and fascinated and made thoughtful, had of course a
significance discoverable by study of her life and character. It was no
mere affectation; she was not conscious, in smiling, of the expression
upon her face. Moreover, there was justice in the sense of wrong
discernible upon her features when the very self looked forth from them.
All through his life, John {\protect\hypertarget{192}{}{}}Hewett had
suffered from the same impulse of revolt; less sensitively constructed
than his daughter, uncalculating, inarticulate, he fumed and fretted
away his energies in a conflict with forces ludicrously personified. In
the matter of his second marriage he was seen at his best, generously
defiant of social cruelties; but self-knowledge was denied him, and
circumstances condemned his life to futility. Clara inherited his
temperament; transferred to her more complex nature, it gained in
subtlety and in power of self-direction, but lost in its nobler
elements. Her mother was a capable and ambitious woman, one in whom
active characteristics were more prominent than the emotional. With such
parents, every probability told against her patient acceptance of a lot
which allowed her faculties no scope. And the circumstances of her
childhood were such as added a peculiar bitterness to the trials waiting
upon her maturity.

Clara, you remember, had reached her eleventh year when her father's
brother died {\protect\hypertarget{193}{}{}}and left the legacy of which
came so little profit. That was in 1873. State education had recently
made a show of establishing itself, and in the Hewetts' world much
argument was going on with reference to the new Board-schools, and their
advantages or disadvantages when compared with those in which
working-folk's children had hitherto been taught. Clara went to a
Church-school, and the expense was greater than the new system rendered
necessary. Her father's principles naturally favoured education on an
independent basis, but a prejudice then (and still) common among
workpeople of decent habits made him hesitate about sending his girl to
sit side by side with the children of the street. And he was confirmed
by Clara's own view of the matter. She spoke with much contempt of
Board-schools, and gave it to be understood that her religious
convictions would not suffer her to be taught by those who made light of
orthodoxy. This attitude was intelligible enough in a child of sharp wit
and abundant self-esteem.
{\protect\hypertarget{194}{}{}}\{hwe\textbar{}withstanding\textbar{}Notwithstanding\}\}
her father's indifferentism, little Clara perceived that a regard for
religion gave her a certain distinction at home, and elsewhere placed
her apart from ``common girls.'' She was subject, also, to special
influences: on the one hand, from her favourite teacher, Miss Harrop; on
the other, from a school-friend, Grace Rudd.

Miss Harrop was a good, warm-hearted woman of about thirty, one of those
unhappy persons who are made for domestic life, but condemned by fate to
school-celibacy. Lonely and impulsive, she drew to herself the most
interesting girl in her classes, and, with complete indiscretion, made a
familiar, a pet, a prodigy of one whose especial need was discipline. By
her confidences and her flatteries she set Clara aflame with spiritual
pride. Ceaselessly she excited her to ambition, remarked on her gifts,
made dazzling forecast of her future. Clara was to be a teacher first of
all, but only that she might be introduced to the notice of people who
would aid her to better things. And the
{\protect\hypertarget{195}{}{}}child came to regard this as the course
inevitably before her. Had she not already received school-prizes, among
them a much-gilded little volume ``for religious knowledge''? Did she
not win universal applause when she recited a piece of verse on
prize-day,---Miss Harrop (disastrous kindness!) even saying that the
delivery reminded her of Mrs. {{------}}, the celebrated actress!

Grace Rudd was busy in the same fatal work. Four years older than Clara,
weakly pretty, sentimental, conceited, she had a fancy for patronising
the clever child, to the end that she might receive homage in return.
Poor Grace! She left school, spent a year or two at home with parents as
foolish as herself, and---disappeared. Prior to that. Miss Harrop had
also passed out of Clara's ken, driven by restlessness to try another
school, away from London.

These losses appeared to affect Clara unfavourably. She began to neglect
her books, to be insubordinate, to exhibit arrogance, which brought down
upon her plenty of {\protect\hypertarget{196}{}{}}wholesome reproof. Her
father was not without a share in the responsibility for it all.
Entering upon his four hundred pounds, one of the first things John did
was to hire a piano, that his child might be taught to play. Pity that
Sidney Kirkwood could not then cry with effective emphasis, ``We are the
working classes! we are the lower orders!'' It was exactly what Hewett
would not bring himself to understand. What! His Clara must be robbed of
chances just because her birth was not that of a young lady? Nay, by all
the unintelligible Powers, she should enjoy every help that he could
possibly afford her. Bless her bright face and her clever tongue! Yes,
it was now a settled thing that she should be trained for a
school-teacher. An atmosphere of refinement must be made for her; she
must be better dressed, more delicately fed.

The bitter injustice of it! In the outcome you are already instructed.
Long before Clara was anything like ready to enter upon a teacher's
career, her father's ill-luck once {\protect\hypertarget{197}{}{}}more
darkened over the home. Clara had made no progress since Miss Harrop's
day. The authorities directing her school might have come forward with
aid of some kind, had it appeared to them that the girl would repay such
trouble; but they had their forebodings about her. Whenever she chose,
she could learn in five minutes what another girl could scarcely commit
to memory in twenty; but it was obviously for the sake of display. The
teachers disliked her; among the pupils she had no friends. So at length
there came the farewell to school and the beginning of practical life,
which took the shape of learning to stamp crests and addresses on
note-paper. There was hope that before long Clara might earn thirteen
shillings a week.

The bitter injustice of it! Clara was seventeen now, and understood the
folly of which she had been guilty a few years ago, but at the same time
she felt in her inmost heart the tyranny of a world which takes revenge
for errors that are inevitable, which misleads a
{\protect\hypertarget{198}{}{}}helpless child and then condemns it for
being found astray. She could judge herself, yes, better than Sidney
Kirkwood could judge her. She knew her defects, knew her vices, and a
feud with fate caused her to accept them defiantly. Many a time had she
sobbed out to herself, ``I wish I could neither read nor write! I wish I
had never been told that there is anything better than to work with
one's hands and earn daily bread!'' But she could not renounce the
claims that Nature had planted in her, that her guardians had fostered.
The better she understood how difficult was every way of advancement,
the more fiercely resolute was she to conquer satisfactions which seemed
beyond the sphere of her destiny.

Of late she had thought much of her childish successes in reciting
poetry. It was not often that she visited a theatre (her father had
always refused to let her go with any one save himself or Sidney), but
on the rare occasions when her wish was gratified, she had watched each
actress with devouring interest, with burning envy, and had said to
herself, "Couldn't {\protect\hypertarget{199}{}{}}I soon learn to do as
well as that? Can't I see where it might be made more lifelike? Why
should it be impossible for me to go on the stage?" In passing a
shop-window where photographs were exposed, she looked for those of
actresses, and gazed at them with terrible intensity. "I am as
good-looking as she is. Why shouldn't \emph{my} portrait be seen some
day in the windows?" And then her heart throbbed, smitten with
passionate desire. As she walked on, there was a turbid gloom about her,
and in her ears the echoing of a dread temptation. Of all this she spoke
to nobody.

For she had no friends. A couple of years ago something like an intimacy
had sprung up between her and Bessie Jones (since married and become
Bessie Byass), seemingly on the principle of contrast in association.
Bessie, like most London workgirls, was fond of the theatre, and her
talk helped to nourish the ambition which was secretly developing in
Clara. But the two could not long harmonise, Bessie, just after her
marriage, ventured to speak with friendly reproof of Clara's
{\protect\hypertarget{200}{}{}}behaviour to Sidney Kirkwood. Clara was
not disposed to admit freedoms of that kind; she half gave it to be
understood that, though others might be easily satisfied, she had views
of her own on such subjects. Thereafter, Mrs. Byass grew decidedly cool.
The other girls with whom Clara had formal intercourse showed no desire
to win her confidence; they were kept aloof by her reticent civility.

As for Sidney himself, it was not without reason that he had seen
encouragement in the girl's first reply to his advances. At sixteen,
Clara found it agreeable to have her good graces sought by the one man
in whom she recognised superiority of mind and purpose. Of all the
unbetrothed girls she knew, not one but would have felt flattered had
Kirkwood thus distinguished her. Nothing common adhered to his
demeanour, to his character; he had the look of one who will hold his
own in life; his word had the ring of truth. Of his generosity she had
innumerable proofs, and it contrasted nobly with the selfishness of
young men as she knew them; she
{\protect\hypertarget{201}{}{}}appreciated it all the more because her
own frequent desire to be unselfish was so fruitless. Of awakening
tenderness towards him she knew nothing, but she gave him smiles and
words which might mean little or much, just for the pleasure of
completing a conquest. Nor did she, in truth, then regard it as
impossible that, sooner or later, she might become his wife. If she
\emph{must} marry a workman, assuredly it should be Sidney. He thought
so highly of her, he understood things in her to which the ordinary
artisan would have been dead; he had little delicacies of homage which
gave her keen pleasure. And yet,---well, time enough!

Time went very quickly, and changed both herself and Sidney in ways she
could not foresee. It was true, all he said to her in anger that night
by the prison wall---true and deserved every word of it. Even in
acknowledging that, she hardened herself against him implacably. Since
he chose to take this tone with her, to throw aside all his graceful
blindness to her faults, he had only himself to
{\protect\hypertarget{202}{}{}}blame if she considered everything at an
end between them. She tried to believe herself glad this had happened;
it relieved her from an embarrassment, and made her absolutely free to
pursue the ambitions which now gave her no rest. For all that, she could
not dismiss Sidney from her mind; indeed, throughout the week that
followed their parting, she thought of him more persistently than for
many months. That he would before long seek pardon for his rudeness she
felt certain, she felt also that such submission would gratify her in a
high degree. But the weeks were passing and no letter came; in vain she
glanced from the window of the bar at the faces which moved by. Even on
Sunday, when she went home for an hour or two, she neither saw nor heard
of Kirkwood. She could not bring herself to ask a question.

Under any circumstances Clara would ill have borne a suspense that
irritated her pride, and at present she lived amid conditions so
repugnant, that her nerves were ceaselessly strung almost beyond
endurance. Before {\protect\hypertarget{203}{}{}}entering upon this
engagement she had formed but an imperfect notion of what would be
demanded of her. To begin with, Mrs. Tubbs belonged to the order of
women who are by nature slave-drivers; though it was her interest to
secure Clara for a permanency, she began by exacting from the girl as
much labour as could possibly be included in their agreement. The hours
were insufferably long; by nine o'clock each evening Clara was so
out-worn that with difficulty she remained standing, yet not until
midnight was she released. The unchanging odours of the place sickened
her, made her head ache, and robbed her of all appetite. Many of the
duties were menial, and to perform them fevered her with indignation.
Then the mere waiting upon such men as formed the majority of the
customers, vulgarly familiar, when not insolent, in their speech to her,
was hateful beyond anything she had conceived. Had there been no one to
face but her father, she would have returned home and resumed her old
occupation at the end of the first fortnight, so extreme
{\protect\hypertarget{204}{}{}}was her suffering in mind and body; but
rather than give Sidney Kirkwood such a triumph, she would work on, and
breathe no word of what she underwent. Even in her anger against him,
the knowledge of his forgiving disposition, of the sincerity of his
love, was an unavowed support. She knew he could not utterly desert her;
when some day he sought a reconciliation, the renewal of conflict
between his pride and her own would, she felt, supply her with new
courage.

Early one Saturday afternoon she was standing by the windows, partly
from heavy idleness of thought, partly on the chance that Kirkwood might
go by, when a young well-dressed man, who happened to be passing at a
slow walk, turned his head and looked at her. He went on, but in a few
moments Clara, who had moved back into the shop, saw him enter and come
forwards. He took a seat at the counter and ordered a luncheon. Clara
waited upon him with her customary cold reserve, and he made no remark
until she returned him change out of the coin he offered.

{\protect\hypertarget{205}{}{}}Then he said with an apologetic smile:

``We are old acquaintances, Miss Hewett, but I'm afraid you've forgotten
me.''

Clara regarded him in astonishment. His age seemed to be something short
of thirty; he had a long, grave, intelligent face, smiled enigmatically,
spoke in a rather slow voice. His silk hat, sober necktie drawn through
a gold ring, and dark morning-coat, made it probable that he was ``in
the City.''

``We used to know each other very well about five years ago,'' he
pursued, pocketing his change carelessly. ``Don't you remember a Mr.
Scawthorne, who used to be a lodger with some friends of yours called
Rudd?''

On the instant memory revived in Clara. In her school-days she often
spent a Sunday afternoon with Grace Rudd, and this Mr. Scawthorne was
generally at the tea-table. Mr. and Mrs. Rudd made much of him, said
that he held a most important post in a lawyer's office, doubtless had
private designs concerning him and their daughter. Thus aided, she even
recognised his features.

{\protect\hypertarget{206}{}{}}``And you knew me again after all this
time?''

``Your's isn't an easy face to forget,'' replied Mr. Scawthorne, with
the subdued polite smile which naturally accompanied his tone of
unemotional intimacy. "To tell you the whole truth, however, I happened
to hear news of you a few days ago. I met Grace Rudd; she told me you
were here. Some old friend had told \emph{her}."

Grace's name awoke keen interest in Clara. She was startled to hear it,
and did not venture to make the inquiry her mind at once suggested. Mr.
Scawthorne observed her for an instant, then proceeded to satisfy her
curiosity. Grace Rudd was on the stage; she had been acting in
provincial theatres under the name of Miss Danvers, and was now waiting
for a promised engagement at a minor London theatre.

``Do you often go to the theatre?'' he added carelessly. "I have a great
many acquaintances connected with the stage in one way or another. If
you would like, I should be {\protect\hypertarget{207}{}{}}very glad to
send you tickets now and then, I always have more given me than I can
well use."

Clara thanked him rather coldly, and said that she was very seldom free
in the evening. Thereupon Mr. Scawthorne again smiled, raised his hat,
and departed.

Possibly he had some consciousness of the effect of his words, but it
needed a subtler insight, a finer imagination than his, to interpret the
pale, beautiful, harassed face which studiously avoided looking towards
him as he paused before stepping out on to the pavement. The rest of the
evening, the hours of night that followed, passed for Clara in hot
tumult of heart and brain. The news of Grace Rudd had flashed upon her
as revelation of a clear possibility where hitherto she had seen only
mocking phantoms of futile desire. Grace was an actress; no matter by
what course, to this she had attained. This man, Scawthorne, spoke of
the theatrical life as one to whom all its details were familiar;
acquaintance with him of a sudden bridged over the
{\protect\hypertarget{208}{}{}}chasm which had seemed impassable. Would
he come again to see her? Had her involuntary reserve put an end to any
interest he might have felt in her? Of him personally she thought not at
all; she could not have recalled his features; he was a mere
abstraction, the representative of a wild hope which his conversation
had inspired.

From that day the character of her suffering was altered; it became less
womanly, it defied weakness and grew to a fever of fierce, unscrupulous
rebellion. Whenever she thought of Sidney Kirkwood, the injury he was
inflicting upon her pride rankled into bitter resentment, unsoftened by
the despairing thought of self- subdual which had at times visited her
sick weariness. She bore her degradations with the sullen indifference
of one who is supported by the hope of a future revenge. The disease
inherent in her being, that deadly outcome of social tyranny which
perverts the generous elements of youth into mere seeds of destruction,
developed day by day, blighting her heart, corrupting her moral sense,
even {\protect\hypertarget{209}{}{}}setting marks of evil upon the
beauty of her countenance. A passionate desire of self-assertion
familiarised her with projects, with ideas, which formerly she had
glanced at only to dismiss as ignoble. In proportion as her bodily
health failed, the worst possibilities of her character came into
prominence. Like a creature that is beset by unrelenting forces, she
summoned and surveyed all the crafty faculties lurking in the dark
places of her nature; theoretically she had now accepted every debasing
compact by which a woman can spite herself on the world's injustice.
Self-assertion; to be no longer an unregarded atom in the mass of those
who are born only to labour for others; to find play for the strength
and the passion which, by no choice of her own, distinguished her from
the tame slave. Sometimes in the silence of night she suffered from a
dreadful need of crying aloud, of uttering her anguish in a scream like
that of insanity. She stifled it only by crushing her face into the
pillow until the hysterical fit had passed, and she lay like one dead.

{\protect\hypertarget{210}{}{}}A fortnight after his first visit Mr.
Scawthorne again presented himself, polite, smiling, perhaps rather more
familiar. He stayed talking for nearly an hour, chiefly of the theatre.
Casually he mentioned that Grace Rudd had got her engagement---only a
little part in a farce. Suppose Clara came to see her play some evening?
Might he take her? He could at any time have places in the dress-circle.

Clara accepted the invitation. She did so without consulting Mrs. Tubbs,
and when it became necessary to ask for the evening's freedom,
difficulties were made. ``Very well,'' said Clara, in a tone she had
never yet used to her employer, ``then I shall leave you.'' She spoke
without a moment's reflection; something independent of her will seemed
to direct her in speech and act. Mrs. Tubbs yielded.

Clara had not yet been able to obtain the dress she wished for. Her
savings, however, were sufficient for the purchase of a few accessories,
which made her, she considered, not
{\protect\hypertarget{211}{}{}}unpresentable. Scawthorne was to have a
cab waiting for her at a little distance from the luncheon-bar. It was
now June, and at the hour of their meeting still broad daylight, but
Clara cared nothing for the chance that acquaintances might see her;
nay, she had a reckless desire that Sidney Kirkwood might pass just at
this moment. She noticed no one whom she knew, however; but just as the
cab was turning into Pentonville Road, Scawthorne drew her attention to
a person on the pavement.

``You see that old fellow,'' he said. " Would you believe that he is
very wealthy?"

Clara had just time to perceive an old man with white hair, dressed as a
mechanic.

``But I know him,'' she replied. ``His name's Snowdon.''

``So it is. How do you come to know him?'' Scawthorne inquired with
interest. She explained.

``Better not say anything about it,'' remarked her companion. "He's an
eccentric chap. I happen to know his affairs in the
{\protect\hypertarget{212}{}{}}way of business. I oughtn't to have told
secrets, but I can trust you."

A gentle emphasis on the last word, and a smile of more than usual
intimacy. But his manner was, and remained through the evening,
respectful almost to exaggeration. Clara seemed scarcely conscious of
his presence, save in the act of listening to what he said. She never
met his look, never smiled. From entering the theatre to leaving it, she
had a high flush on her face. Impossible to recognise her friend in the
actress whom Scawthorne indicated; features and voice were wholly
strange to her. In the intervals, Scawthorne spoke of the difficulties
that beset an actress's career at its beginning.

``I suppose you never thought of trying it?'' he asked. ``Yet I fancy
you might do well, if only you could have a few months' training, just
to start you. Of course it all depends on knowing how to go about it. A
little money would be necessary,---not much.''

{\protect\hypertarget{213}{}{}}Clara made no reply. On the way home she
was mute. Scawthorne took leave of her in Upper Street, and promised to
look in again before long\ldots{}.

Under the heat of these summer days, in the reeking atmosphere of the
bar, Clara panted fever-stricken. The weeks went on; what strength
supported her from the Monday morning to the Saturday midnight she could
not tell. Acting and refraining, speaking and holding silence, these
things were no longer the consequences of her own volition. She wished
to break free from her slavery, but had not the force to do so;
something held her voice as often as she was about to tell Mrs. Tubbs
that this week would be the last. Her body wasted so that all the
garments she wore were loose upon her. The only mental process of which
she was capable was reviewing the misery of days just past and
anticipating that of the days to come. Her only feelings were infinite
self-pity and a dull smouldering hatred of all others in the world. A
doctor would have bidden her take {\protect\hypertarget{214}{}{}}to bed,
as one in danger of grave illness. She bore through it without change in
her habits, and in time the strange lethargy passed.

Scawthome came to the bar frequently. He remarked often on her look of
suffering, and urged a holiday. At length, near the end of July, he
invited her to go up the river with him on the coming Bank-holiday.
Clara consented, though aware that her presence would be more than ever
necessary at the bar on the day of much drinking. Later in the evening
she addressed her demand to Mrs. Tubbs. It was refused.

Without a word of anger, Clara went upstairs, prepared herself for
walking, and set forth among the by-ways of Islington. In half an hour
she had found a cheap bedroom, for which she paid a week's rent in
advance. She purchased a few articles of food and carried them to her
lodging, then lay down in the darkness.
