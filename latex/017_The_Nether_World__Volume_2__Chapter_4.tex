\chapter{Clem Makes a Disclosure}

\textsc{When} Miss Peckover suggested to her affianced that their
wedding might as well take place at the registry-office, seeing that
there would then be no need to go to expense in the article of costume,
Mr. Snowdon readily assented; at the same time it gave him new matter
for speculation. Clem was not exactly the kind of girl to relinquish
without good reason that public ceremony which is the dearest of all
possible ceremonies to women least capable of reverencing its
significance. Every day made it more obvious that the Peckovers desired
to keep this marriage a secret until it was accomplished. In one way
only could Joseph James account for the mystery running through the
whole affair; it must be {}that Miss Peckover had indiscretions to
conceal, certain points in her histoiy with which she feared lest her
bridegroom should be made acquainted by envious neighbours. The thought
had no effect upon Mr. Snowdon save to excite his mirth; his attitude
with regard to such possibilities was that of a philosopher. The views
with which he was entering upon this alliance were so beautifully simple
that he really did not find it worth while to puzzle further as soon as
the plausible solution of his difficulties had presented itself Should
he hereafter discover that something unforeseen perturbed the smooth
flow of life to which he looked forward, nothing could be easier than
his remedy; the world is wide, and a cosmopolitan does not attach undue
importance to a marriage contracted in one of its somewhat numerous
parishes. In any case he would have found the temporary harbour of
refuge which stress of weather had made necessary. He surrendered
himself to the pleasant tickling of his vanity which was an immediate
result of the adventure. For, {}whatever Clem might be hiding, it seemed
to him beyond doubt that she was genuinely attracted by his personal
qualities. Her demonstrations were not extravagant, but in one
noteworthy respect she seemed to give evidence of a sensibility so
little in keeping with her general character that it was only to be
explained as the result of a strong passion. In conversing with him she
at times displayed a singular timidity, a nervousness, a self-subdual
surprisingly unlike anything that could be expected from her. It was
true that at other moments her lover caught a gleam in her eyes, a
movement of her Jips, expressive of anything rather than diffidence, and
tending to confirm his view of her as a cunning as well as fierce
animal, but the look and tone of subjugation came often enough to make
their impression predominant. One would have said that she suffered from
jealous fears which for some reason she did not venture to utter. Now
and then he surprised her gazing at him as if in troubled apprehension,
the effect of which {}upon Mr. Snowdon was perhaps more flattering than
any other look.

``What's up, Clem?'' he inqmred, on one of these occasions. ``Are you
wondering whether I shall cut and leave you when we've had time to get
tired of each other?''

Her face was transformed; she looked at him for an instant with fierce
suspicion, then laughed disagreeably.

``We'll see about that,'' was her answer, with a movement of the head
and shoulders strongly reminding one of a lithe beast about to spring.

The necessary delay passed without accident. As the morning of the
marriage approached there was, however, a perceptible increase of
nervous restlessness in Clem. She had given up her work at Whitehead's,
and contrived to keep her future husband within sight nearly all day
long. Joseph James found nothing particularly irksome in this, for beer
and tobacco were supplied him \emph{ad libitum}, and a succession of
appetising {}meals made the underground kitchen a place of the
pleasantest associations. A loan from Mrs. Peckover had enabled him to
renew his wardrobe. When the last night arrived, Clem and her mother sat
conversing to a late hour, their voices again cautiously subdued. A
point had been for some days at issue between them, and decision was now
imperative.

``It's you as started the job,'' Clem observed with emphasis, ``an' it's
you as'll have to finish it.''

``And who gets most out of it, I'd like to know?'' replied her mother.
``Don't be such a fool! Can't you see as it'll come easier from you? A
nice thing for his mother-inlaw to tell him! If you don't like to do it
the first day, then leave it to the second, or third. But if you take my
advice, you'll get it over the next morning.''

``You'll have to do it yourself,'' Clem repeated stubbornly, propping
her chin upon her fists.

``Well, I never thought as you was such a {}frightened babby! Frightened
of a feller like him! I'd be ashamed o' myself!''

``Who's frightened? Hold your row!''

``Why, you are; what else?''

``I ain't!''

``You are!''

``I ain't I You'd better not make me mad, or I'll tell him before, just
to spite you.''

``Spite \emph{me}, you cat! What difference'll it make to me? I'll tell
you what: I've a jolly good mind to tell him myself beforehand, and then
we'll see who's spited.''

In the end Clem yielded, shrugging her shoulders defiantly.

``I'll have a kitchen-knife near by when I tell him,'' she remarked with
decision. ``If he lays a hand on me I'll cut his face open, an' chance
it!''

Mrs. Peckover smiled with tender motherly deprecation of such extreme
measures. But Clem repeated her threat, and there was something in her
eyes which guaranteed the possibility of its fulfilment.

No personal acquaintance of either the {}Peckover or the Snowdon family
happened to glance over the list of names which hung in the registrar's
office during these weeks. The only interested person who had
foreknowledge of Clem's wedding was Jane Snowdon, and Jane, though often
puzzled in thinking of the matter, kept her promise to speak of it to no
one. It was imprudence in Clem to have run this risk, but the joke was
so rich that she could not deny herself its enjoyment; she knew,
moreover, that Jane was one of those imbecile persons who scruple about
breaking a pledge. On the eve of her wedding-day she met Jane as the
latter came from Whitehead's, and requested her to call in the Close
next Sunday morning at twelve o'clock.

``I want you to see my `usband,'' she said, grinning. ``I'm sure you'll
like him.''

Jane promised to come. On the next day, Saturday, Clem entered the
registry-office in a plain dress, and after a few simple formalities
came forth as Mrs. Snowdon; her usual high colour was a trifle
diminished, and she {}kept glancing at her husband from under neiTously
knitted brows. Still the great event was unknown to the inhabitants of
the Close. There was no feasting, and no wedding-journey; for the
present Mr. and Mrs. Snowdon would take possession of two rooms on the
first floor.

Twenty-four hours later, when the bells of St. James's were ringing
their melodies before service, Clem requested her husband's attention to
something of importance she had to tell him.

Mr. Snowdon had just finished breakfast and was on the point of lighting
his pipe; with the match burning down to his fingers, he turned and
regarded the speaker shrewdly. Clem's face put it beyond question that
at last she was about to make a statement definitely bearing on the
history of the past month. At this moment she was almost pale, and her
eyes avoided his. She stood close to the table, and her right hand
rested near the bread-knife; her left held a piece of paper.

{}``What is it?'' asked Joseph James mildly.

``Go ahead, Clem.''

``You ain't bad-tempered, are you? You said you wasn't.''

``Not I! Best-tempered feller you could have come across. Look at me
smiling.''

His grin was in a measure reassuring, but he had caught sight of the
piece of paper in her hand, and eyed it steadily.

``You know you played mother a trick a long time ago,'' Clem pursued, "
when you went off an' left that child on her `ands.''

``Hollo! What about that?''

``Well, it wouldn't be nothing but fair if some one was to go an play
tricks with \emph{you}---just to pay you off in a friendly sort o'
way,---see?''

Mr. Snowdon still smiled, but dubiously.

``Out with it!'' he muttered. ``I'd have bet a trifle there was some
game on. You're welcome, old girl. Out with it!''

``Did you know as I'd got a brother in `Stralia,---him as you used to
know when you lived here before?''

{}``You said you didn't know where he was.''

``No more we do,---not just now. But he wrote mother a letter about this
time last year, an' there's something in it as I'd like you to see.
You'd better read for yourself.'' Her husband laid down his pipe on the
mantelpiece and began to cast his eye over the letter, which was much
defaced by frequent foldings, and in any case would have been difficult
to decipher, so vilely was it scrawled. But Mr. Snowdon's interest was
strongly excited, and in a few moments he had made out the following
communication:---

``I don't begin with no deering, because it's a plaid out thing, and
because I'm riting to too people at onse, both mother and Clem, and it's
so long since I've had a pen in my hand I've harf forgot how to use it.
If you think I'm making my pile, you think rong, so you've got no need
to ask me when I'm going to send money home, like you did in the last
letter. I jest keep myself and that's about all, because things ain't
what they use to be {}in this busted up country. And that remminds me
what it was as I ment to tell you when I cold get a bit of time to rite.
Not so long ago, I met a chap as use to work for somebody called
Snowdon, and from what I can make out it was Snowdon' s brother at home,
him as we use to ere so much about. He'd made his pile, this Snowdon,
you bet, and Ned Williams says he died worth no end of thousands. Not so
long before he died, his old farther from England come out to live with
him; then Snowdon and a son as he had both got drownded going over a
river at night. And Ned says as all the money went to the old bloak and
to a brother in England, and that's what he herd when he was paid off.
The old farther made traks very soon, and they sed he'd gone back to
England. So it seams to me as you ouht to find Snowdon and make him pay
up what he ose you. And I don't know as I've anything more to tell you
both, ecsep I'm working at a place as I don't know how to spell, and it
woldn't be no good if I did, because there's no saying were I {}shall be
before you could rite back. So good luck to you both, from yours truly,
W. P.''

In reading, Joseph James scratched his bald head thoughtfully. Before he
had reached the end there were signs of emotion in his projecting lower
lip. At length he regarded Clem, no longer smiling, but without any of
the wrath she had anticipated.

``Ha, ha! This was your game, was it? Well, I don't object, old
girl,---so long as you tell me a bit more about it. Now there's no need
for any more lies, perhaps you'll mention where the old fellow is.''

``He's livin' not so far away, an' Jane with him.''

Put somewhat at her ease, Clem drew her hand from the neighbourhood of
the breadknife, and detailed all she knew with regard to old Mr. Snowdon
and his affairs. Her mother had from the first suspected that he
possessed money, seeing that he paid, with very little demur, the sum
she demanded for Jane's board and lodging. True, he went to live in poor
lodgings, but that was doubtless a {}personal eccentricity. An important
piece of evidence subsequently forthcoming was the fact that in sundry
newspapers there appeared advertisements addressed to Joseph James
Snowdon, requesting him to communicate with Messrs. Percival \& Peel of
Furnivars Inn, whereupon Mrs. Peckover made inquiries of the legal firm
in question (by means of an anonymous letter), and received a simple
assurance that Mr. Snowdon Was being sought for his own advantage.

``You're cool hands, you and your mother,'' observed Joseph James, with
a certain involuntary admiration. ``This was not quite three years ago,
you say; just when I was in America. Ha---hum! What I can't make out is,
how the devil that brother of mine came to leave anything to me. We
never did anything but curse each other from the time we were children
to when we parted for good. And so the old man went out to Australia,
did he? That's a rum affair, too; Mike and he could never get on
together. Well, I suppose there's no mistake about it. I {}shouldn't
much mind if there was, just to see the face \emph{you'd} pull, young
woman. On the whole, perhaps it's as well for you that I \emph{am}
fairly good-tempered,---eh?''

Clem stood apart, smiling dubiously, now and then eyeing him askance.
His last words once more put her on her guard; she moved towards the
table again.

``Give me the address,'' said her husband.

``I'll go and have a talk with my relations. What sort of a girl's Janey
grown up,---eh?''

``If you'll wait a bit, you can see for yourself. She's goin to call
here at twelve.''

``Oh, she is? I suppose you've arranged a pleasant little surprise for
her? Well, I must say you're a cool hand, Clem. I shouldn't wonder if
she's been in the house several times since I've been here?''

``No, she hasn't. It wouldn't have been safe, you see.''

``Give me the corkscrew, and I'll open this bottle of whisky. It takes
it out of a fellow, this kind of thing. Here's to you, Mrs. Clem! Have a
drink? All right; go {}down-stairs and show your mother you're alive
still; and let me know when Jane comes. I want to think a bit."

When he had sat for a quarter of an hour in solitary reflection the door
opened, and Clem led into the room a young girl, whose face expressed
timid curiosity. Joseph James stood up, joined his hands under his
coat-tail, and examined the stranger.

``Do you know who it is?'' asked Clem of her companion.

``Your husband,---but I don't know his name.''

``You ought to, it seems to me,'' said Clem, giggling. ``Look at him.''

Jane tried to regard the man for a moment. Her cheeks flushed with
confusion. Again she looked at him, and the colour rapidly faded. In her
eyes was a strange light of painfully struggling recollection. She
turned to Clem, and read her countenance with distress.

"Well, I'm quite sure I should never have known \emph{you}, Janey," said
Snowdon, advancing.

``Don't you remember your father?''

{}Yes; as soon as consciousness could reconcile what seemed
impossibilities Jane had remembered him. She was not seven years old
when he forsook her, and a life of anything but orderly progress had
told upon his features. Nevertheless Jane recognised the face she had
never had cause to love, recognised yet more certainly the voice which
carried her back to childhood. But what did it all mean? The shock was
making her heart throb as it was wont to do before her fits of illness.
She looked about her with dazed eyes.

``Sit down, sit down,'' said her father, not without a note of genuine
feeling. ``It's been a bit too much for you,---like something else was
for me just now. Put some water in that glass, Clem; a drop of this will
do her good.''

The smell of what was offered her proved sufficient to restore Jane; she
shook her head and put the glass away. After an uncomfortable silence,
during which Joseph dragged his feet about the floor, Clem remarked:

{}``He wants you to take him home to see your grandfather, Jane. There's
been reasons why he couldn't go before. Hadn't you better go at once,
Jo?''

Jane rose and waited whilst her father assumed his hat and drew on a new
pair of gloves. She could not look at either husband or wife. Presently
she found herself in the street, walking without consciousness of things
in the homeward direction.

``You've grown up a very nice, modest girl, Jane,'' was her father's
first observation. ``I can see your grandfather has taken good care of
you.''

He tried to speak as if the situation were perfectly simple. Jane could
find no reply.

``I thought it was better,'' he continued, in the same matter-of-fact
voice, ``not to see either of you till this marriage of mine was over.
I've had a great deal of trouble in life,---I'll tell you all about it
some day, my dear,---and I wanted just to settle myself before---I
daresay you'll understand what I mean. I {}suppose your grandfather has
often spoken to you about me?''

``Not very often, father,'' was the murmured answer.

``Well, well; things'll soon be set right. I feel quite proud of you,
Janey; I do, indeed. And I suppose you just keep house for him, eh?''

``I go to work as well.''

``What? You go to work? How's that, I wonder?''

``Didn't Miss Peckover tell you?''

Joseph laughed. The girl could not grasp all these astonishing facts at
once, and the presence of her father made her forget who Miss Peckover
had become.

``You mean my wife, Janey! No, no; she didn't tell me you went to
work;---an accident. But I'm delighted you and Clem are such good
friends. Kind-hearted girl, isn't she?''

Jane whispered an assent.

``No doubt your grandfather often tells you about Australia, and your
uncle that died there?''

{}``No, he never speaks of Australia. And I never heard of my uncle.''

``Indeed? Ha-hum!''

Joseph continued his examination all the way to Hanover Street, often
expressing surprise, but never varying from the tone of aflfection and
geniality. When they reached the door of the house he said:

``Just let me go into the room by myself. I think it'll be better. He's
alone, isn't he?''

``Yes. I'll come up and sliow you the door.''

She did so, then turned aside into her own room, where she sat
motionless for a long time.
