\chapter{Phantoms}

\textsc{Mr.} and Mrs. Joseph Snowdon were now established in rooms in
Burton Crescent, which is not far from King's Cross. Joseph had urged
that Clerkenwell Close was scarcely a suitable quarter for a man of his
standing, and, though with difficulty, he had achieved thus much
deliverance. Of Clem he could not get rid,---just yet; but it was
something to escape Mrs. Peckover's superintendence. Clem herself
favoured the removal, naturally for private reasons. Thus far working in
alliance with her shrewd mother, she was now forming independent
projects. Mrs. Peckover's zeal was assuredly not disinterested, and why,
Clem mused with herself, should the fruits of strategy be shared? Her
husband's father could not, she saw every reason to believe, be much
longer for this world. How his property {}was to be divided she had no
means of discovering; Joseph professed to have no accurate information,
but as a matter of course he was deceiving: her. Should he inherit a
considerable sum, it was more than probable he would think of again
quitting his native land,---and without encumbrances. That movement must
somehow be guarded against; how, it was difficult as yet to determine.
In the next place, Jane was sure to take a large share of the fortune.
To that Clem strongly objected, both on abstract grounds and because she
regarded Jane with a savage hatred,---a hatred which had its roots in
the time of Jane's childhood, and which had grown in proportion as the
girl reaped happiness from life. The necessity of cloaking this
sentiment had not, you may be sure, tended to mitigate it. Joseph said
that there was no longer any fear of a speedy marriage between Jane and
Kirkwood, but that such a marriage would come off some day,---if not
prevented,---Clem held to be a matter of certainty. Sidney Kirkwood was
a wide-awake young man; of course he had his satisfactory reasons for
delay. Now Clem's hatred of Sidney was, from of old, only less than that
{}wherewith she regarded Jane. To frustrate the hopes of that couple
would be a gratification worth a good deal of risk.

She heard nothing of what had befallen Clara Hewett until the latter's
return home, and then not from her husband. Joseph and Scawthorne,
foiled by that event in an ingenious scheme which you have doubtless
understood (they little knowing how easily the severance between Jane
and Kirkwood might be effected), agreed that it was well to get Clara
restored to her father's household,---for, though it seemed unlikely, it
was not impossible that she might in one way or another aid their
schemes,---and on that account the anonymous letter was despatched which
informed John Hewett of his daughter's position. Between John and
Snowdon, now that they stood in the relations of master and servant,
there was naturally no longer familiar intercourse, and, in begging
leave of absence for his journey northwards, Hewett only said that a
near relative had met with a bad accident. But it would be easy, Joseph
decided, to win the man's confidence again, and thus be apprised of all
that went on. With Clem he {}kept silence on the subject; not improbably
she would learn sooner or later what had happened, and indeed, as things
now stood, it did not matter much; but on principle he excluded her as
much as possible from his confidence. He knew she hated him, and he was
not backward in returning the sentiment, though constantly affecting a
cheerful friendliness in his manner to her; after all, their union was
but temporary. In Hanover Street he was also silent regarding the
Hewetts, for there his \emph{rôle} was that of a good, simple-minded
fellow, incapable of intrigue, living for the domestic affections. If
Kirkwood chose to speak to Michael or Jane of the matter, well, one way
or another, that would advance things a stage, and there was nothing for
it but to watch the progress.

Alone all through the day, and very often in the evening, Clem was not
at all disposed to occupy herself in domestic activity. The lodgings
were taken furnished, and a bond-maid of the house did such work as was
indispensable. Dirt and disorder were matters of indifference to the
pair, who represented therein the large class occupying cheap {}London
lodgings; an impure atmosphere, surroundings more or less squalid,
constant bickering with the landlady, coarse usage of the
servant,---these things Clem understood as necessaries of independent
life, and it would have cost her much discomfort had she been required
to live in a more civilised manner. Her ambitions were essentially
gross. In the way of social advancement she appreciated nothing but an
increased power of spending money, and consequently of asserting herself
over others. She had no desire whatever to enter a higher class than
that in which she was born; to be of importance in her familiar circle
was the most she aimed at. In visiting the theatre, she did not so much
care to occupy a superior place,---indeed, such a position made her ill
at ease,---as to astonish her neighbours in the pit by a lavish style of
costume, by loud remarks implying a free command of cash, by purchase
between the acts of something expensive to eat or drink. Needless to say
that she never read anything but police news; in the fiction of her
world she found no charm, so sluggishly unimaginative was her nature.
Till of late she had {}either abandoned herself all day long to a brutal
indolence, eating rather too much, and finding quite sufficient
occupation for her slow brain in the thought of how pleasant it was not
to be obliged to work, and occasionally in reviewing the chances that
she might eventually have plenty of money and no Joseph Snowdon as a
restraint upon her; or else, her physical robustness demanding exercise,
she walked considerable distances about the localities she knew, calling
now and then upon an acquaintance.

Till of late; but a change had come upon her life. It was now seldom
that she kept the house all day; when within-doors she was restless,
quarrelsome. Joseph became aware with surprise that she no longer tried
to conceal her enmity against him; on a slight provocation she broke
into a fierceness which reminded him of the day when he undeceived her
as to his position, and her look at such times was murderous. It might
come, he imagined, of her being released from the prudent control of her
mother. However, again a few weeks and things were somewhat improved;
she eyed him like a wild beast, but {}was less frequent in her
outbreaks. Here, too, it might be that Mrs. Peekover's influence was at
work, for Clara spent at least four evenings of the seven away from
home, and always said she had been at the Close. As indifferent as it
was possible to be, Joseph made no attempt to restrain her independence;
indeed he was glad to have her out of his way.

We must follow her on one of these evenings ostensibly passed at Mrs.
Peekover's,---no, not follow, but discover her at nine o'clock.

In Old Street, not far from Shoreditch Station, was a shabby little
place of refreshment, kept by an Italian; pastry and sweet-stuff filled
the window; at the back of the shop, through a doorway on each side of
which was looped a pink curtain, a room, furnished with three
marble-topped tables, invited those who wished to eat and drink more at
ease than was possible before the counter. Except on Sunday evening this
room was very little used, and there, on the occasion of which I speak,
Clem was sitting with Bob Hewett. They had been having supper
together,---French pastry and a cup of cocoa.

She leaned forward on her elbows, and said {}imperatively, ``Tell
Pennyloaf to make it up with her again.''

``Why?''

``Because I want to know what goes on in Hanover Street You was a fool
to send her away, and you d ought to have told me about it before now.
If they was such friends, I suppose the girl told her lots o' things.
But I expect they see each other just the same. You don't suppose she
does all \emph{you} tell her?''

``I'll bet you what you like she does!'' cried Bob.

Clem glared at him.

``Oh, you an' your Pennyloaf! Likely she tells you the truth. You're so
fond of each other, ain't you! Tells you everything, does she?---and the
way you treat her!''

``Who's always at me to make me treat her worse still?'' Bob retorted
half angrily, half in expostulation.

``Well, and so I am, `cause I hate the name of her 1 I'd like to hear as
you starve her and her brats half to death. How much money did you give
her last week? Now you just tell me the truth. How much was it?''

{}``How can I remember? Three or four bob, I s'pose.''

``Three or four bob!'' she repeated, snarling. ``Give her one, and make
her live all the week on it. Wear her down! Make her pawn all she has,
and go cold!''

Her cheeks were on fire; her eyes started in the fury of jealousy; she
set her teeth together. ``I'd better do for her altogether,'' said Bob,
with an evil grin. Clem looked at him, without speaking; kept her gaze
on him; then she said in a thick voice:

``There's many a true word said in joke.'' Bob moved uncomfortably.
There was a brief silence, then the other, putting her face nearer his:

``Not just yet. I want to use her to get all I can about that girl and
her old beast of a grandfather. Mind you do as I tell you. Pennyloaf's
to have her back again, and she's to make her talk, and you're to get
all you can from Pennyloaf,---understand?''

There came noises from the shop. Three work-girls had just entered and
were buying {}cakes, which they began to eat at the counter. They were
loud in gossip and laughter, and their voices rang like brass against
brass. Clem amused herself in listening to them for a few minutes; then
she became absent, moving a finger round and round on her plate. A
disagreeable flush still lingered under her eyes.

``Have you told her about Clara?''

``Told who?''

``Who? Pennyloaf, of course.''

``No, I haven't. Why should I?''

``Oh, you're such a affectionate couple I See, you're only to give her
two shillin's next week. Let her go hungry this nice weather.''

``She won't do that if Jane Snowdon comes back, so there you're out of
it!''

Clem bit her lip.

``What's the odds? Make it up with a hit in the mouth now aud then.''

``What do you expect to know from that girl?'' inquired Bob.

``Lots o' things. I want to know what the old bloke's goin' to do with
his money, don't I? And I want to know what my beast of a `usband's got
out of him. And I want to know what that feller Kirkwood's goin' to
{}do. He'd ought to marry your sister by rights.''

``Not much fear of that now.''

``Trust him! He'll stick where there's money. See, Bob; if that Jane was
to kick the bucket, do you think the old bloke `ud leave it all to Jo?''

``How can I tell?''

``Well, look here. Supposin' he died an left most to her; an' then
supposin' \emph{she} was to go off; would Jo have all her tin?''

``Course he would.''

Clem mused, eating her lower lip.

``But supposin' Jo was to go off first, after the old bloke? Should I
have all he left?''

``I think so, but I'm not sure.''

``You think so? And then should I have all \emph{her's}? If she had a
accident, you know.''

``I suppose you would. But then that's only if they didn't make wills,
and leave it away from you.''

Clem started. Intent as she had been for a long time on the
possibilities hinted at, the thought of unfavourable disposition by will
had never occurred to her. She shook it away.

{}``Why should they make wills? They ain't old enough for that, neither
of them.''

``And you might as well say they ain't old enough to be likely to take
their hook, either,'' suggested Bob, with a certain uneasiness in his
tone.

Clem looked about her, as if her fierce eyes sought something. Her brows
twitched a little. She glanced at Bob, but he did not meet her look. ``I
don't care so much about the money,'' she said, in a lower and altered
voice. ``I'd be content with a bit of it, if only I could get rid of him
at the same time.''

Bob looked gloomy.

``Well, it's no use talking,'' he muttered.

``It's all your fault.''

``How do you make that out? It was you quarrelled first.''

``You're a liar!''

``Oh, there's no talking to you!''

He shuffled with his feet, then rose.

``Where can I see you on Wednesday morning?'' asked Clem. ``I want to
hear about that girl.''

``It can't be Wednesday morning. I tell you I shall be getting the sack
next thing; they've {}promised it. Two days last week I wasn't at the
shop, and one day this. It can't go on.''

His companion retorted angrily, and for five minutes they stood in
embittered colloquy. It ended in Bob's turning away and going out into
the street. Clem followed, and they walked westwards in silence.
Reaching City Road, and crossing to the corner where lowers St. Luke's
Hospital,---grim abode of the insane, here in the midst of London's
squalor and uproar,---they halted to take leave. The last words they
exchanged, after making an appointment, were of brutal violence.

This was two days after Clara Hewett's arrival in London, and the same
fog still hung about the streets, allowing little to be seen save the
blurred glimmer of gas. Bob sauntered through it, his hands in his
pockets, observant of nothing; now aud then a word escaped his lips,
generally an oath. Out of Old Street he turned into Whitecross Street,
whence by black and all but deserted ways,---Barbican and Long
Lane,---he emerged into West Smithfield. An alley in the shadow of
Bartholomew's Hospital brought him to a certain house; just as he was
about to knock {}at the door it opened, and Jack Bartley appeared on the
threshold. They exchanged a ``Hollo!'' of surprise, and after a
whispered word or two on the pavement, went in. They mounted the stairs
to a bedroom which Jack occupied. When the door was closed:

``Bill's got copped!'' whispered Bartley.

``Copped? Any of it on him?''

``Only the half-crown as he was pitchin', thank God! They let him go
again after he'd been to the station. It was a conductor. I'd never try
them blokes myself; they're too downy.''

``Let's have a look at 'em,'' said Bob, after musing. ``I thought myself
as they wasn't quite the reg'lar.''

As he spoke he softly turned the key in the door. Jack then put his arm
up the chimney and brought down a small tin box, soot-blackened; he
opened it, and showed about a dozen pieces of money,---in appearance
halfcrowns and florins. One of the commonest of offences against the law
in London, this to which our young friends were not unsuccessfully
directing their attention; one of the easiest to commit, moreover, for a
man with {}Bob's craft at his finger-ends. A mere question of a mould
and a pewter-pot, if one be content with the simpler branches of the
industry. ``The snyde'' or ``the queer'' is the technical name by which
such products are known. Distribution is, of course, the main
difficulty; it necessitates mutual trust between various confederates.
Bob Hewett still kept to his daily work, but gradually he was being
drawn into alliance with an increasing number of men who scorned the
yoke of a recognised occupation. His face, his clothing, his speech, all
told whither he was tending, had one but the experience necessary for
the noting of such points. Bob did not find his life particularly
pleasant; he was in perpetual fear; many a time he said to himself that
he would turn back. Impossible to do so; for a thousand reasons
impossible; yet he still believed that the choice lay with him.

His colloquy with Jack only lasted a few minutes, then he walked
homewards, crossing the Metropolitan Meat-market, going up St. John's
Lane, beneath St. John's Arch, thence to Rosoman Street and Merlin
Place, where at present he lived. All the way he pondered Clem's words.
Already their import had become {}familiar enough to lose that first
terribleness. Of course he should never take up the proposal seriously;
no, no, that was going a bit too far; but suppose Clem's husband were
really contriving this plot on his own account? Likely, very likely; but
he'd be a clever fellow if he managed such a thing in a way that did not
immediately subject him to suspicion. How could it be done? No harm in
thinking over an affair of that kind, when you have no intention of
being drawn into it yourself. There was that man at Peckham who poisoned
his sister not long ago; he was a fool to get found out in the way he
did; he might {have{{------}}}

The room in which he found Pennyloaf sitting was so full of fog that the
lamp seemed very dim; the fire had all but died out. One of the children
lay asleep; the other Pennyloaf was nursing, for it had a bad cough and
looked much like a wax doll that has gone through a great deal of
ill-usage. A few more weeks and Pennyloaf would be again a mother; she
felt very miserable as often as she thought of it, and Bob had several
times spoken with harsh impatience on the subject.

At present he was in no mood for {}conversation; to Pennyloaf's remarks
and questions he gave not the slightest heed, but in a few minutes
tumbled himself into bed. ``Get that light put out,'' he exclaimed,
after lying still for a while. Pennyloaf said she was uneasy about the
child; its cough seemed to be better, but it moved about restlessly and
showed no sign of getting to sleep.

``Give it some of the mixture, then. Be sharp and put the light out.''

Pennyloaf obeyed the second injunction, and she too lay down, keeping
the child in her arms; of the ``mixture'' she was afraid, for a few days
since the child of a neighbour had died in consequence of an overdose of
this same anodyne. For a long time there was silence in the room.
Outside, voices kept sounding with that peculiar muffled distinctness
which they have on a night of dense fog, when there is little or no
wheel-traffic to make the wonted rumbling.

``Are y'asleep?'' Bob asked suddenly.

``No.''

``There's something I wanted to tell you. You can have Jane Snowdon here
again, if you like.''

{}``I can? Really?''

``You may as well make use of her. That'll do; shut up and go to
sleep.''

In the morning Pennyloaf was obliged to ask for money; she wished to
take the child to the hospital again, and as the weather was very bad
she would have to pay an omnibus fare. Bob growled at the demand, as was
nowadays his custom. Since he had found a way of keeping his own pocket
tolerably well supplied from time to time, he was becoming so penurious
at home that Pennyloaf had to beg for what she needed copper by copper.
Excepting breakfast, he seldom took a meal with her. The easy
good-nature which in the beginning made him an indulgent husband had
turned in other directions since his marriage was grown a weariness to
him. He did not, in truth, spend much upon himself, but in his leisure
time was always surrounded by companions whom he had a pleasure in
treating with the generosity of the public-house. A word of flattery was
always sure of payment if Bob had a coin in his pocket. Ever hungry for
admiration, for prominence, he found new opportunities of gratifying his
{}taste now that he had a resource when his wages ran out. So far from
becoming freer-handed again with his wife and children, he grudged every
coin that he was obliged to expend on them. Pennyloaf's submissiveness
encouraged him in this habit; where other wives would have ``made a
row,'' she yielded at once to his grumbling and made shift with the
paltriest allowance. You should have seen the kind of diet on which she
habitually lived. Like all the women of her class, utterly ignorant and
helpless in the matter of preparing food, she abandoned the attempt to
cook anything, and expended her few pence daily on whatever happened to
tempt her in a shop, when mealtime came round. In the present state of
her health she often suffered from a morbid appetite and fed on things
of incredible unwholesomeness. Thus, there was a kind of cake exposed in
a window in Rosoman Street, two layers of pastry with half an inch of
something like very coarse mincemeat between; it cost a halfpenny a
square, and not seldom she ate four, or even six, of these squares, as
heavy as lead, making this her dinner. A cookshop within her range
exhibited at {}day great dough-puddings, kept hot by jets of steam that
came up through the zinc on which they lay; this food was cheap and
satisfying and Penny loaf often regaled both herself and the children on
thick slabs of it. Pease-pudding also attracted her she fetched it from
the pork-butcher's in a little basin, which enabled her to bring away at
the same time a spoonful or two of gravy from the joints of which she
was not rich enough to purchase a cut. Her drink was tea she had the pot
on the table all day, and kept adding hot water. Treacle she purchased
now and then, but only as a treat when her dinner had cost even less
than usual she did not venture to buy more than a couple of ounces at a
time, knowing by experience that she could not resist this form of
temptation, and must eat and eat till all was finished.

Bob flung six pence on to the table. He was ashamed of himself,---you
will not under stand him if you fail to recognise that,---but the shame
only served to make him fret under his bondage. Was he going to be tied
to Pennyloaf all his life, with a family constantly increasing
Practically he had already made {}a resolve to be free before very long;
the way was not quite clear to Inm as yet. But he went to work still
brooding; over Clem's words of the night before.

Pennyloaf let the fire go out, locked the elder child into the room for
safety against accidents, and set forth for the hospital. It rained
heavily, and the wind rendered her umbrella useless. She had to stand
for a long time at a street-corner before the omnibus came; the water
soaked into her leaky shoes, but that didn't matter; it was the child on
whose account she was anxious. Having reached her destination, she sat
for a long time waiting her turn among the numerous out-patients. Just
as the opportunity for passing into the doctor's room arrived, a
movement in the bundle she held made her look closely at the child's
face; at that instant it had ceased to live.

The medical man behaved kindly to her, but she gave way to no outburst
of grief; with tearless eyes she stared at the unmoving body in a sort
of astonishment. The questions addressed to her she could not answer
with any intelligence; several times she asked {}stupidly, ``Is she
really dead?'' There was nothing to wonder at, however; the doctor
glanced at the paper on which he had written prescriptions twice or
thrice during the past few weeks, and found the event natural
enough\ldots{}.

Towards the close of the afternoon Pennyloaf was in Hanover Street. She
wished to see Jane Snowdon, but had a fear of going up to the door and
knocking. Jane might not be at home, and, if she were, Pennyloaf did not
know in what words to explain her coming and say what had happened. She
was in a dazed, heavy, tongue-tied state indeed she did not clearly
remember how she had come thus far, or what she had done since leaving
the hospital at midday. However, her steps drew nearer to the house, and
at last she had raised the knocker,---just raised it and let it fall.

Mrs. Byass opened she did not know Penny loaf by sight. The latter tried
to say V something, but only stammered a meaningless sound; thereupon
Bessie concluded she was a beggar, and with a shake of the head shut the
door upon her.

Pennyloaf turned away in confusion and dull {}misery. She walked to the
end of the street and stood there. On leaving home she had forgotten her
umbrella, and now it was raining' heavily again. Of a sudden her need
became powerful enough to overcome all obstacles; she knew that she
\emph{must} see Jane Snowdon, that she could not go home till she had
done so. Jane was the only friend she had; the only creature who would
speak the kind of words to her for which she longed.

Again the knocker fell, and again Mrs. Byass appeared.

``What do you want? I've got nothing for you,'' she cried impatiently.

``I want to see Miss Snowdon, please, mum,---Miss Snowdon,
please''{{------}}

``Miss Snowdon? Then why didn't you say so? Step inside.''

A few moments and Jane came running downstairs.

``Pennyloaf!''

Ah! that was the voice that did good. How it comforted and blessed,
after the hospital, and the miserable room in which the dead child was
left lying, and the rainy street!
