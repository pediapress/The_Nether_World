\hypertarget{headerContainer}{}
\hypertarget{navigationHeader}{}
\protect\hypertarget{headerprevious}{}{←\href{/wiki/The_Nether_World/Volume_1/Chapter_10}{Chapter
10}}

\textbf{\protect\hypertarget{header_title_text}{}{\href{/w/index.php?title=The_Nether_World/Volume_1\&action=edit\&redlink=1}{The
Nether World/Volume 1}}} \emph{by
\href{/wiki/Author:George_Gissing}{\protect\hypertarget{header_author_text}{}{{George
Gissing}}}}\\
\protect\hypertarget{header_section_text}{}{Chapter 11}

\protect\hypertarget{headernext}{}{\href{/wiki/The_Nether_World/Volume_1/Chapter_12}{Chapter
12}→}

\hypertarget{navigationNotes}{}

\hypertarget{ws-data}{}
\protect\hypertarget{ws-article-id}{}{1807758}\protect\hypertarget{ws-title}{}{\href{/w/index.php?title=The_Nether_World/Volume_1\&action=edit\&redlink=1}{The
Nether World/Volume 1} --- \emph{Chapter
11}}\protect\hypertarget{ws-author}{}{George Gissing}

{\protect\hypertarget{234}{}{}}

{CHAPTER XI.}

A DISAPPOINTMENT.

\textsc{On} ordinary Sundays the Byasses breakfasted at ten o'clock;
this morning the meal was ready at eight, and Bessie's boisterous
spirits declared the exception to be of joyous significance. Finding
that Samuel's repeated promises to rise were the merest evasion, she
rushed into the room where he lay fly-fretted, dragged the pillows from
under his tousled head, and so belaboured him in schoolboy fashion that
he had no choice but to leap towards his garments. In five minutes he
roared down the kitchen-stairs for shaving-water, and in five minutes
more was seated in his shirt-sleeves, consuming fried bacon with
prodigious appetite. Bessie had the twofold occupation of waiting upon
him and finishing the toilet of the baby; she talked
{\protect\hypertarget{235}{}{}}incessantly and laughed with an echoing
shrillness which would have given a headache for the rest of the day to
any one of average nervons sensibility.

They were going to visit Samuel's parents, who lived at Greenwich.
Bessie had not yet enjoyed an opportunity of exhibiting her firstborn to
the worthy couple; she had, however, written many and long letters on
the engrossing subject, and was just a little fluttered with natural
anxiety lest the infant's appearance or demeanour should disappoint the
expectations she had excited. Samuel found his delight in foretelling
the direst calamities.

``Don't say I didn't advise you to draw it mild,'' he remarked whilst
breakfasting, when Bessie had for the tenth time obliged him to look
round and give his opinion on points of costume. ``Remember it was only
last week you told them that the imp had never cried since the day of
his birth, and I'll bet you three half-crowns to a bad halfpenny he
roars all through to-night.''

{\protect\hypertarget{236}{}{}}``Hold your tongue, Sam, or I'll throw
something at you!''

Samuel had just appeased his morning hunger, and was declaring that the
day promised to be the hottest of the year, such a day as would bring
out every vice inherent in babies, when a very light tap at the door
caused Bessie to abandon her intention of pulling his ears.

``That's Jane,'' she said. ``Come in!''

The Jane who presented herself was so strangely unlike her namesake who
lay ill at Mrs. Peckover's four months ago, that one who had not seen
her in the interval would with difficulty have recognised her. To begin
with, she had grown a little; only a little, but enough to give her the
appearance of her full thirteen years. Then her hair no longer straggled
in neglect, but was brushed very smoothly back from her forehead, and
behind was plaited in a coil of perfect neatness; one could see now that
it was soft, fine, mousecoloured hair, such as would tempt the fingers
to the lightest caress. No longer were her
{\protect\hypertarget{237}{}{}}limbs huddled over with a few shapeless
rags; she wore a full-length dress of quiet grey, which suited well with
her hair and the pale tones of her complexion. As for her face---oh yes,
it was still the good, simple, unremarkable countenance, with the
delicate arched eyebrows, with the diffident lips, with the cheeks of
exquisite smoothness, but so sadly thin. Here too, however, a noteworthy
change was beginning to declare itself. You were no longer distressed by
the shrinking fear which used to be her constant expression; her eyes no
longer reminded you of a poor animal that has been beaten from every
place where it sought rest and no longer expects anything but a kick and
a curse. Timid they were, drooping after each brief glance, the eyes of
one who has suffered and cannot but often brood over wretched memories,
who does not venture to look far forward lest some danger may loom
inevitable,---meet them for an instant, however, and you saw that lustre
was reviving in their still depths, that a woman's soul had begun to
manifest itself {\protect\hypertarget{238}{}{}}under the shadow of those
gently falling lids. A kind word, and with what purity of silent
gratitude the grey pupils responded! A merry word, and mark if the light
does not glisten on them, if the diffident lips do not form a smile
which you would not have more decided lest something of its sweetness
should be sacrificed.

``Now come and tell me what you think about baby,'' cried Bessie. ``Will
he do? Don't pay any attention to my husband; he's a vulgar man!''

Jane stepped forward.

``I'm sure he looks very nice, Mrs. Byass.''

``Of course he does, bless him! Sam, get your coat on, and brush your
hat, and let Miss Snowdon teach you how to behave yourself. Well, we're
going to leave the house in your care, Jane. We shall be back some time
to-morrow night, but goodness knows when. Don't you sit up for us.''

``You know where to wire to, if there's a fire breaks out in the back
kitchen,'' observed Samuel facetiously. ``If you hear footsteps
{\protect\hypertarget{239}{}{}}in the passage at half-past two to-morrow
morning, don't trouble to come down; wait till daylight to see whether
they've carried off the dresser.''

Bessie screamed with laughter.

``What a fool you are, Sam! If you don't mind, you'll be making Jane
laugh. You're sure you'll be home before dark to-morrow, Jane?''

``Oh, quite sure. Mr. Kirkwood says there's a train gets to Liverpool
Street about seven, and grandfather thought that would suit us.''

``You'll be here before eight then. Do see that your fire's out before
you leave. And you'll be sure to pull the door to? And see that the
area-gate's fastened.''

``Can't you find a few more orders?'' observed Samuel.

``Hold your tongue! Jane doesn't mind; do you, Jane? Now, Sam, are you
ready? Bless the man, if he hasn't got a great piece of bread sticking
in his whiskers! How \emph{did} it get there? Off you go!''

{\protect\hypertarget{240}{}{}}Jane followed them, and stood at the
front door for a moment, watching them as they departed.

Then she went upstairs. On the first floor the doors of the two rooms
stood open, and -the rooms were bare. The lodgers who had occupied this
part of the house had recently left; a card was again hanging in the
window of Bessie's parlour. Jane passed up the succeeding flight and
entered the chamber which looked out upon Hanover Street. The trucklebed
on which her grandfather slept had been arranged for the day some two
hours ago; Snowdon rose at six, and everthing was orderly in the room
when Jane came to prepare breakfast an hour later. At present the old
man was sitting by the open window, smoking a pipe. He spoke a few words
with reference to the Byasses, then seemed to resume a train of thought,
and for a long time there was unbroken silence. Jane seated herself at a
table, on which were a few books and writing materials. She began to
copy something, using the pen with difficulty, and taking extreme
{\protect\hypertarget{241}{}{}}pains. Occasionally her eyes wandered,
and once they rested upon her grandfather's face for several minutes.
But for the cry of a milkman or a paper-boy in the street, no sound
broke the quietness of the summer morning. The blessed sunshine, so
rarely shed from a London sky,---sunshine, the source of all solace to
mind and body,---reigned gloriously in heaven and on earth. When more
than an hour had passed, Snowdon came and sat down beside the girl.
Without speaking she showed him what she had written. He nodded
approvingly.

``Shall I say it to you, grandfather?''

``Yes.''

Jane collected her thoughts, then began to repeat the parable of the
Samaritan. From the first words it was evident that she frequently thus
delivered passages committed to memory; evident, too, that instruction
and a natural good sense guarded her against the gabbling method of
recitation. When she had finished Snowdon spoke with her for a while on
the subject of the story. In all he said there was
{\protect\hypertarget{242}{}{}}the earnestness of deep personal feeling.
His theme was the virtue of Compassion; he appeared to rate it above all
other forms of moral goodness, to regard it as the saving principle of
human life.

``If only we had pity on one another, all the worst things we suffer
from in this world would be at an end. It's because men's hearts are
hard that life is so full of misery. If we could only learn to be kind
and gentle and forgiving---never mind anything else. We act as if we
were all each other's enemies; we can't be merciful, because we expect
no mercy; we struggle to get as much as we can for ourselves and care
nothing for others. Think about it; never let it go out of your mind.
Perhaps some day it'll help you in your own life.''

Then there was silence again. Snowdon went back to his seat by the
window and relit his pipe; to muse in the sunshine seemed sufficient
occupation for him. Jane opened another book and read to herself.

In the afternoon they went out together.
{\protect\hypertarget{243}{}{}}The old man had grown more talkative. He
passed cheerfully from subject to subject, now telling a story of his
experiences abroad, now reviying recollections of London as he had known
it sixty years ago. Jane listened with quiet interest. She did not say
much herself, and when she did speak it was with a noticeable effort to
overcome her habit of diffidence. She was happy, but her nature had yet
to develop itself under these strangely novel conditions.

A little before sunset there came a knocking at the house-door. Jane
went down to open, and found that the visitor was Sidney Kirkwood. The
joyful look with which she recognised him changed almost in the same
moment; his face wore an expression that alarmed her; it was stern,
hard-set in trouble, and his smile could not disguise the truth. Without
speaking, he walked upstairs and entered Snowdon's room. To Sidney there
was always something peculiarly impressive in the first view of this
quiet chamber; simple as were its appointments, it produced a sense
{\protect\hypertarget{244}{}{}}of remoteness from the common conditions
of life. Invariably he subdued his voice when conversing here. A few
flowers such as can be bought in the street generally diffused a slight
scent through the air, making another peculiarity which had its effect
on Sidney's imagination. When Jane moved about, it was with a soundless
step; if she placed a chair or arranged things on the table, it was as
if with careful avoidance of the least noise. ``When his thoughts turned
hitherwards, Sidney always pictured the old man sitting in his familiar
mood of reverie, and Jane, in like silence, bending over a book at the
table. Peace, the thing most difficult to find in the world that Sidney
knew, had here made itself a dwelling.

He shook hands with Snowdon and seated himself. A few friendly words
were spoken, and the old man referred to an excursion they had agreed to
make together on the morrow, the general holiday.

``I'm very sorry,'' replied Kirkwood, ``but it'll be impossible for me
to go.''

{\protect\hypertarget{245}{}{}}Jane was standing near him; her
countenance fell, expressing uttermost disappointment.

``Something has happened,'' pursued Sidney, ``that won't let me go away,
even for a few hours. I don't mean to say that it would really prevent
me, but I should be so uneasy in my mind all the time that I couldn't
enjoy myself, and I should only spoil your pleasure. Of course you'll go
just the same?''

Snowdon reassured him on this point. Jane had just been about to lay
supper; she continued her task, and Sidney made a show of sharing the
meal. Soon after, as if conscious that Sidney would speak with more
freedom of his trouble but for her presence, Jane bade them good-night
and went to her own room. There ensued a break in the conversation; then
Kirkwood said, with the abruptness of one who is broaching a difficult
subject:

``I should like to tell you what it is that's going wrong with me. I
don't think any one's advice would be the least good, but it's
{\protect\hypertarget{246}{}{}}a miserable affair, and I shall feel
better for speaking about it.''

Snowdon regarded him with eyes of calm sympathy. There is a look of
helpful attention peculiar to the faces of some who have known much
suffering; in this instance, the grave force of character which at all
times made the countenance impressive heightened the effect of its
gentleness. In external matters, the two men knew little more of each
other now than after their first meeting, but the spiritual alliance
between them had strengthened with every conversation. Each understood
the other's outlook upon problems of life which are not commonly
discussed in the top rooms of lodging-houses; they felt and thought
differently at times, but in essentials they were at one, and it was the
first time that either had found such fruitful companionship.

``Did you hear anything from the Peckovers of Clara Hewett?'' Sidney
began by asking.

``Not from them. Jane has often spoken of her.''

{\protect\hypertarget{247}{}{}}Sidney again hesitated, then, from a
fragmentary beginning, passed into a detailed account of his relations
with Clara. The girl herself, had she overheard him, could not have
found fault with the way in which the story was narrated. He represented
his love as from the first without response which could give him serious
hope; her faults he dealt with not as characteristics to be condemned,
but as evidences of suffering, the outcome of cruel conditions. Her
engagement at the luncheon-bar he spoke of as a detestable slavery,
which had wasted her health and driven her in the end to an act of
desperation. What now could be done to aid her? John Hewett was still in
ignorance of the step she had taken, and Sidney described himself as
distracted by conflict between what he felt to be his duty and fear of
what might happen if he invoked Hewett's authority. At intervals through
the day he had been going backwards and forwards in the street where
Clara had her lodging. He did not think she would seek to escape from
her friends {\protect\hypertarget{248}{}{}}altogether, but her character
and circumstances made it perilous for her to live thus alone.

``What does she really wish for?'' inquired Snowdon, when there had been
a short silence.

``She doesn't know, poor girl! Eveiything in the life she has been
living is hateful to her,---everything since she left school. She can't
rest in the position to which she was born; she aims at an impossible
change of circumstances. It comes from her father; she can't help
rebelling against what seem to her unjust restraints. But what's to come
of it? She may perhaps get a place in a large restaurant,---and what
does that mean?''

He broke off, but in a moment resumed even more passionately.

``What a vile, cursed world this is, where you may see men and women
perish before your eyes, and no more chance of saving them than if they
were going down in midocean! She's only a child,---only just
seventeen,---and already she's gone through a lifetime of miseries. And
I, like a fool, I've {\protect\hypertarget{249}{}{}}often been angry
with her; I was angry yesterday. How can she help her nature? How can we
any of us help what we're driven to in a world like this? Clara isn't
made to be one of those who slave to keep themselves alive. Just a
chance of birth! Suppose she'd been the daughter of a rich man; then
everything we now call a fault in her would either have been of no
account or actually a virtue. Just because we haven't money we may go to
perdition, and comfortable people tell us we've only ourselves to blame.
Put them in our place!'

Snowdon's face had gone through various changes as Sidney flung out his
vehement words. When he spoke, it was in a tone of some severity.

`Has she no natural affection for her father? Does she care nothing for
what trouble she brings him?'

Sidney did not reply at once; as he was about to speak, Snowdon bent
forward suddenly and touched his arm.

`Let me see her. Let me send Jane to {\protect\hypertarget{250}{}{}}her
to-morrow morning, and ask her to come here. I might---I can't say---but
I might do some good.'

To this Sidney gave willing assent, but without sanguine expectation. In
further talk it was agreed between them that, if this step had no
result, John Hewett ought to be immediately informed of the state of
things.

This was at ten o'clock on Sunday evening. So do we play our
tragi-comedies in the eye of fate.

The mention of Jane led to a brief conversation regarding her before
Sidney took his leave. Since her recovery she had been going regularly
to school, to make up for the time of which she had been defrauded by
Mrs. Peckover. Her grand-father's proposal was, that she should continue
thus for another six months, after which, he said, it would be time for
her to learn a business. Mrs. Byass had suggested the choice of
artificial-flower making, to which she herself had been brought up;
possibly that would do as well as anything else.

{\protect\hypertarget{251}{}{}}``I suppose so,'' was Sidney's reluctant
acquiescence. ``Or as ill as anything else, would be a better way to put
it.''

Snowdon regarded him with unusual fixedness, and seemed on the point of
making some significant remark; but immediately his face expressed
change of purpose, and he said, without emphasis:

``Jane must be able to earn her own living.''

Sidney, before going home, walked round to the street in which he had
already lingered several times to-day, and where yesterday he had spoken
with Clara. The windows of the house he gazed at were dark.
