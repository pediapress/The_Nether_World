\part{}

\chapter{The Soup-Kitchen}

\textsc{With} the first breath of winter there passes a voice
half-menacing, half-mournful, through all the barren ways and phantom
-haunted refuges of the nether world. Too quickly has vanished the brief
season when the sky is clement, when a little food suffices, and the
chances of earning that little are more numerous than at other times;
this wind that gives utterance to its familiar warning is the
vaunt-courier of cold and hunger and solicitude that knows not sleep.
Will the winter be a hard one? It is the question that concerns this
world before all others, that occupies alike the patient workfolk who
have yet their home unbroken, the strugglers foredoomed to loss of such
scant needments as the summer gifted {\protect\hypertarget{2}{}{}}them
withal, the hopeless and the self-abandoned and the lurking creatures of
prey. To all of them the first chill breath from a lowering sky has its
voice of admonition; they set their faces; they sigh, or whisper a
prayer, or fling out a curse, each according to his nature.

And as though the strife here were not already hard enough, behold from
many corners of the land come needy emigrants, prospectless among their
own people, fearing the dark season which has so often meant for them
the end of wages and of food, tempted hither by thought that in the
shadow of palaces work and charity are both more plentiful. Vagabonds,
too, no longer able to lie about the country roads, creep back to their
remembered lairs and join the combat for crusts flung forth by casual
hands. Day after day the stress becomes more grim. One would think that
hosts of the weaker combatants might surely find it seasonable to let
themselves be trodden out of existence, and so make room for those of
more useful sinew; somehow they cling to life; so few in comparison
yield utterly. The thoughtful in the world above look about them with
contentment when carriage-ways are deep
{\protect\hypertarget{3}{}{}}with new-fallen snow. ``Good; here is work
for the unemployed.'' Ah, if the winter did but last a few months
longer, if the wonted bounds of endurance were but, by some freak of
nature, sensibly overpassed, the carriage-ways would find another kind
of {sweeping! {.~.~.}}

This winter was the last that Shooter's Gardens were destined to know.
The leases had all but run out; the middlemen were garnering their
latest profits; in the spring there would come a wholesale demolition,
and model-lodgings would thereafter occupy the site. Meanwhile the
Gardens looked their surliest; the walls stood in a perpetual black
sweat; a mouldy reek came from the open doorways; the beings that passed
in and out seemed soaked with grimy moisture, puffed into distortions,
hung about with rotting garments. One such was Mrs. Candy, Pennyloaf's
mother. Her clothing consisted of a single gown and a shawl made out of
the fragments of an old counterpane; her clothing,---with exception of
the shoes on her feet, those two articles were literally all that
covered her bare body. Rage for drink was with her reaching the final
mania. Useless to bestow anything {\protect\hypertarget{4}{}{}}upon her;
straightway it or its value passed over the counter of the beershop in
Rosoman Street. She cared only for beer, the brave, thick, medicated
draught, that was so cheap and frenzied her so speedily.

Her husband was gone for good. One choking night of November he beat her
to such purpose that she was carried off to the police-station as dead;
the man effected his escape, and was not likely to show himself in the
Gardens again. With her still lived her son Stephen, the potman. His
payment was ten shillings a week (with a daily allowance of three
pints), and he saw to it that there was always a loaf of bread in the
room they occupied together. Stephen took things with much philosophy;
his mother would, of course, drink herself to death,---what was there
astonishing in that? He himself had heart disease, and surely enough
would drop down dead one of these days; the one doom was no more to be
quarrelled with than the other. Pennyloaf came to see them at very long
intervals; what was the use of making her visits more frequent? She,
too, viewed with a certain equanimity the progress of her mother's fate.
{\protect\hypertarget{5}{}{}}Vain every kind of interposition; worse
than imprudence to give the poor creature money or money's worth. It
could only be hoped that the end would come before very long.

An interesting house, this in which Mrs. Candy resided. It contained in
all seven rooms, and each room was the home of a family; under the roof
slept twenty-five persons, men, women, and children; the lowest rent
paid by one of these domestic groups was four-and-sixpence. You would
have enjoyed a peep into the rear chamber on the ground floor. There
dwelt a family named Hope,---Mr. and Mrs. Hope, Sarah Hope, aged
fifteen, Dick Hope, aged twelve, Betsy Hope, aged three. The father was
a cripple; he and his wife occupied themselves in the picking of
rags,---of course at home,---and I can assure you that the atmosphere of
their abode was worthy of its aspect. Mr. Hope drank, but not
desperately. His forte was the use of language so peculiarly violent
that even in Shooter's Gardens it gained him a proud reputation. On the
slightest excuse he would threaten to brain one of his children, to
disembowel another, to gouge out the eyes of the
{\protect\hypertarget{6}{}{}}third. He showed much ingenuity in varying
the forms of menaced punishment. Not a child in the Gardens but was
constantly threatened by its parents with a violent death; this was so
familiar that it had lost its effect; where the nurse or mother in the
upper world cries, ``I shall scold you!'' in the nether the phrase is,
``I'll knock yer 'ed orff!'' To ``I shall be very angry with you'' in
the one sphere, corresponds in the other, ``I'll murder you!'' These are
conventions,---matters of no importance. But Mr. Hope was a man of
individuality; he could make his family tremble; he could bring lodgers
about the door to listen and admire his resources.

In another room abode a mother with four children. This woman drank
moderately, but was very conscientious in despatching her three younger
children to school. True, there was just a little inconvenience in this
punctuality of hers, at all events from the youngsters' point of view,
for only on the first three days of the week had they the slightest
chance of a mouthful of breakfast before they departed. ``Never mind,
I'll have some dinner for you,'' their parent was wont to say. Common
enough {\protect\hypertarget{7}{}{}}in the board-schools, this pursuit
of knowledge on an empty stomach. But then the end is so inestimable!

Yet another home. It was tenanted by two persons only; they appeared to
be man and wife, but in the legal sense were not so, nor did they for a
moment seek to deceive their neighbours. With the female you are
slightly acquainted; christened Sukey Jollop, she first became Mrs. Jack
Bartley, and now, for courtesy's sake, was styled Mrs. Higgs. Sukey had
strayed on to a downward path; conscious of it, she abandoned herself to
her taste for strong drink, and braved out her degradation. Jealousy of
Clem Peckover was the first cause of discord between her and Jack
Bartley; a robust young woman, she finally sent Jack about his business
by literal force of arms, and entered into an alliance with Ned Higgs, a
notorious swashbuckler, the captain of a gang of young ruffians who at
this date were giving much trouble to the Clerkenwell police. Their
specialty was the skilful use, as an offensive weapon, of a stout
leathern belt heavily buckled; Mr. Higgs boasted that with one stroke of
his belt he could, if it seemed good {\protect\hypertarget{8}{}{}}to
him, kill his man, but the fitting opportunity for this display of
prowess had not yet offered{.~.~.~.}

Now it happened that, at the time of her making Jane Snowdon's
acquaintance, Miss Lant was particularly interested in Shooter's Gardens
and the immediate vicinity. She had associated herself with certain
ladies who undertook the control of a soup-kitchen in the neighbourhood,
and as the winter advanced she engaged Jane in this work of charity. It
was a good means, as Michael Snowdon agreed, of enabling the girl to
form acquaintances among the very poorest, those whom she hoped to serve
effectively,---not with aid of money alone, but by her personal
influence. And I think it will be worth while to dwell a little on the
story of this same soup-kitchen; it is significant, and shall take the
place of abstract comment on Miss Laut's philanthropic enterprises.

The kitchen Lad been doing: successful work for some years; the society
which established it entrusted its practical conduct to very practical
people, a man and wife who were themselves of the nether world, and knew
the ways {\protect\hypertarget{9}{}{}}thereof. The ``stock'' which
formed the basis of the soup was wholesome and nutritious; the peas were
of excellent quality; twopence a quart was the price at which this fluid
could be purchased (one penny if a ticket from a member of the committee
were presented), and sometimes as much as five hundred quarts would be
sold in a day. Satisfactory enough this. When the people came with
complaints, saying that they were tired of this particular soup, and
would like another kind for a change, Mr. and Mrs. Batterby, with
perfect understanding of the situation, bade their customers ``take it
or leave it,---an' none o' your cheek here, or you won't get nothing at
all!'' The result was much good-humour all round.

But the present year saw a change in the constitution of the committee:
two or three philanthropic ladies of great conscientiousness began to
inquire busily into the working of the soup-kitchen, and they soon found
reason to be altogether dissatisfied with Mr. and Mrs. Batterby. No, no;
these managers were of too coarse a type; they spoke grossly; what
possibility of their exerting a humanising
{\protect\hypertarget{10}{}{}}influence on the people to whom they
dispensed soup? Soup and refinement must be disseminated at one and the
same time, over the same counter. Mr. and Mrs. Batterby were dismissed,
and quite a new order of things began. Not only were the ladies zealous
for a high ideal in the matter of soup-distributing; they also aimed at
practical economy in the use of funds. Having engaged a cook after their
own hearts, and acting upon the advice of competent physiologists, they
proceeded to make a ``stock'' out of sheep's and bullocks' heads;
moreover, they ordered their peas from the City, thus getting them at
two shillings a sack less than the price formerly paid by the Batterbys
to a dealer in Clerkenwell. But, alas! these things could not be done
secretly; the story leaked out; Shooter's Gardens and vicinity broke
into the most excited feeling. I need not tell you that the nether world
will consume---when others supply it---nothing but the very finest
quality of food,---that the heads of sheep and bullocks are peculiarly
offensive to its stomach,---that a saving effected on sacks of peas
outrages its dearest sensibilities. What was the result? Shooter's
Gardens, convinced {\protect\hypertarget{11}{}{}}of the fraud practised
upon them, nobly brought back their quarts of soup to the Kitchen and
with proud independence of language demanded to have their money
returned. On being met with a refusal, they---what think you?---emptied
the soup on to the floor, and went away with heads exalted.

Vast was the indignation of Miss Lant and the other ladies. ``This is
their gratitude!'' Now if you or I had been there, what an opportunity
for easing our minds!'' Gratitude, mesdames? You have entered upon this
work with expectation of gratitude?---And can you not perceive that
these people of Shooter's Gardens are poor, besotted, disease-struck
creatures, of whom---in the mass---scarcely a human quality is to be
expected? Have you still to learn what this nether world has been made
by those who belong to the sphere above it?---Gratitude, quotha?---Nay,
do \emph{you} be grateful that these hapless, half-starved women do not
turn and rend you. At present they satisfy themselves with insolence.
Take it silently, you who at all events hold some count of their dire
state; and endeavour to feed them without arousing their animosity!''

{\protect\hypertarget{12}{}{}}Well, the Kitchen threatened to be a
failure. It turned out that the cheaper peas were, in fact, of inferior
quality, and the ladies hastened to go back to the dealer in
Clerkenwell. This was something, but now came a new trouble; the
complaint with which Mr. and Mrs. Batterby had known so well how to deal
revived in view of the concessions made by the new managers. Shooter's
Gardens would have no more peas; let some other vegetable be used. Again
the point was conceded; a trial was made of barley-soup. Shooter's
Gardens came, looked, smelt, and shook their heads. ``It don't look
nice,'' was their comment; they would none of it.

For two or three weeks, just at this crisis in the Kitchen's fate, Jane
Snowdon attended with Miss Lant to help in the dispensing of the
decoction. Jane was made very nervous by the disturbances that went on,
but she was able to review the matter at issue in a far more fruitful
way than Miss Lant and the other ladies. Her opinion was not asked,
however. In the homely grey dress, with her modest, retiring manner, her
gentle, diffident countenance, she was taken by the customers for a
{\protect\hypertarget{13}{}{}}paid servant, and if ever it happened that
she could not supply a can of soup quickly enough sharp words reached
her ear. ``Now then, you gyurl there! Are you goin' to keep me all d'y?
I've got somethink else to do but stand `ere.'' And Jane, by her timid
hastening, confirmed the original impression, with the result that she
was treated yet more unceremoniously next time. Of all forms of
insolence there is none more flagrant than that of the degraded poor
receiving charity which they have come to regard as a right.

Jane did speak at length. Miss Lant had called to see her in Hanover
Street; seated quietly in her own parlour, with Michael Snowdon to
approve,---with him she had already discussed the matter,---Jane
ventured softly to compare the present state of things and that of
former winters, as described to her by various people.

``Wasn't it rather a pity,'' she suggested, ``that the old people were
sent away?''

``You think so?'' returned Miss Lant, with the air of one to whom a
novel thought is presented. ``You really think so. Miss Snowdon?''

``They got on so well with everybody,''
{\protect\hypertarget{14}{}{}}Jane continued. ``And don't you think it's
better, Miss Lant, for everybody to feel satisfied?''

``But really, Mr. Batterby used to speak so very harshly. He destroyed
their self-respect.''

``I don't think they minded it,'' said Jane, with simple good faith.
``And I'm always hearing them wish he was back, instead of the new
managers.''

``I think we shall have to consider this,'' remarked the lady,
thoughtfully.

Considered it was, and with the result that the Batterbys before long
found themselves in their old position, uproariously welcomed by
Shooter's Gardens. In a few weeks the soup was once more concocted of
familiar ingredients, and customers, as often as they grumbled, had the
pleasure of being rebuked in their native tongue.

It was with anything but a cheerful heart that Jane went through this
initiation into the philanthropic life. Her brief period of joy and
confidence was followed by a return of anxiety, which no resolve could
suppress. It was not only that the ideals to which she strove to form
herself made no genuine appeal {\protect\hypertarget{15}{}{}}to her
nature; the imperative hunger of her heart remained unsatisfied. At
first, when the assurance received from Michael began to lose a little
of its sustaining force, she could say to herself, ``Patience, patience;
be faithful, be trustful, and your reward will soon come.'' Nor would
patience have failed her had but the current of life flowed on in the
old way. It was the introduction of new and disturbing things that
proved so great a test of fortitude. Those two successive absences of
Sidney on the appointed evening were strangely unlike him, but perhaps
could be explained by the unsettlement of his removal; his manner when
at length he did come proved that the change in himself was still
proceeding. Moreover, the change affected Michael, who manifested
increase of mental trouble at the same time that he yielded more and
more to physical infirmity.

The letter which Sidney wrote after receiving Joseph Snowdon's
confidential communications was despatched two days later. He expressed
himself in carefully chosen words, but the purport of the letter was to
make known that he no longer thought of Jane save
{\protect\hypertarget{16}{}{}}as a friend; that the change in her
position had compelled him to take another view of his relations to her
than that he had confided to Michael at Danbury. Most fortunately---he
added---no utterance of his feelings had ever escaped him to Jane
herself, and henceforth he should be still more careful to avoid any
suggestion of more than brotherly interest. In very deed nothing was
altered; he was still her steadfast friend, and would always aid her to
his utmost in the work of her life.

That Sidney could send this letter, after keeping it in reserve for a
couple of days, proved how profoundly his instincts were revolted by the
difficulties and the ambiguity of his position. It had been bad enough
when only his own conscience was in play; the dialogue with Joseph,
following upon Bessie Byass's indiscretion, threw him wholly off his
balance, and he could give no weight to any consideration but the
necessity of recovering self-respect. Even the sophistry of that
repeated statement that he had never approached Jane as a lover did not
trouble him in face of the injury to his pride. Every
{\protect\hypertarget{17}{}{}}word of Joseph Snowdon's transparently
artful hints was a sting to his sensitiveness; the sum excited him to
loathing. It was as though the corner of a curtain had been raised,
giving him a glimpse of all the vile greed, the base machination,
hovering about this fortune that Jane was to inherit. Of Scawthorne he
knew nothing, but his recollection of the Peckovers was vivid enough to
suggest what part Mrs. Joseph Snowdon was playing in the present
intrigues, and he felt convinced that in the background were other
beasts of prey, watching with keen, envious eyes. The sudden revelation
was a shock from which he would not soon recover; he seemed to himself
to be in a degree contaminated; he questioned his most secret thoughts
again and again, recognising with torment the fears which had already
bidden him draw back; he desired to purify himself by some unmistakable
action.

That which happened he had anticipated. On receipt of the letter Michael
came to see him; he found the old man waiting in front of the house when
he returned to Red Lion Street after his work. The conversation that
followed {\protect\hypertarget{18}{}{}}was a severe test of Sidney's
resolve. Had Michael disclosed the fact of his private understanding
with Jane, Sidney would probably have yielded; but the old man gave no
hint of what he had done,---partly because he found it difficult to make
the admission, partly in consequence of an indecision in his own mind
with regard to the very point at issue. Though agitated by the
consciousness of suffering in store for Jane, his thoughts disturbed by
the derangement of a part of his plan, he did not feel that Sidney's
change of mind gravely affected the plan itself. Age had cooled his
blood; enthusiasm had made personal interests of comparatively small
account to him; he recognised his granddaughter's feeling, but could not
appreciate its intensity, its supreme significance. When Kirkwood made a
show of explaining himself, saying that he shrank from that form of
responsibility, that such a marriage suggested to him many and
insuperable embarrassments, Michael began to reflect that perchance this
was the just view. With household and family cares, could Jane devote
herself to the great work after the manner of his ideal? Had he not been
tempted by {\protect\hypertarget{19}{}{}}his friendship for Sidney to
introduce into his scheme what was really an incompatible element? Was
it not decidedly, infinitely better that Jane should be unmarried?

Michael had taken the last step in that process of dehumanisation which
threatens idealists of his type. He had reached at length the pass of
those frenzied votaries of a supernatural creed who exact from their
disciples the sacrifice of every human piety. Returning home, he
murmured to himself again and again, ``She must not marry. She must
overcome this desire of a happiness such as ordinary women may enjoy.
For my sake, and for the sake of her suffering fellow-creatures, Jane
must win this victory over herself.''

He purposed speaking to her, but put it off from day to day. Sidney paid
his visits as usual, and tried desperately to behave as though he had no
trouble. Could he have divined why it was that Michael had ended by
accepting his vague pretences with apparent calm, indignation, wrath,
would have possessed him; he believed, however, that the old man out of
kindness subdued what he really felt. Sidney's state was pitiable. He
knew not whether {\protect\hypertarget{20}{}{}}he more shrank from the
thought of being infected with Joseph Snowdon's baseness or despised
himself for his attitude to Jane. Despicable entirely had been his
explanations to Michael, but how could he make them more sincere? To
tell the whole truth, to reveal Joseph's tactics, would be equivalent to
taking a part in the dirty contest; Michael would probably do him
justice, but who could say how far Joseph's machinations were becoming
effectual? The slightest tinct of uncertainty in the old man's thought,
and he, Kirkwood, became a plotter like the others, meeting mine with
countermine.

``There will be no possibility of perfect faith between men until there
is no such thing as money! H'm, and when is that likely to come to
pass?''

Thus he epigrammatised to himself one evening, savagely enough, as with
head bent forward he plodded to Red Lion Street. Some one addressed him;
he looked up and saw Jane. Seemingly it was a chance meeting, but she
put a question at once almost as though she had been waiting for him.
``Have you seen Pennyloaf lately, Mr. Kirkwood?''

{\protect\hypertarget{21}{}{}}Pennyloaf? The name suggested Bob Hewett,
who again suggested John Hewett, and so Sidney fell upon thoughts of
some one who two days ago had found a refuge in John's home. To Michael
he had said nothing of what he knew concerning Clara, a fresh occasion
of uneasy thought. Bob Hewett---so John said---had no knowledge of his
sister's situation, otherwise Pennyloaf might have come to know about
it, and in that case, perchance, Jane herself. Why not? Into what a
wretched muddle of concealments and inconsistencies and insincerities
had he fallen!

``It's far too long since I saw her,'' he replied, in that softened tone
which he found it impossible to avoid when his eyes met Jane's.

She was on her way home from the soupkitchen, where certain occupations
had kept her much later than usual; this, however, was far out of her
way, and Sidney remarked on the fact, perversely, when she had offered
this explanation of her meeting him. Jane did not reply. They walked on
together, towards Islington.

``Are you going to help at that place all the winter?'' he inquired.

{\protect\hypertarget{22}{}{}}``Yes; I think so.''

If he had spoken his thought, he would have railed against the
soup-kitchen and all that was connected with it. So far had he got in
his revolt against circumstances; Jane's ``mission'' was hateful to him;
he could not bear to think of her handing soup over a counter to ragged
wretches.

``You're nothing like as cheerful as you used to be,'' he said,
suddenly, and all but roughly.

``Why is it?''

What a question! Jane reddened as she tried to look at him with a smile;
no words would come to her tongue.

``Do you go anywhere else, besides to---to that place?''

Not often. She had accompanied Miss Lant on a visit to some people in
Shooter's Gardens. Sidney bent his brows. A nice spot. Shooter's
Gardens.

``The houses are going to be pulled down, I'm glad to say,'' continued
Jane. ``Miss Lant thinks it'll be a good opportunity for helping a few
of the families into better lodgings. We're going to buy furniture for
them,---so many have as good as none at all, you
{\protect\hypertarget{23}{}{}}know. It'll be a good start for them,
won't it?''

Sidney nodded. He was thinking of another family who already owed their
furniture to Jane's beneficence, though they did not know it.

``Mind you don't throw away kindness on worthless people,'' he said
presently.

``We can only do our best, and hope they'll keep comfortable for their
own sakes.''

``Yes, yes. Well, I'll say good-night to you here. Go home and rest; you
look tired.''

He no longer called her by her name. Tearing himself away, with a last
look, he raged inwardly that so sweet and gentle a creature should be
condemned to such a waste of her young life.

Jane had obtained what she came for. At times the longing to see him
grew insupportable, and this evening she had yielded to it, going out of
her way in the hope of encountering him as he came from work. He spoke
very strangely. What did it all mean, and when would this winter of
suspense give sign of vanishing before sunlight?
