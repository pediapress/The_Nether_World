\chapter{Glimpses of the Past}

\textsc{Sidney Kirkwood} had a lodging in Tysoe Street, ClerkenwelL It
is a short street, which, like so many in London, begins reputably and
degenerates in its latter half. The cleaner end leads into Wilmington
Square, which consists of decently depressing houses occupied in the
main, as the lower windows and front-doors indicate, by watchmakers,
working jewellers, and craftsmen of allied pursuits. The open space,
grateful in this neighbourhood, is laid out as a garden, with trees,
beds, and walks. Near the iron gate, which, for certain hours in the
day, gives admission, is a painted notice informing the public that, by
the grace of the Marquis of Northampton, they may here take their ease
on condition of good behaviour; to children
{\protect\hypertarget{120}{}{}}is addressed a distinct warning that
``This is not a playing ground.'' From his window Sidney had a good view
of the square. The house in which he lived was of two storeys; a brass
plate on the door showed the inscription, ``Hodgson, Dial Painter.'' The
window on the ground-floor was arched, as in the other dwellings at this
end of the street, and within stood an artistic arrangement of wax fruit
under a glass shade, supported by a heavy volume of Biblical appearance.
The upper storey was graced with a small iron balcony, on which
straggled a few flower-pots. However, the exterior of this abode was, by
comparison, promising; the curtains and blinds were clean, the step was
washed and whitened, the brass plate shone, the panes of glass had at
all events acquaintance with a duster. A few yards in the direction away
from the square, and Tysoe Street falls under the dominion of dry-rot.

It was not until he set forth to go to work next morning that Sidney
called to mind his conversation with Jane. That the child should
{\protect\hypertarget{121}{}{}}have missed by five minutes a meeting
with some one who perchance had the will and the power to befriend her,
seemed to him, in his present mood, merely an illustration of a vice
inherent in the nature of things. He determined to look in at the
public-house of which she had spoken, and hear for himself what manner
of man had made inquiries for people named Snowdon. The name was not a
common one; it was worth while to spend a hope or two on the chance of
doing Jane a kindness. Her look and voice when he bade her be of good
courage had touched him. In his rejected state, he felt that it was
pleasant to earn gratitude even from so humble a being as the Peckovers'
drudge.

His workshop, it has been mentioned, was in St. John's Square. Of all
areas in London thus defined, this square of St. John is probably the
most irregular in outline. It is cut in two by Clerkenwell Road, and the
buildings which compose it form such a number of recesses, of abortive
streets, of shadowed alleys, that from no point of the Square can
{\protect\hypertarget{122}{}{}}anything like a general view of its
totality be obtained. The exit from it on the south side is by St.
John's Lane, at the entrance to which stands a survival from a buried
world---the embattled and windowed archway which is all that remains
above ground of the great Priory of St. John of Jerusalem. Here dwelt
the Knights Hospitallers, in days when Clerkenwell was a rural parish,
distant by a long stretch of green country from the walls of London. But
other and nearer memories are revived by St. John's Arch. In the rooms
above the gateway dwelt, a hundred and fifty years ago, one Edward Cave,
publisher of the \emph{Gentleman's Magazine}, and there many a time has
sat a journeyman author of his, by name Samuel Johnson, too often
\emph{impransus}. There it was that the said Samuel once had his dinner
handed to him behind a screen, because of his unpresentable costume,
when Cave was entertaining an aristocratic guest. In the course of the
meal, the guest happened to speak with interest of something he had
recently read by an obscure Mr. Johnson;
{\protect\hypertarget{123}{}{}}whereat there was joy behind the screen,
and probably increased appreciation of the un-wonted dinner. After a
walk amid the squalid and toil-infested ways of Clerkenwell, it
impresses one strangely to come upon this monument of old time. The
archway has a sad, worn, grimy aspect. So closely is it packed in among
buildings which suggest nothing but the sordid struggle for existence,
that it looks depressed, ashamed, tainted by the ignobleness of its
surroundings. The wonder is that it has not been swept away, in
obedience to the great law of traffic and the spirit of the time.

St. John's Arch had a place in Sidney Kirkwood's earliest memories. From
the window of his present workshop he could see its grey battlements,
and they reminded him of the days when, as a lad just beginning to put
questions about the surprising world in which he found himself, he used
to listen to such stories as his father could tell him of the history of
Clerkenwell. Mr. Kirkwood occupied part of a house in St. John's Lane
{\protect\hypertarget{124}{}{}}not thirty yards from the Arch; he was a
printers' roller maker, and did but an indifferent business. A year
after the birth of Sidney, his only child, he became a widower. An
intelligent, warm-hearted man, the one purpose of his latter years was
to realise such moderate competency as should place his son above the
anxieties which degrade. The boy had a noticeable turn for drawing and
colouring; at ten years old, when (as often happened) his father took
him for a Sunday in the country, he carried a sketch-book and found his
delight in using it. Sidney was to be a draughtsman of some kind;
perhaps an artist, if all went well. Unhappily things went the reverse
of well. In his anxiety to improve his business, Mr. Kirkwood invented a
new kind of ``composition'' for printers' use; he patented it, risked
capital upon it, made in a short time some serious losses. To add to his
troubles, young Sidney was giving signs of an unstable character; at
fifteen he had grown tired of his drawing, wanted to be this, that, and
the other thing, {\protect\hypertarget{125}{}{}}was self-willed, and
showed no consideration for his father's difficulties. It was necessary
to take a decided step, and, though against his will, Sidney was
apprenticed to an uncle, a Mr. Roach, who also lived in Clerkenwell, and
was a working jeweller. Two years later, the father died, all but
bankrupt. The few pounds realised from his effects passed into the hands
of Mr. Roach, and were soon expended in payment for Sidney's board and
lodging.

His bereavement possibly saved Sidney from a young-manhood of
foolishness and worse. In the upper world a youth may ``sow his wild
oats'' and have done with it; in the nether, ``to have your fling'' is
almost necessarily to fall among criminals. The death was sudden; it
affected the lad profoundly, and filled him with a remorse which was to
influence the whole of his life. Mr. Roach, a thick-skinned and rather
thick-headed person, did not spare to remind his apprentice of the most
painful things wherewith the latter had to reproach himself. Sidney bore
it, {\protect\hypertarget{126}{}{}}from this day beginning a course of
self-discipline of which not many are capable at any age, and very few
indeed at seventeen. Still, there had never been any sympathy between
him and his uncle, and before very long the young man saw his way to
live under another roof and find work with a new employer.

It was just after leaving his uncle's house that Sidney came to know
John Hewett; the circumstances which fostered their friendship were such
as threw strong light on the characters of both. Sidney had taken a room
in Islington, and two rooms on the floor beneath him were tenanted by a
man who was a widower and had two children. In those days, our young
friend found much satisfaction in spending his Sunday evenings on
Clerkenwell Green, where fervent, if ungrammatical, oratory was to be
heard, and participation in debate was open to all whom the spirit
moved. One whom the spirit did very frequently move was Sidney's
fellowlodger; he had no gift of expression
{\protect\hypertarget{127}{}{}}whatever, but his brief stammering
protests against this or that social wrong had such an honest, indeed
such a pathetic sound, that Sidney took an opportunity of walking home
with him and converting neighbourship into friendly acquaintance. John
Hewett gave the young man an account of his life. He had begun as a
lath-render; later he had got into cabinet-making, started a business on
his own account and failed. A brother of his, who was a builder's
foreman, then found employment for him in general carpentry on some new
houses; but John quarrelled with his brother, and after many
difficulties fell to the making of packing-cases; that was his work at
present, and with much discontent he pursued it. John was curiously
frank in owning all the faults in himself which had helped to make his
career so unsatisfactory. He confessed that he had an uncertain temper,
that he soon became impatient with work ``which led to nothing,'' that
he was tempted out of his prudence by anything which seemed to offer ``a
better start.'' With all these admissions, he
{\protect\hypertarget{128}{}{}}maintained that he did well to be angry.
It was wrong that life should be so hard; so much should not be required
of a man. In body he was not strong; the weariness of interminable days
over-tried him and excited his mind to vain discontent. His wife was the
only one who could ever keep him cheerful under his lot, and his wedded
life had lasted but six years; now there was his lad Bob and his little
girl Clara to think of, and it only made him more miserable to look
forward and see them going through hardships like his own. Things were
wrong somehow, and it seemed to him that "if only we could have
universal suffrage{{------}}"

Sidney was only eighteen, and strong in juvenile Radicalism, but he had
a fund of common sense, and such a conclusion as this of poor John's
half astonished, half amused him. However, the man's personality
attracted him; it was honest, warm-hearted, interesting; the logic of
his pleadings might be at fault, but Sidney sympathised with him, for
all that. He too felt that "things were
{\protect\hypertarget{129}{}{}}wrong somehow," and had a pleasure in
joining the side of revolt for revolt's sake.

Now in the same house with them dwelt a young woman of about nineteen
years old; she occupied a garret, was seldom seen about, and had every
appearance of being a simple, laborious girl, of the kind familiar
enough as the silent victims of industrialism. One day the house was
thrown into consternation by the news that Miss Barnes---so she was
named---had been arrested on a charge of stealing her employer's goods.
It was true, and perhaps the best way of explaining it will be to
reproduce a newspaper report which Sidney Kirkwood thereafter preserved.

"On Friday, Margaret Barnes, nineteen, a single woman, was indicted for
stealing six jackets, value £5, the property of Mary Oaks, her mistress.
The prisoner, who cried bitterly during the proceedings, pleaded guilty.
The prosecutrix is a single woman, and gets her living by mantle-making.
She engaged the prisoner to do what is termed `finishing off,' that is,
making the button holes and sewing {\protect\hypertarget{130}{}{}}on the
buttons. The prisoner was also employed to fetch the work from the
warehouse, and deliver it when finished. On September 7th, her mistress
sent her with the six jackets, and she never returned. Sergeant Smith, a
detective, who apprehended the prisoner, said he had made inquiries in
the case, and found that up to this time the prisoner had borne a good
character as an honest, hard-working girl. She had quitted her former
lodgings, which had no furniture but a small table and a few rags in a
corner, and he discovered her in a room which was perfectly bare. Miss
Oaks was examined, and said the prisoner was employed from nine in the
morning to eight at night. The Judge: How much did you pay her per week?
Miss Oaks: Four shillings. The Judge: Did you give her her food? Miss
Oaks: No; I only get one shilling each for the jackets myself when
completed. I have to use two sewing-machines, find my own cotton and
needles, and I can, by working hard, make two in a day. The Judge said
it was a sad state of things. The
{\protect\hypertarget{131}{}{}}prisoner, when called upon, said she had
had nothing to eat for three days, and so gave way to temptation, hoping
to get better employment. The Judge, while commiserating with the
prisoner, said it could not be allowed that distress should justify
dishonesty, and sentenced the prisoner to six weeks' imprisonment."

The six weeks passed, and about a fortnight after that, John Hewett came
into Sidney's room one evening with a strange look on his face. His eyes
were very bright, the hand which he held out trembled.

``I've something to tell you,'' he said.

``I'm going to get married again.''

``Really? Why, I'm glad to hear it!''

``And who do you think? Miss Barnes.''

Sidney was startled for a moment. John had had no acquaintance with the
girl prior to her imprisonment. He had said that he should meet her when
she came out and give her some money, and Sidney had added a
contribution. For a man in Hewett's circumstances this latest step was
{\protect\hypertarget{132}{}{}}somewhat astonishing, but his character
explained it.

``I'm goin' to marry her,'' he exclaimed excitedly, ``and I'm doin' the
right thing! I respect her more than all the women as never went wrong
because they never had occasion to. I'm goin' to put her as a mother
over my children, and I'm goin' to make a happier life for her. She's a
good girl, I tell you. I've seen her nearly every day this fortnight; I
know all about her. She wouldn't have me when I first asked her,---that
was a week ago. She said no; she'd disgrace me. If you can't respect her
as you would any other woman, never come into my lodging!''

Sidney was warm with generous glow. He wrung Hewett's hand and stammered
incoherent words.

John took new lodgings in an obscure part of Clerkenwell, and seemed to
have become a young man once more. His complaints ceased; the energy
with which he went about his work was remarkable. He said his wife
{\protect\hypertarget{133}{}{}}was the salvation of him. And then befell
one of those happy chances which supply mankind with instances for its
pathetic faith that a good deed will not fail of reward. John's brother
died, and bequeathed to him some four hundred pounds. Hereupon, what
must the poor fellow do but open workshops on his own account, engage
men, go about crying that his opportunity had come at last. Here was the
bit of rock by means of which he could save himself from the sea of
competition that had so nearly whelmed him. Little Clara, now eleven
years old, could go on steadily at school; no need to think of how the
poor child should earn a wretched living. Bob, now thirteen, should
shortly be apprenticed to some better kind of trade. New rooms were
taken and well furnished. Maggie, the wife, could have good food, such
as she needed in her constant ailing, alas! The baby just born was no
longer a cause of anxious thought, but a joy in the home. And Sidney
Kirkwood came to supper as soon as the new rooms were in order, and his
bright, {\protect\hypertarget{134}{}{}}manly face did every one good to
look at. He still took little Clara upon his knee. Ha! there would come
a day before long when he would not venture to do that, and then
perhaps---perhaps! What a supper that was, and how smoothly went the
great wheels of the world that evening!

One baby, two babies, three babies; before the birth of the third,
John's brow was again clouded, again he had begun to rail and fume at
the unfitness of things. His business was a failure, partly because he
dealt with a too rigid honesty, partly because of his unstable nature,
which left him at the mercy of whims and obstinacies and airy projects.
He did not risk the ordinary kind of bankruptcy, but came down and down,
until at length he was the only workman in his own shop; then the shop
itself had to be abandoned; then he was searching for some one who would
employ him.

Bob had been put to the die-sinker's craft; Clara was still going to
school, and had no thought of earning a livelihood,---ominous
{\protect\hypertarget{135}{}{}}state of things. When it shortly became
clear even to John Hewett that he would wrong the girl if he did not
provide her with some means of supporting herself, she was sent to learn
``stamping'' with the same employer for whom her brother worked. The
work was light; it would soon bring in a little money. John declared
with fierceness that his daughter should never be set to the usual
needle-slavery, and indeed it seemed very unlikely that Clara would ever
be fit for that employment, as she could not do the simplest kind of
sewing. In the meantime the family kept changing their abode, till at
length they settled in Mrs. Peckover's house. All the best of their
furniture was by this time sold; but for the two eldest children, there
would probably have been no home at all. Bob, aged nineteen, earned at
this present time a pound weekly; his sister, an average of thirteen
shillings. Mrs. Hewett's constant ill-health (the result, doubtless, of
semi-starvation through the years of her girlhood), would have excused
defects of housekeeping; but {\protect\hypertarget{136}{}{}}indeed the
poor woman was under any circumstances incapable of domestic management,
and therein represented her class. The money she received was wasted in
comparison with what might have been done with it. I suppose she must
not be blamed for bringing children into the world when those already
born to her were but half-clothed, half-fed; she increased the sum total
of the world's misery in obedience to the laws of the Book of Genesis.
And one virtue she had which compensated for all that was lacking,---a
virtue merely negative among the refined, but in that other world the
rarest and most precious of moral distinctions,---she resisted the
temptations of the public-house.

This was the story present in Sidney Kirkwood's mind as often as he
climbed the staircase in Clerkenwell Close. By contrast, his own life
seemed one of unbroken ease. Outwardly it was smooth enough. He had no
liking for his craft, and being always employed upon the meaningless
work which is demanded {\protect\hypertarget{137}{}{}}by the rich
vulgar, he felt such work to be paltry and ignoble; but there seemed no
hope of obtaining better, and he made no audible complaint. His wages
were considerably more than he needed, and systematically he put money
aside each week.

But this orderly existence concealed conflicts of heart and mind which
Sidney himself could not have explained, could not lucidly have
described. The moral shock which he experienced at his father's death
put an end to the wanton play of his energies, but it could not ripen
him before due time; his nature was not of the sterile order common in
his world, and through passion, through conflict, through endurance, it
had to develop such maturity as fate should permit. Saved from
self-indulgence, he naturally turned into the way of political
enthusiasm; thither did his temper point him. With some help---mostly
negative---from Clerkenwell Green, he reached the stage of confident and
aspiring Radicalism, believing in the perfectibility of man, in human
brotherhood, in {\protect\hypertarget{138}{}{}}---anything you like that
is the outcome of a noble heart sheltered by ignorance. It had its turn,
and passed.

To give place to nothing very satisfactory. It was not a mere
coincidence that Sidney was going through a period of mental and moral
confusion just in those years which brought Clara Hewett from childhood
to the state of woman. Among the acquaintances of Sidney's boyhood there
was not one but had a chosen female companion from the age of fifteen or
earlier; he himself had been no exception to the rule in his class, but
at the time of meeting with Hewett he was companionless, and remained
so. The Hewetts became his closest friends; in their brief prosperity he
rejoiced with them, in their hardships he gave them all the assistance
to which John's pride would consent; his name was never spoken among
them but with warmth and gratitude. And of course the day came to which
Hewett had looked forward---the day when Sidney could no longer take
Clara upon his knee and stroke her {\protect\hypertarget{139}{}{}}brown
hair and joke with her about her fits of good and ill humour. Sidney
knew well enough what was in his friend's mind, and, though with no
sense of constraint, he felt that this handsome, keen-eyed, capricious
girl was destined to be his wife. He liked Clara; she always attracted
him and interested him; but her faults were too obvious to escape any
eye, and the older she grew, the more was he impressed and troubled by
them. The thought of Clara became a preoccupation, and with the love
which at length he recognised there blended a sense of fate fulfilling
itself. His enthusiasms, his purposes, never defined as education would
have defined them, were dissipated into utter vagueness. He lost his
guiding interests, and found himself returning to those of boyhood. The
country once more attracted him; he took out his old sketch-books,
bought a new one, revived the regret that he could not be a painter of
landscape. A visit to one or two picture-galleries, and then again
profound discouragement, recognition of the fact that
{\protect\hypertarget{140}{}{}}he was a mechanic and never could be
anything else.

It was the end of his illusions. For him not even passionate love was to
preserve the power of idealising its object. He loved Clara with all the
desire of his being, but could no longer deceive himself in judging her
character. The same sad clearness of vision affected his judgment of the
world about him, of the activities in which he had once been zealous, of
the conditions which enveloped his life and the lives of those dear to
him. The spirt of revolt often enough stirred within him, but no longer
found utterance in the speech which brings relief; he did his best to
dispel the mood, mocking at it as folly. Consciously he set himself the
task of becoming a practical man, of learning to make the best of life
as he found it, of shunning as the fatal error that habit of mind which
kept John Hewett on the rack. Who was he that he should look for
pleasant things in his course through the world? ``We are the lower
orders; we are the working-classes,'' he said
{\protect\hypertarget{141}{}{}}bitterly to his friend, and that seemed
the final answer to all his aspirations.

~

This was a dark day with him. The gold he handled stung him to hatred
and envy, and every feeling which he had resolved to combat as worse
than profitless. He could not speak to his fellow-workmen. From morning
to night it had rained. St. John's Arch looked more broken-spirited than
ever, drenched in sooty moisture.

During the dinner-hour, he walked over to the public-house of which Jane
had spoken, and obtained from the barman as full a description as
possible of the person he hoped to encounter. Both then and on his
return home in the evening he shunned the house where his friends dwelt.

It came round to Monday. For the first time for many months he had
allowed Sunday to pass without visiting the Hewetts. He felt that to go
there at present would only be to increase the parents' depression by
his own low spirits. Clara had left them now,
{\protect\hypertarget{142}{}{}}however, and if he still stayed away, his
behaviour might be misinterpreted. On returning from work, he washed,
took a hurried meal, and was on the point of going out, when some one
knocked at his door. He opened, and saw an old man who was a stranger to
him.
